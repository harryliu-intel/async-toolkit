\documentclass{article}
\usepackage{graphicx}
\usepackage{epsf}

\usepackage{floatflt}
\usepackage{fancyhdr}
\usepackage{array}

\usepackage{sectsty}
%\allsectionsfont{\mdseries\sffamily}
\allsectionsfont{\mdseries\sc}

\usepackage{caption}
\captionsetup{margin=2pc,font=small,labelfont=bf}

%%%%%%%%%%%%%%Mika's figure macro%%%%%%%%%%%%%%%%%%%%%%%%%%%
\def\listcaption_ins_epsfig#1#2#3#4{
  \begin{figure}[!tbph!]
  \bigskip
  \begin{center}
  \includegraphics[width=#1]{#2}
  \end{center}
  \caption[#4]{#3}\label{fig:#2}
  \bigskip
  \end{figure}
}

\def\ins_epsfig#1#2#3{
  \begin{figure}[!tbph!]
  \bigskip
  \begin{center}
  \includegraphics[width=#1]{#2}
  \end{center}
  \caption{#3}\label{fig:#2}
  \bigskip
  \end{figure}
}

\def\rotins_epsfig_listcaption#1#2#3#4#5{
  \begin{figure}[!tbph!]
  \bigskip
  \begin{center}
  \includegraphics[angle=#4,width=#1]{#2}
  \end{center}
  \caption[#5]{#3}\label{fig:#2}
  \bigskip
  \end{figure}
}

%%%%%%%%%%%%%%%%%%%%%%%%%%%%%%%%%%%%%%%%%%%%%%%%%%%%%%%%%%%%%%%%%%%%%%
\oddsidemargin=0.15in
\evensidemargin=0.15in
\topmargin=0.2in
\textwidth=6.2truein


\pagestyle{fancy}
\lhead{\scriptsize\bfseries\sffamily DRAFT---INTEL CONFIDENTIAL---DRAFT}
\chead{}\rhead{\thepage}
\lfoot{}\cfoot{}\rfoot{}
\renewcommand{\headrulewidth}{0pt}
%%%%%%%%%%%%%%%%%%%%%%%%%%%%%%%%%%%%%%%%%%%%%%%%%%%%%%%%%%%%%%%%%%%%%%

\title{Functional Modeling of Digital Networking Hardware: \\
  Challenges, Opportunities, and Perspectives}
\author{Mika Nystr\"om \\ {\tt mika.nystroem@intel.com} }
%\date{January 22, 2018}
\date{\today}

\begin{document}

\maketitle
\parindent=0pt
\parskip=1.5ex

\arraycolsep=1.4pt\def\arraystretch{1.5}

\begin{abstract}
  Abstract.
\end{abstract}

\tableofcontents
%\listoffigures
%\listoftables

\newpage

\section{Introduction}

At the first VLSI conference, held in Pasadena in January of 1979,
Gordon Moore delivered a challenge to the assembled engineers and
academics: build a complex (large) integrated circuit whose complexity
(size) was attributable to functions other than memory.
Moore wrote, ``$\ldots$if the semiconductor industry had a
million-transistor technology like VLSI, I'm not so sure it would know
what to do with it.  Besides products containing memory devices, it
isn't clear what future electronic products that take advantage of
VLSI will be.''\cite{vlsi2}  He identified the problems of interconnection and
product definition as key to be solved to make use of large numbers of
transistors on a single dice, arguing that the increasing design costs
related to larger designs may even keep up with the ability of the
fabs to deliver larger chips, with the exception of memory chips.  He
closed, ``In fact, unless we address and solve these problems, as we
look back on the VLSI era, we may only be able to say, 'Thanks for the
memories.' ''

While certainly we can argue that Moore's direst predictions of an
insurmountable logjam in the design process for large silicon devices
did not come true, to some extent the same problems are with us today
as 40 years ago.  It remains extremely difficult to do anything with a
large piece of silicon but to put either (a)~memories or (b)~some
highly iterated design (e.g., GPUs, Xeon Phi) on it.  Indeed at the same conference where
Moore spoke, Carver Mead elaborated by saying that the future
challenges were to be met either with ``A. Design methodologies to
manage complexity'' or ``B. Architecture of ultra concurrent
machines.''\cite{mead} Iterated-hardware designs  are a
case of B; design methodologies to manage complexity remain a pressing
issue even today.


\section{Structure and Approach}

We take meeting Gordon Moore's 1979 challenge as the goal of our work.

In this document, we will first, in section~\ref{sec:context}, explore
the context of Moore's idea and how it has been addressed by different
groups.  We will then explore the historical perspective, mostly from
a software point of view, in section~\ref{sec:historical}.  After
that, we will state our central thesis in section~\ref{sec:our-view},
where we will attempt to succinctly identify the path we advocate.  In
following sections, we explore some of the shortcomings of alternative
views, some of the history of how we came to our view
(section~\ref{sec:our-history}), and some of the topics relating to
our current practices, our current customers, and some of the major
challenges we are currently dealing with
(section~\ref{sec:challenges}).  We will finish with a number of
sections, starting with section~\ref{sec:general-principles} and to
the end of the document, giving prescriptions and other consequences
of our view of the task at hand.  Among the closing appendices is an
extensive bibliography.


\section{Context, Motivation, and Goals}\label{sec:context}

In our team, we feel that Gordon Moore's 1979 challenge to the digital
VLSI industry has only been imperfectly addressed, especially at large
companies such as Intel.  Development costs for new products are
astronomically high, and schedules are only met by making extremely
generous allowances for development time: new products, even those
that are broadly derivative of prior products, take several years from
start of engineering work until they are available in the marketplace.
As a result, planning new products is extremely risky, and programs
that fail often fail spectacularly, are cancelled at a late stage of
development, or retargeted several times, at immense cost both in
repeated development and opportunity costs from being late-to-market.
Finally, reuse of design work in our industry is often abysmally poor:
it often seems to take as long (or possibly longer) to use already
validated RTL as it does to write things from scratch.  In short,
resource (time and manpower) efficiency in digital hardware design is
low.  Even the projects that have been executed well are not always
effective in all areas: manpower, product quality (bugs, performance),
time to design, etc.

We are aware of examples from industry as well as from academia that
show that it is possible to design sophisticated hardware systems
quickly and with small, effective teams\cite{hot21164,hotalta}.  However, these
examples do seem to be the exception and not the rule.

We are interested in exploring why some groups and some approaches
have been more effective at design efficiency than others.  In this
document, we attempt to synthesize a technical strategy that strives
to use ideas that we have found to be useful in the past.  Some of the
ideas are very old (older than any of the authors), whereas some of
the ideas are newer.  Very little we propose is our own original
ideas; rather, the ideas have been tried and tested successfully by
ourselves or others in some conext.  The interactions between the
different ideas can be difficult to manage: sometimes, concepts that
appear to be pedestrian and secondary turn out to be difficult to
escape, for example, because they are built deep into computer-aided
design (CAD) tools used across the industry (and it is {\em
  emphatically not\/} our goal to rewrite every CAD tool in
existence).  As we shall see, experience in our sister field of
software engineering, where the problems of complexity are similar to
ours but clearer and easier to define, will give us a great deal of
guidance.

Our goal here is to describe a path towards streamlining the design
process, starting from the concept of {\em Functional Modeling,} that
is, a high-level software description of the hardware we are building.
At its most rudimentary, a functional model is a software simulation
of the hardware design.  This is clearly not a new idea\cite{soul},
but we have found that the structure that is present in such a
simulator can be exploited to guide a large fraction of the entire
design process---that is, it is profitable to see the functional
modeling effort not as a byproduct of or a minor stepping stone in the
hardware design effort but instead as a scaffolding upon which the
entire hardware design effort can be erected.  Indeed, this idea is to
a large extent present in contemporary commercial hardware-description
languages (HDLs) such as SystemVerilog\cite{sv}, but while these HDLs
in theory can be used as described, it is in practice very difficult
to do so, for a large number of reasons, which we will examine in more
detail on its own, in section~\ref{sec:sv_shortcomings}.  


\section{Historical Perspective}\label{sec:historical}

Concerns regarding design complexity in engineered systems are not
unique to the hardware business.  Similar concerns have arisen as
pressing across many engineering industries, starting in the early
20th century.  Especially after the Second World War, the dazzling technologies
developed in a few years of engineering frenzy during that war started
to be applied in civilian life, and the issue of complexity became so prominent
as to be a concern at every level of business and government; the
interplay between complex engineered systems is in some ways part of
the general issues relating to management of complex organizations,
another thread of 20th-century business development.\cite{org-man}

{\em Brief aside.} Our concern here is not to deal directly with
management structure and other management issues, but it is important
to understand that issues relating to team structure, recruiting,
communications within teams, etc., are never far away when we discuss
engineering methodology. at least in a commercial environment.  We won't fully
be able to avoid those issues here, try as we may.  {\em End brief aside.}

\subsection{Software engineering and the Software Crisis}

Perhaps the purest instance of design complexity's becoming
overwhelming was observed in the 1960s in the work of our software colleagues,
namely in the then-nascent field of software engineering.  Electrical
engineers, hardware designers, civil engineers, can always attempt to
shift the blame of difficulties to physical issues: ``the technical
problems we are trying to solve are just so difficult and so novel
that solutions are bound to be a little bit tricky and complex---it's
not our fault.''  In software, this is not so, since the entire design
of a software system is ``synthetic'' (although software people can
and do blame the hardware designers for providing infrastructure that is
difficult to work with).  This situation has been known as ``the
software crisis'' at least since 1968.  Edsger Dijkstra put it plainly
in his 1972 Turing Lecture,

``in those days [1950s] one often encountered the naive expectation that, once
more powerful machines were available, programming would no longer be
a problem, for then the struggle to push the machine to its limits
would no longer be necessary and that was all what programming was
about, wasn't it? But in the next decades something completely
different happened: more powerful machines became available, not just
an order of magnitude more powerful, even several orders of magnitude
more powerful. But instead of finding ourselves in the state of
eternal bliss of all programming problems solved, we found ourselves
up to our necks in the software crisis! How come?

\ldots

the major cause is... that the machines have become several orders of
magnitude more powerful! To put it quite bluntly: as long as there
were no machines, programming was no problem at all; when we had a few
weak computers, programming became a mild problem, and now we have
gigantic computers, programming had become an equally gigantic
problem. \ldots as the power of available
machines grew by a factor of more than a thousand, society's ambition
to apply these machines grew in proportion, and it was the poor
programmer who found his job in this exploded field of tension between
ends and means. The increased power of the hardware, together with the
perhaps even more dramatic increase in its reliability, made solutions
feasible that the programmer had not dared to dream about a few years
before. And now, a few years later, he had to dream about them and,
even worse, he had to transform such dreams into reality! Is it a
wonder that we found ourselves in a software crisis?''\cite{ewd340}

These developments in software are exactly analogous (replacing the word ``machines'' with ``fabrication technologies'' and ``programmer'' with ``circuit designer'') to another of
Mead's observations at the 1979 conference: ``VLSI is a statement
about system complexity, not about transistor size or circuit
performance.  VLSI defines a technology capable of creating systems so
complicated that coping with the raw complexity overwhelms all other
difficulties.''\cite{mead}  This situation has long since come to pass, but the recognition
that complexity is {\em the\/} problem in system design is still not as
clear for us hardware engineers as it is for the software engineers.  We
conclude that the best way we know to tackle design complexity in hardware
design (at least without inventing completely new technologies) is to apply,
judiciously, techniques from the software world to the hardware design process.

\subsection{The response of software engineering to the software crisis}\label{sec:sw-crisis-responses}

Software leaders in the late 1960s and early 1970s responded to the software
crisis in several ways.  The most obvious was a strong emphasis on structured
development of software, as in the following list of technologies.

\begin{itemize}
\item Structured programming (Dijkstra)\cite{structured}

\item Object-oriented programming (O.-J. Dahl's SIMULA67)\cite{simula}

\item Modular programming (Wirth's Modula)\cite{pascal,modula}

\item Structured concurrent programming (Hoare's monitors\cite{monitors}, as applied in Brinch Hansen's Concurrent Pascal\cite{concurrent-pascal})

\item Communicating Sequential Processes (CSP)\cite{ref:csp}
\end{itemize}

At the same time that these techniques were being developed, a strong effort to
apply mathematical techniques to verify the correctness of the programs being
written was underway.

\begin{itemize}

\item Floyd-Hoare logic (Floyd's application to flowcharts\cite{floyd}, Hoare's to programs\cite{hoare-logic})

\item Lamport logic (generalization of Hoare logic to concurrent programs)\cite{lamport-logic}

\item Invariant-based correctness proofs (e.g., the classic solution by Dekker to the synchronization problem\cite{dekker})

\item Abstract data types (ADTs)\cite{adt-liskov}

\item Temporal logic and model checking\cite{browne,pneuli}
  
\end{itemize}

\subsection{Model checking}\label{sec:model-checking}

We will now discuss model checking, not because it is a central topic
of our further discussions but because it is a technology with a special
relationship to hardware design, and we can draw general conclusions from
some aspects of this relationship.

In the  second list above, model checking stands out, for two reasons: it is
the only technique on the list that is limited to bounded-size (in
time and space) systems, and it was from the start developed in the
context of hardware verification (probably because of its limitation
to bounded-size systems).  Apart from this, all the listed techniques
were first used in software design and verification before being
applied to hardware design---some of the techniques have never been
fully applied to hardware design.  Nevertheless, in daily practice as
of today (2018) the formal techniques are probably more widely
familiar among hardware practitioners than among software
practitioners.  This is likely because of the perceived higher costs
of failure in hardware design than in software.

Model checking is an instructive example for discussing techniques for
managing design complexity.  There is no question that the tools used
in model checking (SAT solvers\cite{knuth-satisfiability}, BDD packages\cite{ref:bryant}, etc., and commercial
packages such as JasperGold) can achieve results in terms of finding
programming errors in RTL code in ways that were not possible before.
Yet model checking is not at all a panacea: seen over the entirety of
a SoC program, it is rare that the deployment of model checking makes
a significant impact on staffing costs or development schedules
(although one can certainly argue that the presence of model-checking
tools ensures higher-quality end results).  The reason for this is
simple: model checking deals with design problems ``in the small,'' on
the level of a single finite-state machine (FSM), or very few such machines.  Because model
checking executes code until it reaches a fixed point, it
fundamentally cannot deal with systems that can have unbounded
behavior.  A hardware designer now objects: all hardware systems are finite,
therefore that should be good enough.  The problem is that even a very small
system can have such complex behavior that it is ``as good as unbounded'' (no
existing system can verify properties across its entire behavior).

As an example, the author has written an example SystemVerilog code that is
parameterized in size.  Since a model checker relies on direct
execution of the code, the size parameter must be bound {\em before\/}
attempting a model check.  With the parameter implying 16 free
variables, JasperGold succeeds in validating the design in a few
seconds.  With the size parameter implying 64 free variables, a
JasperGold run of several days shows no sign of approaching a proof.
Of course, the Verilog code can be proved correct ``on the back of a
napkin''\ldots (See appendix~\ref{apdx:corners}.)

This discussion gets us closer to the problem we are trying to solve.  We must
realize that design complexity is a disease, not a symptom.  We do not
seek to enrich those that peddle design-automation tools to ameliorate the
symptoms of design complexity; instead we must attack the disease of design
complexity directly, by removing the complexity itself.  One of the obvious
challenges with this approach is that it likely leads to customer-visible
specification changes (that is, where a specification is bad, it must be changed,
not adapted to).

If it appears that the problem of late-bound parameters (size
parameters) is what makes model-checking insufficient for an effective
and reusable proof and, perhaps, some better method of checking
program correctness will appear, then the reader has misunderstood the
point of this example.  Model checking is not the enemy, and the
source of the problem is not with the proof technique! As the example
in appendix~\ref{apdx:corners} shows, even a very small and simple
program may have behavior that is in practice indistinguishable from
being unbounded.  (This observation can be credited to Marvin
Minsky in 1967\cite{minsky}.)  No proof technique can on its own
overcome this issue; that the program is correct is, {\it without
  further information,} simply beyond human (or mechanical)
understanding.

Of course, as we know, the example program is easily provable correct.  But it
is provable only because the technique used to generate it is known, and the
design exploits a classic theorem in discrete mathematics.

The fact that the correctness of the example program (or another one
of similar complexity) is on its own beyond mechanical and human
understanding is a simple consequence of what is known in computability
theory as the ``Halting Problem,'' shown by Turing in 1936--1937 to be unsolvable\cite{turing}.

\subsection{The finite and specific size of hardware designs is not a safe haven}\label{sec:finite}

That the halting problem has immediate and practical consequences can
be seen from a study of the so-called ``Busy Beaver'' problem, which
is outside of the scope of this discussion, except as to the fact that
the busy beavers are examples of very simple machines with very
complicated behaviors\cite{busy-beaver}.

{\em Straw man argument.} any piece of hardware is of finite extent (well,
once the size parameters are bound, and we do not care about hardware
with unbound size parameters, because we will only build hardware with
bound sizes [in reality, we already take issue with this restriction]).  Therefore
hardware can be validated by considering the entire state space of the
hardware.  In software, we are, in contrast, often interested in having software
that can run on hardware the characteristics of which (the size, in particular) we do
not know.  Therefore, assuming static sizes to the software data structures is less
useful. {\em End straw man argument.}

To look at things from an extreme
perspective, if the known universe were considered to be a computer
useful for a single task, in its history it could have performed no
more than about $10^{120}$ elementary
operations on that task\cite{computational-universe}---a number much smaller than
the number of states available to even a fairly small digital circuit.  Therefore,
the argument that hardware design is somehow different from software design
because of the finiteness of (a specific piece of) hardware, as opposed to the unboundedness (of a piece of portable-to-future-systems software) is futile.

\subsection{Summary}\label{sec:invariant-summary}

In summary, then, the issue with proving the correctness of the given program is not
mainly with the proof algorithm but rather that the programmer has neglected to state
the theorem he used when he developed the program.
For a discussion on why programmers are negligent in their program description, see appendix~\ref{apdx:annotations}.

The solution to the halting problem is to use the loop invariant and
variant function\cite{semantics-pred-calc}.  But these cannot in general be derived from the
program text: they must be stated by the programmer as
``predocumentation.''

The fact remains that more than eighty years of work in theoretical
computer science shows that methods that attempt to find bugs in
already designed systems presented without design invariants lead to
intractable problems for anything but the most trivial of examples and
that this approach is generally doomed to failure.  Industrial practice
is to concentrate on more or less haphazard approaches to program
testing, which, no matter how sophisticated they appear, can only explore
infinitesimal parts of the state space and are often no better.
Therefore, the cardinal rule is: {\em instead of expending inordinate
  effort on removing bugs from a designed system, expend that effort on
  developing the design in such a way that there are fewer bugs in it in
  the first place.}\cite{ewd209}

\section{Our View of Functional Modeling}\label{sec:our-view}

We now get to the central thesis of our work.

Our view is to adopt functional modeling, i.e., the development of an executable
model of our hardware architecture, as a central description of the architecture.
The goal is to use a functional-modeling approach as a means of expressing
architectural intent and to have this model be written in a way that it is
immediately useful to a large number of consumers within (and possibly outside)
the company.  In an ideal world, the functional model would capture every
structural relationship present in the design.

As a central unifying formal description of the design, the idea is that the
functional model can replace a large amount of duplicated work, as it is common
in the hardware-design world to generate a large number of independent models for
a single hardware design.  Our idea is to represent all these models in a format
that is manipulable enough to serve the needs formerly served by each of the many
models used.

Since no commercial or even research-grade tools are able to represent all the views
necessary for a large product design in a commercial environment (some, such as CAST,
come close), it follows that an amount of software development has to be performed
by our team.  It would defeat the purpose to farm this work out on other teams,
except in small, well-defined chunks: part of our goal is to put the needs of
architects and engineering teams in a position where they can drive not only
the hardware-design effort but also the design of the entire development process.

Some of the concepts mentioned here (e.g., ``design'', ``consumers'') are somewhat
vague and will be elucidated in later sections.



\section{Shortcomings of SystemVerilog as a Functional Modeling Environment}\label{sec:sv_shortcomings}

SystemVerilog\cite{sv} is the de-facto standard HDL today in industry.
A large number of CAD tools vendors support the various facilities
available in SystemVerilog (to a greater or lesser extent).  Obviously
CAD tools vendors are well aware of the costs of developing hardware,
and indeed, SystemVerilog today incorporates a large number of
facilities for parameterization of designs, specifying assertions and
other properties, and even for developing sophisticated
object-oriented test benches.

Yet, SystemVerilog is as a design description not part of the solution to high
development costs and slipping schedules: in fact, we would say it is a big part
of the cause.

One of the biggest problems with SystemVerilog is that once design data
has been entered into the SystemVerilog language, it becomes very difficult
to extract it from the design again.  The language is obscenely complex in
definition, and no usable parsers (to our knowledge) are available that are not
directly coupled to a tool vendor's (usually quite expensive) CAD tool.  As
a result, only what the tool vendors have decided is achievable is achievable.
It is extremely difficult to innovate or to simplify in the environment.  Straightforward
ideas, such as cosimulation, become difficult and expensive to implement, requiring
special-purpose tools, special-purpose libraries, special-purpose training, and even
special-purpose personnel.  It becomes completely impossible to pursue a unified
design methodology.

From the larger point of view of the electronics industry, it becomes unclear
who really is to do the innovation in SystemVerilog.  The hardware designers
have the cutting-edge problems, and are the ones that know what the next generation
of hardware might look like.  But they are effectively prevented from extending
the tools in that direction; instead this is the task of the specialized EDA
vendors, who are less interested in next year's cutting edge tools than they are
in next year's bestselling tools.  And in many cases, EDA vendors appear to
have discovered that it is better to sell ongoing treatments than it is to sell
cures.

We are not the first to point out shortcomings in contemporary common
practice\cite{must-change}.

\section{A Brief History of Functional Modeling Frameworks in the Intel Switching Team}
\label{sec:our-history}

\subsection{Early history: Caltech research group}\label{sec:caltech}

The history of functional modeling as used by our team can be traced
back to the work in asynchronous circuit design by A.~J.~Martin at
Caltech, starting in the 1980s, and others.  These practitioners
followed the developments in software engineering in the 1960s and
1970s described in section~\ref{sec:sw-crisis-responses} and applied
those software techniques to hardware design, i.e., ``programming in
silicon.''  Martin in particular realized that the techniques for
concurrency pioneered by Dijkstra, Hoare, and Brinch Hansen could be
taken to a logical extreme, where the ultimate view is of the
communicating processes being very small circuits, possibly as small
as a single transistor.  This viewpoint leads naturally to
speed-independent~\cite{de_muller} or, more accurately, quasi
delay-insensitive circuit design~\cite{qdi}.

The view of hardware as concurrent programming in silicon has
far-reaching implications.  While many critics of the approach have
fixated on the challenges in commercializing {\em asynchronous
  circuits,} we believe that the implications of the viewpoint in
terms of high-level system design are at least as great, with much
less controversy regarding tools, implementability, etc.  However,
the difference in viewpoint is immense.

The concurrent-programming view of asynchronous silicon development
involves a number of concepts (beyond clockless operation) that are relatively foreign to
practitioners of mainstream design methodology:
\begin{itemize}

\item Top-down design by decomposition and successive refinement.

\item Correctness by construction (through invariant-based reasoning
  applied to the {\em decomposition steps,} i.e., by an argument
  ``lifted'' from the level of the design to the level of the design
  methodology itself).

\item (Corollary of the above) traceable sequence of more-and-more decomposed designs from a specification to a final implementation.

\item Universal use of handshake protocols at every level of the design.

\item Latency-insensitive design style.

\end{itemize}

Of course, not every item in this list has been perfectly achieved in
every design.  In particular, the decomposition techniques used have
generally been more ambitious (or haphazard?) than for them to be
amenable to formal proof techniques.  It is not clear that there exist
formalisms able to fully handle the very sophisticated reasoning required to prove
the correctness of the decomposition techniques.  Nevertheless, the
ambitions are clear and undoubtedly quite different from what most
hardware designers are familiar with.  (Although some of these techniques
have started to become used in ``normal'' synchronous design, in particular,
latency-insensitive design.)

One of the great challenges in adopting a decomposition-driven
methodology within standard EDA tools is that SystemVerilog has no
support for this approach (see section~\ref{sec:sv_shortcomings}).
However, design by refinement is not unusual in industry (although it
is almost never done to the extent implied here).  The current industrial
solution is to have about two models, one in SystemVerilog RTL and the
other in a SystemVerilog validation framework such as UVM.  These two
models are stitched together using an elaborate object-oriented
framework and then the models can be run and compared.  This procedure
is extremely labor-intensive and costly and accounts for a large fraction
of the development costs of a new design.  In comparison, it is possible
to deploy the multi-level Martin method in a way that accomplishes more
thorough verification with vastly less human labor.



\subsection{Fulcrum years}\label{sec:fulcrum-cast}

During the years that our group was part of Fulcrum Microsystems, the
design approach followed the Martin approach, with the {\em CAST\/}\cite{rajit_cast}
language used to describe hierarchy, decomposition, and refinement.
Custom tools were used to perform cosimulation between arbitary
(refinement) levels of description and of almost arbitrary subsystems.

It is important to realize that in this view, the higher-level model
of the hardware was not really an independent model but the specification
itself, where the specification was expressed in an executable form
(namely, in a CSP-derived language, variously called CHP (Caltech) or CSP (Fulcrum)).

\subsection{Alta/Red Rock Canyon/Highland Park}\label{sec:alta-rrc-hlp}

Starting with the Alta (Canyon) (FM6000) program\cite{hotalta}, customer demand led
to the creation of a C ``white model'' of the design.  This was
continued into Red Rock Canyon (RRC) (FM10000).  Starting with RRC,
design validation, in the normal industrial sense, could use the white
model to do cosimulation and OVM scoreboarding.  While certainly a
great advance over not having the white model available, it is clear
from the above discussion that many ways this was a big step back for
the methodology: the validation could only occur at a few carefully
(and manually) chosen cut points, and only against a single other
model (the white model) rather than the full sequence of
transformations available in Alta and earlier programs.

Highland Park (HLP) continued the approach used in RRC but with more
systematic test plans and in general better (earlier) execution.
Another important point to consider is that HLP's software drivers
were (are being) developed using the same interfaces used within the
white model for access to configuration and status registers (CSRs).  Also, HLP was modeled in Simics (see
section~\ref{sec:simics}), where the Simics model was written from
scratch in Simics DML, but using the HLP white model as a guide.

In programs up to and including HLP, the representation of the CSRs was as
a flat chunk of memory coupled with a descriptive set of memory macros,
allowing registers and register fields to be read out of the flat memory.

\subsection{Mount Steller/Mount Evans}

For staffing reasons, the Mount Steller (MST) team decided to modify
their approach to CSRs in order that it could more closely match
Simics DML directly.  Since CSRs are required also to be accessed by address
(by, e.g., a remote host), the translation of data in this model had to
proceed ``backwards'' from HLP practice: operations on flat memory had
to be converted to operations on DML data structures.


%%%%%%%%%%%%%%%%%%%%%%%%%%%%%%%%%%%%%%%%%%%%%%%%%%%%%%%%%%%%%%%%%%%%%%

\section{Current Practice of Functional Modeling in the Intel Switching Team}

The current design of the functional model synthesizes work from
Highland Park and Mount Steller.  It is written in a combination of
programming languages, and there is an effort to study functional modeling
in Scala.  (See section~\ref{sec:challenges}.)

%%%%%%%%%%%%%%%%%%%%%%%%%%%%%%%%%%%%%%%%%%%%%%%%%%%%%%%%%%%%%%%%%%%%%%

\section{Consumers of Functional Modeling of the Intel Switching Team}\label{sec:consumers}

Before embarking on the functional modeling effort for the current
program, we investigated the uses of functional modeling in previous
programs (in particular Highland Park) and set out some goals for improvement.

\subsection{Design validation}

In our current view, as described in section~\ref{sec:alta-rrc-hlp},
functional modeling is a key part of digital design validation.  For
this to be effective, bit accurate modeling of a usefully large subset
of the chip's behaviors must be accomplished.  We have followed prior programs
in not attempting to model congestion and queuing, because doing so and maintaining
bit accuracy would necessitate a cycle-accurate model of the design, which stands
in contradition to the latency-insensitive design style we are advocating.

In general, design validation frameworks such as OVM and UVM have great difficulty
with nondeterminism.  This is a serious shortcoming of these frameworks, because
nondeterminism is an extremely useful tool in formal specifications.\cite{discipline}

In any case, design validation teams can use the functional model in several ways,
to approach the CAST-based design style of times past.

\subsubsection{Full system}

The ``full system'' simulation model of a chip design (or large
subsystem) is a simulation of the entire chip (or large subsystem), where the ``full''
system usually involves a design at a level where it would be directly interacted
with from system software.  This is the most basic form of modeling that the functional
modeling framework can achieve.

\subsubsection{Partial system}\label{sec:partial-system-validation}

The functional modeling framework can also be used to model smaller
subsystems, e.g., at the level of the independently designed RTL
blocks of the design.  These subsystems have to be developed in a way
that the RTL interfaces at that level match the more abstract
interfaces in the functional model.  Because of the different tools
and programming languages involved, and the need for automating the tasks
of setting up the details of these simulations, generating the necessary
glue code for these types of simulations is a challenging software engineering
task, which is still in progress as of this writing.


\subsection{``Shift-left'' software strategy}\label{sec:shift-left}

``Shift left'' is what we call the technique E.~W.~Dijkstra described having
practiced around 1955\cite{ewd1308}: ``the practice of programming for
nonexistent machines.''  However, the quality of today's
specifications is not as high as they were in those days, and accordingly,
we do not expect our software teams to program against a written,
nonexecutable specification.  Instead, we use the functional model as
an executable specification against which the system programmers may
develop their system software (long before the hardware has been
manufactured, or for that matter, even long before it has been
designed).

From the preceding paragraph it is clear that the quality of the functional
model must be very high if this approach is to work---it must be at least as
high as the quality of a well-written programmer's reference manual.  

\subsection{Architectural exploration}

Architectural exploration is an obvious use of functional modeling resources.  As
functional modeling is closely tied to architecture in the first place, and the amount
of work required to adjust the functional model is very small, there is great potential
for collaborative development of new hardware and software features in this area.


%%%%%%%%%%%%%%%%%%%%%%%%%%%%%%%%%%%%%%%%%%%%%%%%%%%%%%%%%%%%%%%%%%%%%%

\subsection{Documentation}

The functional model can be used for documentation in two ways: either
documentation can be embedded in it, either explicitly or implicitly in the
structure of the model.  The functional model can also be used itself as documentation.

\subsubsection{Using FM to generate documentation (``Literate programming'')}

{\em Literate programming\/} is the name given to a type of program
development pioneeered by D.~E.~Knuth in 1984 that allows a program to be
seen either as an executable program or as a publishable
document~\cite{literate-programming}.  Literate programming normally implies
the ability to rearrange code fragments almost arbitrarily in the published
form of the program, but language-specific
software-documentation systems such as javadoc can be seen as a weaker
version of this concept (usually without the ability to rearrange code fragments
in arbitrary orders).

The literate programming concept is straightforward to incorporate in
functional modeling: either documentation can be made an output of the
functional model (this is done as a matter of course to some extent),
or the language-agnostic literate-programming tools developed by Knuth
and others can be applied to the functional model.

All the approaches of literate programming (including systems such as javadoc) involve using program comments as
documentation and generating markup or typesetter controls from the program comments,
which are written in specific formats.

\subsubsection{Using FM as documentation}\label{sec:fm-spec}

A more interesting approach to documentation, which derives from the
Martin synthesis method, is to see the functional model itself as a
form of documentation.  There is an important distinction here against
the classic view of literate programming: in literate programming,
documentation is comments in code.  In the Martin view (which we share
to a large extent) the functional model's executable text is itself
the documentation.  One may view this as an instance of the somewhat
loosely defined concept ``self-documenting code'' (which sometimes
really just means that the programmer couldn't be bothered to write
down the necessary comments for understanding).  But at the same time,
we need to remember that the functional model does not exist in a vacuum.
In the Martin method, a block of code can be the implementation of a higher-level
specification and the specification for a lower-level implementation.  In that sense,
when reading the lower-level implementation, the middle-level code can fruitfully
be approached both as specification and documentation for the lower-level code.

There are important caveats here regarding program invariants (see
section~\ref{sec:invariant-summary}) and ``predocumentation'' (see
appendix~\ref{apdx:annotations}).  The point here being that one
cannot entirely escape the need for auxiliary information, such as
program invariants, by using the successive-refinement method of program
development.  In fact, consider the case of introducing a loop to make
concrete some (more abstract) atomic notion.  In order to be able to show
that the introduced loop in fact implements the atomic higher-level notion,
the loop invariant of the new lower-level loop is necessarily required and
must be stated, ideally in a machine-processable form, or otherwise as
a human-readable program comment.

In our work, this idea of ``functional model as specification'' is central to our work,
because we have found that text documents written in English are neither
easy to use (owing to their size) nor are they specific enough about technical
details (because there are no automated checks to guard against vagueness,
ambiguity, or inconsistency of English documentation).  The alternative we
are following is to use a rudimentary English description of the system at
a high level and fill in the details of the architecture with the functional model.
This in some sense reverses the roles of modeler and hardware architect: the modeler
takes on the role of the hardware architect, whereas the hardware architect is left
to document the model.  Our goal is that the formal, functional model will be
the ``golden'' architectural model.

%%%%%%%%%%%%%%%%%%%%%%%%%%%%%%%%%%%%%%%%%%%%%%%%%%%%%%%%%%%%%%%%%%%%%%

\subsubsection{Architectural model vs. microarchitectural model}

A challenge for functional modeling is to decide the subblocks of the
model.  Design validation pressures are that the functional model
match the hardware microarchitecture to the extent possible, so that
validation of subblocks can proceed using the functional model.  (See
section~\ref{sec:partial-system-validation}.)

The introduction of microarchitectural requirements into the functional
model sets up challenges for the functional modeling team.  The purpose
of the functional model is to serve as a specification for the design, therefore
it cannot be counted on to be elaborated at the microarchitectural level (since
it by definition precedes the microarchitecture).

The solution to the problem here is to allow multiple implementations
to exist in the functional model.  The Fulcrum CAST tools already
permitted this (see section~\ref{sec:fulcrum-cast}).  In this case it
is unobjectionable for there to exist a high-level architectural model
together with more elaborate microarchitectural models.  It is
important to realize that what is advocated here is not a morphing of
a single architectural model into a microarchitectural model; rather
what we wish to do is maintain a clear sequence of models that proceed
from initial specification to final implementation.  A number of
challenges exist here, not the least of which is that the initial
specifications are likely to be highly nondeterministic.  The
challenges have mostly been met within the CAST framework, so there is an
existing prototype.

If CAST cannot be used, for whatever reason, the progression of models
can be captured using object-oriented programming.  The sequence of
refinements would in this case be captured as a sequence of object
types (a.k.a.\ ``classes'').  Desire for formal verification suggests
the use of the easy-to-handle formats described in
section~\ref{sec:input} for the program text.


\subsection{High-level synthesis specification}

One potential area of use for the formal functional model is to use it as a specification
for high-level synthesis (HLS) directly to gates.  One challenge with this approach is that
it would put very high demands on the quality of the functional model, because it makes
regular ``design validation'' impossible, since there would no longer be two
independently developed models that can be checked against each other.  However, this
is not an insurmountable obstacle, and indeed, the fact that traditional techniques
have multiple models for the same object is precisely the problem we were trying to
solve in the first place.  Instead of validating the designs by comparing them
against each other (which we know is difficult and frequently impossible to do
well) we will need to develop languages and methods for describing the architectural
intent and validating that that intent is captured in the design.  In the long run,
this is likely to be a more fruitful way of developing our designs in any case.

In the short term, the desire to support HLS is a challenge because HLS implies
a need for describing detailed functional behavior.  Our preference is always not
to tie ourselves to specific proprietary tools and methods, so to do this in a way that
is adaptable to different HLS approaches will likely require some careful engineering.  See section~\ref{sec:programming-languages}.

\subsection{Security modeling}

\subsection{High-level formal validation}

\subsection{Simics}\label{sec:simics}

What's special about this?

\subsection{Emulation}

\subsection{FM team itself}

\subsection{Internal vs. external customers of FM}

%%%%%%%%%%%%%%%%%%%%%%%%%%%%%%%%%%%%%%%%%%%%%%%%%%%%%%%%%%%%%%%%%%%%%%

\section{Some Major Challenges to Functional Modeling}\label{sec:challenges}

The general goal of our functional modeling effort is to collect
design collateral (for both software and hardware designs) in formats that
make design automation (again, both software and hardware) easy or at least
possible.  This suggests using sophisticated software-engineering environments.

For many years (since at least the 1980s) it has been obvious that the
software-engineering tools that are the most popular are not the most
convenient to use when designing new systems.  Unfortunately (and
somewhat unexpectedly) the explosion of open-source software in the
decades since does not appear to have improved the situation.  We
still live in a world where low-level errors such as memory
corruption, deadlocks, lack of reasonable support for concurrency,
lack of expressive type systems, lack of name space isolation, and
many other similar problems, are still the norm, both in commercial
software development and in open-source projects.  This is even though
software systems have existed since at least the early 1970s that have
demonstrated how to avoid these issues.  Pascal\cite{pascal},
Modula\cite{modula}, Concurrent Pascal\cite{hansen-book},
Modula-3\cite{spwm3}, and Java\cite{java} are all by now decades old
and have demonstrated how to avoid many of the software-engineering
deficiencies of C and C++, yet nothing of the sort is widely used
today in our field.  In fact, even the use of C for modeling hardware
is in some quarters still considered quite novel, as there remain
groups that simply do not produce software-based models of their designs
and instead proceed directly to synthesizable RTL from English-language
specification documents.

The application of strict coding standards can somewhat alleviate the
problems of using antiquated software-development environments, but
the application of such coding standards requires careful code review
and strict management practices, which are also costly measures.  They also
tend to require superhuman discipline to follow consistently.

\subsection{Choice of programming tools (including programming languages)}\label{sec:programming-languages}

Our general direction is to work towards using the best technologies
for the task at hand.  A large part of the functional modeling team is
working towards demonstrating effective functional modeling in the
Scala programming language\cite{scala,functional-scala}.  This effort
is still somewhat experimental: from prior programs the hardware
modeling was performed in the C programming language, which suffers
from all the shortcomings mentioned above.  The main line model
currently is written mostly in C but with infrastructure in
Modula-3\cite{spwm3} and Scheme\cite{scheme,r4rs}.  The mixture of
programming languages causes some headaches (see
section~\ref{sec:agnostic}) but is mostly unavoidable and in any case
we believe its benefits far outweigh its drawbacks.


\subsection{Language agnostic approaches}\label{sec:agnostic}

We advocate language-agnostic approaches.  There are several reasons
for this.  Certain interfaces need to be exported in specific
programming languages (usually C or Verilog) to be useful to some of the
consumers identified in section~\ref{sec:consumers}.  Another reason is that
it is difficult and expensive to rewrite everything in a functional model
from the ground up; in other words, components already exist or are easily
generated in specific programming languages.  Even when new code is written,
staffing constraints may demand that certain technologies be used and others avoided.

Our solution to this is to expend significant effort on inter-language
translation software.  Such software already exists in the forms of
SWIG\cite{swig}, Google's Protocol Buffers\cite{protobuf}, Apache
Thrift\cite{thrift}, and others.  However, none of these systems are
hardware centric, in the sense of supporting data types with the level
of specificity necessary or available to hardware designers (i.e.,
bit-accurate representations).  Also, perhaps more importantly, the
information that the programmer has to specify in order to enable
inter-language linkage contains enough information to be useful for
many other things (such as tagging for debugging, logging, and
serialization for other purposes than linkage); the existing
frameworks have generally not been designed in such a way that their
input descriptions can easily be reused for arbitrary purposes.  A
further, and very important, reason for using the multi-language
approach is that it enables our modeling system to transition slowly
from one technology to another: we do not need to throw everything
away and start from scratch, which in a commercial environment can be
very difficult to do.  The multi-language approach therefore lowers
the overall risks associated with adopting new technology (since they
can be adopted piecemeal and in case of unsatisfactory progress
rejected without restarting the entire program).

A final, and perhaps obvious, advantage of the language-agnostic
approach is that a great many programming languages have been created
over the years, and many of them are suited for different things.  The
language-agnostic approach, for example, allows strongly-typed
languages such as Modula-3 to be used where there is a high cost to
making type errors at the same time as easy-to-parse languages such as
Scheme can be used in contexts where that is a more important goal
(see section~\ref{sec:general-principles}).


\section{General Principles of Design}\label{sec:general-principles}

The major idea behind the work of the functional modeling team is to
capture design information in a way that makes that information conveniently
available for further processing by machine (i.e., amenable to automation).

\subsection{Input formats}\label{sec:input}

This means that data formats used must be:
\begin{itemize}

\item open (not proprietary)

\item possible to parse with tools we are able to produce ourselves

\item fit for purpose (i.e., fit for describing hardware structures and networking protocols and the other objects of our designs)

\item effective (see section~\ref{sec:goodcode})
  
\end{itemize}

Formats that {\em syntactically\/} meet the above requirements include
JSON\cite{json} and Lisp\cite{mccarthy} S-expressions (e.g., Scheme code).
Of course, what {\em semantically\/} is stated in the format is more related
to our design problems and tools.

S-expressions are particularly effective because they allow us {\em embed\/} a
domain-specific language (of our design) in an existing and well-defined programming
language (i.e., Scheme).  See section~4.2 of Abelson and Sussman\cite{sicp} for more information about embedded languages.

Formats that do {\em not\/} meet the above requirements include
Microsoft Word, SystemVerilog, and other proprietary formats but even
open formats such as C source code that are difficult to parse
(extract information from).

As mentioned in section~\ref{sec:agnostic}, which format is used in what context
can vary based on the domain-specific requirements in each case.

\subsection{Human time is the scarce resource}

The scarcest resource of all in engineering is human time.  The goal
of the design process must therefore be to capture this human input in
a succinct format.  An example we run across frequently is when deciding
what code to ``check in'' (i.e., to store in our revision-control systems).
The basic idea is that only human-generated inputs are saved.  The rest of
the design collaterals (and there may be dozens of automated steps
between a human input and the ultimate output) are re-generated as a
matter of course.  Caching of intermediate results is not in itself
objectionable but these intermediate results should {\em never\/} be accorded
the status of ``source code''.  Strict adherence to this rule leads to
an efficient engineering environment; sloppiness can easily lead to many
megabytes of text being accorded the status of ``source code,'' at which
point human engineers are overwhelmed and productivity falls near zero if
that code ever needs to be modified in any way.

\subsection{Tool performance matters}

It is a direct corollary of the previous sections that the runtime of design
tools cannot be allowed to grow without bound.  If tools run quickly, the
temptation to treat intermediate data as source code is much less enticing than
if it takes hours or days to generate some intermediate format.

\subsection{``Elegance'' is not pointless}

A well-designed system is much easier to work with than one that is haphazardly
designed with many special cases and unexplainable legacy.  Therefore, we often
allow ourselves the time to ``reinvent the wheel'' (when there is some problem
with the original wheel).  The reason for this is that the time spent in redesigning
a tool is usually repaid (with interest) even within a single program (not to mention
if that cost can be amortized over several programs).  {\em It is almost always
  easier and faster to do the job right than to look for a shortcut.}


\section{Hardware/Software Co-Design and Co-Engineering}\label{sec:codesign}

As the functional modeling effort straddles the hardware/software interface in
several ways (it both involves making software models of hardware and modeling
specifically that aspect of hardware that most closely communicates with software),
it is natural for the functional modeling to contemplate moving the boundaries
of what is designed.  Several questions are evident:
\begin{itemize}

\item Since the functional modeling effort involves system software
  communicating with hardware modeled with software, it is easy to
  model the movement of features from software implementation to
  hardware implementation and vice versa.

\item Perhaps more importantly, if the modeling effort involves system
  software communicating with the modeled hardware, an important
  question arises: what, precisely, is being modeled?  Is it the
  hardware, or is it the hardware/software system, i.e., the product
  seen by the ultimate customer?
\end{itemize}

A reasonable answer to the second question above is that the goal of the
modeling team is not really to model the specific hardware behavior, but to
model an entire system, together with system software.  This is because the
customer is buying not hardware for its own sake but for the sake of the services
provided by that hardware.  The customer is really unlikely to be overly concerned
with how the engineering teams accomplish their mission, as long as it implements
an agreed-upon interface with agreed-upon performance characteristics.

The concept being advanced here is known as hardware/software
codesign\cite{hw-sw}.  This is in fact always how systems are ultimately delivered
to customers, but in the hardware industry it is still sometimes
considered eccentric to choose the ultimate customer's software
interface to the product as the important modeling target.  In fact,
at Intel it is quite common that the software team for a new product
is not even identified until well after the hardware design has been
mostly finalized.  In our analysis, this is backwards---the software
ought to come first, or be done very early, so that time is not
expended on hardware features that no one will use.  Experienced
readers will see that this design approach is also the natural consequence
of following a top-down design style through a system large enough to
encompass both hardware and system software.  See for example
Dijkstra's discussion in EWD1308\cite{ewd1308}.


\section{What Does ``Good Code'' Look Like?}\label{sec:goodcode}

An important question that engineers and their managers have to
address every day is, ``what does good code look like?''  Some would
say, ``I know it when I see it,'' but similarly to the case in
constitutional law, this approach has its faults.  Probably its major
fault is that implies the existence of a kind of programming
guild---that programming can only learned through apprenticeship.
This mindset runs counter to our objective of articulating clear goals and
directions.  Therefore, we are obligated to attempt to define what ``good code'' is.

\subsection{Kolmogorov complexity}

Certainly one aspect of ``good code'' comes from the concept of the
Kolmogorov complexity\cite{cover+thomas} of the code.  Kolmogorov
complexity applies to any string, not just a program, and measures the
amount of information present in that string as the length of the
shortest program that can print the string.  To give some examples:
the Kolmogorov complexity of an entirely random sequence of bits is
(with probability approaching 1) a little more than the length of the
sequence (in bits).  The Kolmogorov complexity of a very long run of
the digits ($n$ digits) of $\pi$ (which looks very random) is very
low: it is the length of a program for finding digits of $\pi$ plus
the length of the specification of the number of digits, which can be
close to zero for some numbers but never more than $\log n$.  The
exact value of Kolmogorov complexity is uncomputable because of the
Halting Problem and the existence of Busy Beavers (see sections~\ref{sec:model-checking}--\ref{sec:finite}).  See Cover and
Thomas\cite{cover+thomas} for more detail on this topic.

The application of Kolmogorov complexity to programming is that the programs
written by humans should have Kolmogorov complexity that is not too far from
their length.  Replicated code and coordinated strings should be avoided,
following the old maxim {\em never write the same code twice.}

\subsection{More traditional software engineering metrics}

However, Kolmogorov complexity is not the whole story here---that would be too easy.  Many
program-design methodologies involve a number of program fragments
having closely coordinated structure, which implies low Kolmogorov
complexity.  One example is objected-oriented programming, for example
using design patterns\cite{gang-of-4}.  Generally this type of
repeated structure is relatively harmless {\em as long as the
  correspondences can be checked by a compiler.}  It may still be OK
if the check must be deferred to runtime.  What is definitely not OK
is to have a methodology that depends on multiple pieces of strictly
coordinated code where a mistake is not caught through any other
method than human validation (usually program testing).

In other words, in falling order of desirability:
\begin{itemize}

\item A code fragment is written only once.  {\em Excellent}

\item A code fragment is written multiple times and the correspondence is checked at compile time.  {\em Good}

\item A code fragment is written multiple times and the correspondence is unconditionally checked at runtime.  {\em Fair}

\item A code fragment is written multiple times and the correspondence is conditionally checked at runtime.  {\em Poor}

\item A code fragment is written multiple times  and the correspondence is checked only when a specific test for the correspondence is deployed.  {\em Unacceptable}

\end{itemize}

\section{Indirection}

From the observations on input formats in section~\ref{sec:input} and
desirable qualities of source code in section~\ref{sec:goodcode} we
have an outline of what human engineers should be producing as input
to the design process.  It is not unusual that some tool or some step or steps
of the design process requires input with properties quite different from
what we recommend.  We can draw a very simple conclusion from this: in cases
we must use such tools or go through such design steps, the input to those
tools {\em must be automatically generated.}  In other words, to solve the
problem that we require low-complexity input at some point in our process but
we have made a rule that humans only are to produce high-complexity input, we add
{\em indirection.}

Motto: {\em Any problem in computer science can be solved with another level of indirection.} (Attributed to David Wheeler\cite{lampson}.)

\section{Effects on Hardware Microarchitecture of Our Choices}

As described in section~\ref{sec:caltech}, one of the properties of the
design style we pursue here is latency-insensitivity (insensitivity to pipelining
depth).  This may not always be desirable in systems with tight decision
loops (e.g., CPUs and similar hardware).  It is an open question how much
or how little our approach is applicable to such designs.

%%%%%%%%%%%%%%%%%%%%%%%%%%%%%%%%%%%%%%%%%%%%%%%%%%%%%%%%%%%%%%%%%%%%%%

\section{Organizational Observations}

We now have to make a diversion into a topic that we have been avoiding, that is,
organizational structure.  The functional modeling approach, in particular, the
functional-model-as-specification (section~\ref{sec:fm-spec}) is sufficiently
different from common practice that it makes new (or different) demands on organizational structure.

\subsection{Interactions between hardware modelers and software team}

In large companies, hardware and software teams are often insulated from
each other.  Hardware development is driven ``open loop'' by hardware architects
who work on the basis of many years of experience architecting hardware systems.
The problem with this is that in case of truly new features or feedback from
software developers, the feedback to the hardware designers is too little, too late.
The situation is exacerbated by the engineering teams' use of completely different
tools: different programming languages, different documentation standards, different
revision-control systems, and different issue-tracking systems.  Hardware and
software developers working on the same product generally work in completely
different, siloed environments, where in the cases that they are working on the
same code, the code is forked into two completely different versions and never
again merged.  Sometimes the hardware and software designers do not even know of
each others' existence.  Clearly, this type of environment is a great challenge
for the kind of development cycle we are advocating here.

Our recommendation is to management to integrate the hardware and
software teams tighter on a technical level, or we feel that the
chances of producing a coherent product to the customer are very low.
Specific recommendations are for the teams to work on the same code
base, to work in parallel (not sequentially), and to have software
teams available to provide feedback on hardware design features early
on in the design process.  In fact, following the description of Dijkstra's
work in the 1950s\cite{ewd1308} and the general idea of top-down design, it makes sense
to start software development {\em before\/} hardware development, not after, and it
makes sense for software designers to be working closely with hardware architects
on defining hardware features.


\subsection{Interactions between hardware modelers and hardware architects}

The view that the functional model is the architectural specification of the
design that is presented in section~\ref{sec:shift-left} places new organizational
demands on the development process.  Functional modeling programmers are generally
not extremely experienced in silicon architecture, and silicon architects are
often not highly skilled software engineers.  In other words, for us to have a hope
that the functional model will capture the intended silicon architecture requires
that the silicon architects and functional modelers work very closely with each
other.

\subsection{Interactions between hardware modelers and software architects}

Software architects develop the overall design direction of the
software systems that will run on our hardware.  From the remarks on
the subject of hardware/software codesign in
section~\ref{sec:codesign} it is clear that the design approach that we are
advocating involves bringing software and hardware design closer together
than is traditionally done in large hardware companies.  Our team faces challenges
in this area, because corporate management may even be unaware of these engineering
requests, and so even rudimentary issues such as software staffing and budgets have
not yet been worked out by the time our team starts to engage the technical issues.

%%%%%%%%%%%%%%%%%%%%%%%%%%%%%%%%%%%%%%%%%%%%%%%%%%%%%%%%%%%%%%%%%%%%%%

\section{Summary and Conclusion}

\section{Acknowledgements}

Michael Wrighton has given many comments and been instrumental in expressing many
of the ideas described in this document.

\newpage

\appendix

\section{On Usable Annotations}\label{apdx:annotations}

{\em An exchange from the 1968 NATO report.}\cite{crisis}

Dijkstra: I have a point with respect to the fact that people are
willing to write programs and fail to make the documentation
afterwards. I had a student who was orally examined to show that he
could program. He had to program a loop, and programmed the body of it
and had to fill in the Boolean condition used to stop the
repetition. I did not say a thing, and actually saw him, reading,
following line by line with his finger, five times the whole interior
part of his coding. Only then did he decide to fill in the Boolean
condition — and made it wrong. Apparently the poor boy spent ten
minutes to discover what was meant by what he had written down. I then
covered up the whole thing and asked him, what was it supposed to do,
and forced out of him a sentence describing what it had to do,
regardless of how it had been worked out. When this formulation had
been given, then one line of reasoning was sufficient to fill in the
condition. The conclusion is that making the predocumentation at the
proper moment, and using it, will improve the efficiency with which
you construct your whole thing incredibly. One may wonder, if this is
so obvious, why doesn’t it happen? I would suggest that the reason why
many programmers experience the making of predocumentation as an
additional burden, instead of a tool, is that whatever
predocumentation he produces can never be used mechanically. Only if
we provide him with more profitable means, preferably mechanical, for
using predocumentation, only then will the spiritual barrier be
crossed.

Perlis: The point that Dijkstra just made is an extremely important
one, and will probably be one of the major advantages of
conversational languages over non-conversational ones. However, there
is another reason why people don’t do predocumentation: They don't
have a good language for it since we have no way of writing predicates
describing the state of a computation.

\newpage

\section{{\tt corners} module}\label{apdx:corners}

{\tt corners} is an easy-to-prove-correct program that is completely
intractable to model-checking tools and techniques.  The proof is, of course,
left as an instructive exercise for the reader.

\begin{verbatim}
module corners
  #(parameter LN = 3)
   (input  logic                             clk,
    input  logic                             rst_n,

    input  logic [LN-1:0]                    i_x,
    input  logic [LN-1:0]                    i_y,
    input  logic                             i_horiz, 
    input  logic                             i_clr,   

    output logic [(1<<LN)-1:0][(1<<LN)-1:0]  o_ok_h,
    output logic [(1<<LN)-1:0][(1<<LN)-1:0]  o_ok_v,
    output logic [(1<<LN)-1:0][(1<<LN)-1:0]  o_cov    
    );

   localparam N=(1<<LN);

   logic [N-1:0][N-1:0] cov_d, cov_q, resval;

   generate
      for (genvar iii=0; iii<N; ++iii) begin : res_o_b
         for (genvar jjj=0; jjj<N; ++jjj) begin : res_i_b
            if (iii==0 & jjj == 0 | iii == N-1 & jjj == N-1)
              assign resval[iii][jjj] = '1; 
            else
              assign resval[iii][jjj] = '0; 
         end
      end
   endgenerate

   assign o_cov = cov_q;

   always_ff @(posedge clk) begin
      cov_q <= (rst_n & ~i_clr) ? cov_d : resval;
   end

   always_comb begin
      cov_d = cov_q;
      cov_d[i_x][i_y] = '1;
      if (i_horiz)
        cov_d[i_x+1][i_y]   = '1;
      else
        cov_d[i_x]  [i_y+1] = '1;
   end

   generate
      for (genvar ii=0; ii<N; ++ii) begin : gen_ok_outer_blk
         for (genvar jj=0; jj<N; ++jj) begin : gen_ok_inner_blk
            if (ii != N-1)
              assign o_ok_h[ii][jj] = ~cov_q[ii][jj] & ~cov_q[ii+1][jj  ];
            else
              assign o_ok_h[ii][jj] = '0;

            if (jj != N-1)
              assign o_ok_v[ii][jj] = ~cov_q[ii][jj] & ~cov_q[ii  ][jj+1];
            else
              assign o_ok_v[ii][jj] = '0;

         end
      end
   endgenerate

   property legal_request;
      (~rst_n                      |
        i_clr                      |
        i_horiz & o_ok_h[i_x][i_y] |
       ~i_horiz & o_ok_v[i_x][i_y] );
   endproperty

   request_legal:
     assume property (@(posedge clk) legal_request)
       else $error("illegal request to cover already covered squares");

   logic [N*N-1:0] x;

   generate
      for (genvar i=0; i<N; ++i) begin : ass_x_blk
         assign x[N*i+N-1 : N*i] = cov_d[i];
      end
   endgenerate

   property fail_property;
      x == '1;
   endproperty

   fail_corners:
     assert property (@(posedge clk) not fail_property)
       else $error("truncated square death pattern entered");

   int weight;
   assign weight = $countones(x);

   property full;
      (weight >= N*N-2);
   endproperty

   full_cover: cover property (@(posedge clk) full);
endmodule
\end{verbatim}

A test bench is not really needed because the {\tt assume} property
{\tt legal\_request} is a specification of the test bench.  A working test
bench is given below for completeness, for those interested.

\begin{verbatim}
module corners_tb #()();
   parameter LN    = 4;
   parameter print = 0;

   localparam N = (1<<LN);

   logic                                clk;
   logic                                rst_n;

   logic [LN-1:0]                       i_x;
   logic [LN-1:0]                       i_y;
   logic                                i_horiz;
   logic                                i_clr;

   logic [(1<<LN)-1:0][(1<<LN)-1:0]     o_ok_h;
   logic [(1<<LN)-1:0][(1<<LN)-1:0]     o_ok_v;
   logic [(1<<LN)-1:0][(1<<LN)-1:0]     o_cov; // debug only

   corners #(.LN(LN)) u_dut(.*);

   initial begin : genclk
      clk = '0;
      while (1) begin
         #(50);
         clk = ~clk;
      end
   end

   initial begin
      rst_n = '0;
      @(posedge clk); @(posedge clk); @(posedge clk); @(posedge clk);
      rst_n = '1;
   end

   logic full;

   assign full = (o_ok_v == '0) & (o_ok_h == '0);

   initial begin
      logic fail;

      while(1) begin
         @(negedge clk);

         i_clr = full;

         if (~i_clr) begin

            int   x, y;
            logic h;

            while (1) begin
               x = $urandom % N;
               y = $urandom % N;
               h = $urandom % 2;

               if ( h & o_ok_h[x][y])    break;
               if (~h & o_ok_v[x][y])    break;
            end
            i_x     = x;
            i_y     = y;
            i_horiz = h;
         end
      end   
   end

   initial begin
      while(print) begin
         @(posedge clk); #1;

         if (i_clr) $write("======================\n");

         $write("\ncount %d pos %d %d %d\n", $countones(o_cov), i_x, i_y, i_horiz);
         for (int j=0; j<N; ++j) begin
            for (int i=0; i<N; ++i)
               $write("%d", o_cov[i][j]);
            $write("\n");
         end
      end
   end
endmodule
\end{verbatim}


\newpage

\begin{thebibliography}{99}

\bibitem{vlsi2}{Gordon~E.~Moore.  Are we really ready for VLSI$^2$?
  In  Charles~L.~Seitz, ed., {\it Proceedings of the Caltech Conference on Very Large Scale
    Integration.}  Caltech Computer Science Technical Report 3340.
  Pasadena, Calif.: California Institute of Technology, January 1979, pp.~3--14.}

\bibitem{mead}{Carver~A.~Mead.  VLSI and technological innovation.
  In  Charles~L.~Seitz, ed., {\it Proceedings of the Caltech Conference on Very Large Scale
    Integration.}  Caltech Computer Science Technical Report 3340.
  Pasadena, Calif.: California Institute of Technology, January 1979, pp.~15--28.}

\bibitem{rem}{Martin~Rem.  Mathematical aspects of VLSI design.
  In  Charles~L.~Seitz, ed., {\it Proceedings of the Caltech Conference on Very Large Scale
    Integration.}  Caltech Computer Science Technical Report 3340.
  Pasadena, Calif.: California Institute of Technology, January 1979, pp.~55--63.}

\bibitem{m+c} {Carver Mead and Lynn Conway, {\it Introduction to VLSI
Systems.}  Reading, Mass.:\ Addison-Wesley, 1980.}

\bibitem{spwm3}{G.~Nelson, ed.  {\it Systems Programming with
    Modula-3.}  Prentice-Hall Series in Innovative Technology.
  Prentice-Hall, 1991.}
  
\bibitem{ethernet}{IEEE Std 802.3-2015.  {\it IEEE Standard for
    Ethernet.}  Institute of Electrical and Electronics Engineers,
  2015.}

\bibitem{ieee1588}{IEEE Std 1588-2008.  {\it IEEE Standard for a
    Precision Clock Synchronization Protocol for Networked Measurement
    and Control Systems.}  Institute of Electrical and Electronics
  Engineers, 2008.}

\bibitem{sv}{ANSI/IEEE 1800-2017.  {\it IEEE Standard for
    SystemVerilog-Unified Hardware Design, Specification, and
    Verification Language.}  American National Standards Institute,
  2017.}

\bibitem{dragon}{Alfred~V.~Aho, Ravi Sethi, and Jeffrey~D.~Ullman. {\it Compilers:\ Principles, Techniques, and Tools.}  Reading, Mass.:\ Addison-Wesley, 1986.}

\bibitem{tangram}{K.~van~Berkel, J.~Kessels, M.~Roncken, R.~Saeijs, and F.~Schalij.  The VLSI-programming language Tangram and its translation into handshake circuits.  In {\it Proc. European Conference on Design Automation}, pp.~384--389, 1991.}

\bibitem{ref:bryant}{Randal~E.~Bryant. Graph-Based Algorithms for Boolean Function Manipulation. {\it  IEEE Transactions on Computers}, {\bf C-35}(8), August 1986,
pp.~677--691.}

\bibitem{parallel_program_design}{K.~Mani Chandy and Jayadev Misra.  {\it Parallel Program Design.}  Reading, Mass.:\ Addison-Wesley, 1988.}

\bibitem{dennis} {Jack~B.~Dennis.  Data Flow Supercomputers. {\it Computer},
November 1980, pp.~48--56.  IEEE Computer Society, 1980.}


\bibitem{discipline}{Edsger~W.~Dijkstra.  {\it A Discipline of Programming.}
Englewood Cliffs, N.J.:\ Prentice-Hall, 1976.}

  
\bibitem{ref:csp}{C.~A.~R.~Hoare. Communicating Sequential Processes. {\it
Communications of the ACM}, {\bf 21}(8):666--677, 1978.}


\bibitem{k+r}{Brian~W.~Kernighan and Dennis~M.~Ritchie. {\it The C Programming Language}, second ed.  Englewood Cliffs, N.J.:\ Prentice-Hall, 1988.}

\bibitem{ref:synthesis}{Alain~J.~Martin.  Compiling Communicating
Processes into Delay-insensitive VLSI circuits. {\it Distributed
Computing}, {\bf 1}(4), 1986.}


 
\bibitem{this-and-that} {Alain~J.~Martin.  Synthesis of Asynchronous VLSI
  Circuits. In J.~Staunstrup, ed., {\it Formal Methods for VLSI Design}. North-Holland, 1990.}

\bibitem{bips-tools} {Jeremy Dion and Louis Monier.  Design Tools For BIPS-0.  WRL Technical Note TN-32.  Palo Alto, Calif.: Digital Equipment Corp., 1992.}

\bibitem{chisel}{Jonathan Bachrach, Huy Vo, Brian Richards, Yunsup Lee,
Andrew Waterman, Rimas Avi\v{z}ienis, John Wawrzynek, Krste Asanovi\'{c}.  Chisel: Constructing Hardware in a Scala Embedded Language.  In {\it Proceedings of DAC 2012.}  Association for Computing Machinery, 2012.}


\bibitem{must-change}{Shacham, O., Azizi, O., Wachs, M., Qadeer, W.,
Asgar, Z., Kelley, K., Stevenson, J.,
Solomatnikov, A., Firoozshahian, A., Lee, B.,
Richardson, S., and M., H. Rethinking digital
design: Why design must change. IEEE Micro
(Nov/Dec 2010).}

\bibitem{blue-verilog}{Bluespec Inc. Bluespec$^{\rm TM}$ SystemVerilog Reference
Guide: Description of the Bluespec SystemVerilog
Language and Libraries. Waltham, MA, 2004.}


\bibitem{crete-tutorial}{Alain~J.~Martin, et al.  Full-Day Tutorial
   Caltech Asynchronous Synthesis Tools (CAST) at ASYNC 2004,
   FORTH, Crete, April 19, 2004.}

\bibitem{alain-ifip}{Alain~J.~Martin.  Tomorrow's Digital Hardware
  will be Asynchronous and Verified.  In {\it Proceedings of the IFIP
    (International Federation for Information Processing) Congress
    1992: Information Processing 1992,} Volume~I.  Amsterdam:
  Elsevier, 1992.}

\bibitem{crisis}{Peter Naur and Brian Randell, eds.  {\it Software Engineering,} report on a conference sponsored by the NATO Science Committee, Garmisch, Germany, 7th to 11th October 1968.  Brussels: NATO, 1969.}


\bibitem{structured}{Edsger~W.~Dijkstra.  Notes on Structured Programming.  EWD249.  Technische Hogeschool Eindhoven, Department of Mathematics, Report 70-WSK-03, August 1969--April 1970.}  

\bibitem{ewd1308}{Edsger~W.~Dijkstra.  What led to ``Notes on Structured Programming.''  EWD1308.  Nuenen, the Netherlands: 10 June 2001.}

\bibitem{ewd123}{Edsger~W.~Dijkstra.  Cooperating Sequential Processes.  EWD123.  Lecture notes.  September 1965.}

\bibitem{org-man}{William~H.~Whyte.  {\it The Organization Man.}  New York:\ Simon~\&~Schuster, 1956.}

\bibitem{ewd340}{Edsger~W.~Dijkstra.  The humble programmer.  EWD340.  1972~Turing~Lecture.  {\it Communications of the ACM,} {\bf 15}(10):859--866, 1972.}

\bibitem{lampson}{Butler Lampson.  Principles for computer system design.  1993~Turing~Lecture.  ACM, February 1993.}

\bibitem{ewd209}{Edsger~W.~Dijkstra.  A constructive approach to the problem of program correctness.  EWD209.  August 1967.  Also in {\it BIT\/} {\bf 8}, pp.~174--186, 1968.}

\bibitem{retiming}{Charles~E.~Leiserson and James~B.~Saxe.  Retiming synchronous circuitry.  {\it Algorithmica,} {\bf 6}(1--6):5--35, June 1991.}

\bibitem{simulink}{{\it Matlab \& Simulink R2015b.} MathWorks, 2015.}


\bibitem{rdl}{Register Description Working Group.  {\it SystemRDL
    v1.0: A specification for a Register Description Language.}  Napa,
  Calif.:\ The SPIRIT Consortium, 2009.  Republished by Accelera, 2012.}

\bibitem{hot21164}{Gregg Bouchard and Pete Bannon.  Design objective of the 0.35-micron Alpha 21164 microprocessor.  In {\it Presentations of the 1996 Hot Chips Symposium,} held at Stanford University, August 18--20, 1996.  IEEE, 1996.}
  
\bibitem{hotalta}{Mike Davies.  One billion packet per second frame processing pipeline. In {\it Presentations of the 2011 Hot Chips Symposium,} held at Stanford University, August 17--19, 2011.  IEEE, 2011.}

\bibitem{soul}{Tracy Kidder.  {\it The Soul of a New Machine.}  Columbus, Ga.:\ Atlantic-Little, Brown, 1981.}

\bibitem{mmm}{Frederick Brooks.  {\it The Mythical Man-Month.}  Boston, Mass.:\ Addison-Wesley, 1975.}
  

\bibitem{browne}{M.C.~Browne, E.M.~Clarke, D.L.~Dill.  Checking the correctness of sequential circuits.  In {\it Proceedings of the 1985 International Conference on Computer Design,} October 1985.  IEEE, 1985.}

\bibitem{pneuli}{A.~Pneuli.  The temporal semantics of concurrent programs.  In {\it 18th Annual Symposium on Foundations of Computer Science.}  1977.}

\bibitem{dekker}{Theodorus~J.~Dekker, related by E.W.~Dijkstra in: Over de sequentialiteit van procesbeschrijvingen (EWD-35) (in Dutch).  Undated, 1962 or 1963.  See also: E.W.~Dijkstra.  Solution of a problem in concurrent programming control.  {\it CACM\/} {\bf 8}(9), September 1965.  Association for Computing Machinery, 1965.}

\bibitem{minsky-book}{Marvin~L.~Minsky.  {\it Computation: Finite and Infinite Machines.}  Englewood Cliffs, N.J.:\ Prentice-Hall, 1967.}

\bibitem{minsky}{Marvin~L.~Minsky.  The unsolvability of the halting problem.  Section 8.2 of {\it Computation: Finite and Infinite Machines.}
  Englewood Cliffs, N.J.:\ Prentice-Hall, 1967.}

\bibitem{turing}{Alan Turing.  On computable numbers, with an application to the Entscheidungsproblem.  In {\it Proceedings of the London Mathematical Society\/}, Series 2 {\bf 42}.  London Mathematical Society, 1936.}

\bibitem{busy-beaver}{Tibor Rad\'{o}.  On non-computable functions.  {\it Bell System Technical Journal,} {\bf 41}(3), May 1962.}

\bibitem{lamport-logic}{Leslie Lamport.  The `Hoare Logic' of concurrent programs.  {\it Acta Informatica.} {\bf 14}(1):21--37, 1980.}

\bibitem{computational-universe}{Seth Lloyd.  Computational capacity of the universe.  {\it Phys.\ Rev.\ Lett.} {\bf 88}, 237901, May 24, 2002.}

\bibitem{knuth-satisfiability}{Donald~E.~Knuth.  Satisfiability.  {\it The Art of Computer Programming,} Volume 4, Fascicle 6.  Boston, Mass.: Pearson Education, 2016.  Corrected reprint, April 2018.}

\bibitem{loh-jasper}{Lawrence Loh.  {\it Upgrade your verification with Jasper!}  Jasper Design Automation, 2013.  (Jasper Design Automation is now part of Cadence Design Systems, Inc.)}

\bibitem{c}{ISO JTC1/SC22/WG14.  ISO/IEC 9899:2011.  Programming Languages~-~C.  International Organization for Standardization, 2011.}

\bibitem{cover+thomas}{Thomas~M.~Cover and Joy~A.~Thomas.  {\it Elements of Information Theory.}  New York, N.Y.:\ John Wiley \& Sons, 1991.}

\bibitem{floyd}{Robert~W.~Floyd.  Assigning meanings to programs.  In
  J.T.~Schwarz, ed., {\it Mathematical Aspects of Computer Science,}
  Proceedings of Symposium on Applied Mathematics {\bf 19}. American
  Mathematical Society, 1967.}

\bibitem{hoare-logic}{C.A.R.~Hoare.  An axiomatic basis for computer programming.  {\it CACM\/} {\bf 12}(10), pp.~576--580, October 1969.}

\bibitem{java}{James Gosling, Bill Joy, Guy Steele, Gilad Bracha, and Alex Buckley.  {\it The Java Language Specification, Java SE 8 Edition.}  Redwood City, Calif.:\ Oracle, 2015.}

\bibitem{scala}{Martin Odersky, Lex Spoon, and Bill Venners.  {\it Programming in Scala,} third edition.  Walnut Creek, Calif.:\ Artima Press, 2016.}

\bibitem{functional-scala}{Paul Chiusano and R\'{u}nar Bjarnson.  {\it Functional Programming in Scala.} Manning Publications, 2014.}

\bibitem{gang-of-4}{Erich Gamma, John Vlissides, Ralph Johnson, and Richard Helm.  {\it Design Patterns: Elements of Reusable Object-Oriented Software.}  Addison-Wesley, 1994.}

\bibitem{hanson-c}{David~R.~Hanson.  {\it C Interfaces and Implementations: Techniques for Creating Reusable Software.}  Addison-Wesley, 1996.}

\bibitem{monitors}{C.A.R.~Hoare.  Monitors: An operating system structuring concept.  {\it CACM,} {\bf 17}(10), pp.~549--557, October 1974.}

\bibitem{concurrent-pascal}{Per Brinch Hansen.  The programming language Concurrent Pascal.  {\it IEEE Transactions on Software Engineering I.} {\bf 2}, June 1975, pp.~199--207.}

\bibitem{hansen-book}{Per Brinch Hansen.  {\it The Architecture of Concurrent Programs.} Englewood Cliffs, N.J.:\ Prentice-Hall, 1977.}

\bibitem{adt-liskov}{Barbara Liskov and Stephen Zilles.  Programming with abstract data types.  {\it Proceedings of the ACM SIGPLAN symposium on very high level languages.}  New York, N.Y.:\ Association for Computing Machinery, 1974.}

\bibitem{hw-sw}{David~A.~Patterson and John~L.~Hennessey.  {\it
    Computer Organization and Design: The Hardware/Software
    Interface,} fifth edition.  Elsevier Morgan Kaufmann, 2014.}

\bibitem{scheme}{Gerald Jay Sussman and Guy Lewis Steele Jr.  {\it SCHEME: An Interpreter for Extended Lambda Calculus.}  Massachusetts Institute of Technology, Artificial Intelligence Laboratory Memo AIM-349.  December 1975.}
  
\bibitem{r4rs}{William Clinger and Jonathan Rees, eds.  {\it
    Revised$^4$ Report on the Algorithmic Language Scheme.} Cambridge,
  Mass.:\ Massachusetts Institute of Technology, 2 November 1991.}

\bibitem{sicp}{Harold Abelson and Gerald Jay Sussman with Julie
  Sussman.  {\it Structure and Interpretation of Computer Programs,}
  second edition.  Cambridge, Mass.:\ MIT Press, 1996.}

\bibitem{mccarthy}{John McCarthy.  Recursive functions of symbolic expressions and their computation by machine, Part I.  {\it CACM,} {\bf 3}(4), pp.~184--195, April 1960.}
  
\bibitem{alpha_architecture}{Alpha Architecture Committee.  {\it Alpha Architect
ure Reference Manual}, third edition.  Boston, Mass.:\ Digital Press, 1998.}

\bibitem{abe_bs}{Abraham Ankumah.  Designing an Energy-Efficient Asynchronous 80
C51 Microcontroller.  B.S. thesis, California Institute of Technology, Division
of Engineering and Applied Science, Department of Electrical Engineering, 2001.}

\bibitem{algol60}{J.~W.~Backus, F.~L.~Bauer, J.~Green, C.~Katz, J.~McCarthy,
P.~Naur, A.~J.~Perlis, H.~Rutishauser, K.~Samelson, B.~Vauquois, J.~H.~Wegstein,
 A.~van~Wijngaarden, and M.~Woodger. {\it Revised Report on the Algorithmic Language}~{\scshape Algol 60}.  Berlin:\ Springer-Verlag, 1969.}

\bibitem{ref:burns_ms}  Steven~M.~Burns. {\it Automated Compilation of Concurrent Programs into Self-timed Circuits.} M.S.\ thesis, Caltech \hbox{CS-TR-88-2}, California Institute of Technology, 1988.

\bibitem{filter}{U.~V.~Cummings, A.~M.~Lines, and A.~J.~Martin.  An Asynchronous
 Pipelined  Lattice Structured Filter.  In {\it Proceedings of the International
 Symposium on Advanced Research in Asynchronous Circuits and Systems.}  Los Alam
itos, Calif.:\ IEEE Computer Society Press, 1994.}

\bibitem{andrew_ms} {Andrew Lines.  {\it Pipelined Asynchronous
Circuits.}  M.S.\ thesis, California Institute of Technology CS-TR-95-21, 1995.}

\bibitem{rajit_cast}{Rajit Manohar.  {\it Cast: Caltech Asynchronous Tools.} \TeX{}info documentation package.  Unpublished, California Institute of Technology
Department of Computer Science/Cornell University Department of Electrical Engineering, 1998--2001. }



\bibitem{what??}{A.~J.~Martin. Programming in VLSI: From communicating
processes to delay-insensitive VLSI circuits. In C.~A.~R.~Hoare,
ed., {\it Developments in Concurrency and Communication}, in UT Year of Programm
ing Series, pp.~1--64. Englewood Cliffs, N.J.:\ Addison-Wesley, 1990. }

\bibitem{ref:uP}{Alain~J.~Martin, Steven~M.~Burns, Tak-Kwan Lee, Drazen
Borkovic, and Pieter~J.~Hazewindus. The design of an asynchronous
microprocessor. In Charles L. Seitz, ed., {\it Advanced Research in
VLSI:\ Proceedings of the Decennial Caltech Conference on VLSI\/},
pp.~351--373. Cambridge, Mass.:\ MIT Press, 1991.}

\bibitem{synthesis_tr} {Alain~J.~Martin.  {\it Synthesis of Asynchronous VLSI Circuits.}  Caltech Computer Science Technical Report \hbox{CS-TR-93-28}.  Pasaden
a, Calif.:\ California Institute of Technology Computer Science Department, 1993.}

\bibitem{mips97} {A.~J.~Martin, A.~Lines, R.~Manohar, M.~Nystr\"{o}m,
P.~P\'enzes, R.~Southworth, U.~Cummings, and T.~K.~Lee.  The Design of an
Asynchronous MIPS R3000 Processor. In R.~B.~Brown and A.~T.~Ishii, eds., {\it Proceedings of the 17th
Conference on Advanced Research in VLSI.}  Los Alamitos, Calif.:\ IEEE
Computer Society Press, 1997.}

\bibitem{de_muller}{D.~E.~Muller and W.~S.~Bartky.  A theory of asynchronous circuits.  In {\it The Annals of the Computation Laboratory of Harvard University.
 Volume~XXIX:\ Proceedings of an International Symposium on the Theory of Switching, Part~I.}, pp.~204--243, 1959.}

\bibitem{qdi} {A.~J.~Martin.  The limitations to delay-insensitivity
in asynchronous circuits.  In W.~J.~Dally, ed., {\em Sixth MIT Conference on Advanced
Research in VLSI}.  Cambridge, Mass.:\ MIT Press, 1990.}

\bibitem{literate-programming}{Donald~E.~Knuth.  {\it Literate Programming.}  Stanford, Calif.:\ Stanford University Center for the Study of Language and Information, 1992.} 

\bibitem{semantics-pred-calc}{Edsger~W.~Dijkstra and Carel~S.~Scholten.  {\it Predicate Calculus and Program Semantics.}  Berlin:\ Springer-Verlag, 1990.}

\bibitem{swig}{{\it SWIG Users Manual.} {\tt http://www.swig.org/Doc1.3/Contents.html}}

\bibitem{protobuf}{Google Protocol Buffers.  {\tt https://developers.google.com/protocol-buffers/}}

\bibitem{thrift}{Apache Thrift. {\tt https://thrift.apache.org/}}

\bibitem{json}{JSON.  {\tt https://www.json.org/}}

\bibitem{pascal}{Kathleen Jensen and Niklaus Wirth.  {\it PASCAL~---~User Manual and Report.}  Lecture Notes in Computer Science {\bf 18}. Berlin: Springer-Verlag, 1974.}

\bibitem{modula}{Niklaus Wirth.  Modula: A language for modular programming.  {\it Software~---~Practice and Experience,} {\bf 7}, pp.~37--65, 1977.}

\bibitem{simula}{Ole-Johan Dahl, Bj{\o}rn Myhrhaug, and Kristen Nygaard.  {\it Common Base Language.}  Oslo:\ Norwegian Computing Center, 1970.}

\bibitem{smalltalk-72}{Adele Goldberg and Alan Kay, eds.  {\it Smalltalk-72 Instruction Manual.}  Palo Alto, Calif.:\ Xerox Palo Alto Research Center, 1976.}

\bibitem{smalltalk-80}{Adele Goldberg, David Robson, and Michael~A.~Harrison.  {\it Smalltalk-80: The Language and its Implementation.}  Addison-Wesley, May 1983.}

\end{thebibliography}

\newpage

\section{About the Authors}

{\bf Mika Nystr\"om} is the author.

\end{document}
