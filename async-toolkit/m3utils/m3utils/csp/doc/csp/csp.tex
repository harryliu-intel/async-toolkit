\documentclass[11pt]{report}

% Page layout
\usepackage[letterpaper, margin=1in]{geometry}
\usepackage{parskip}
\setlength{\parindent}{0pt}
\usepackage[greek,english]{babel}

% Typography
%\usepackage{times}
%\usepackage[T1]{fontenc}

% Math and symbols
\usepackage{amsmath}
\usepackage{amssymb}
\usepackage{longtable}
\usepackage{fancyvrb}
\usepackage{booktabs}

% Code listings for syntax
\usepackage{listings}
\lstset{
    basicstyle=\ttfamily\small,
    columns=fullflexible,
    keepspaces=true,
    showstringspaces=false,
    breaklines=true,
    frame=none,
    xleftmargin=2em,
}

% For BNF-style syntax rules
\usepackage{syntax}
\setlength{\grammarparsep}{4pt}
\setlength{\grammarindent}{8em}

% Hyperlinks
\usepackage[colorlinks=true, linkcolor=blue, urlcolor=blue, citecolor=blue]{hyperref}

% Headers and footers
\usepackage{fancyhdr}
\pagestyle{fancy}
\fancyhf{}
\fancyhead[L]{\textit{The Fulcrum CSP Language}}
\fancyhead[R]{\thepage}
\renewcommand{\headrulewidth}{0.4pt}

% Section formatting
\usepackage{titlesec}
\titleformat{\section}{\large\bfseries}{\thesection}{1em}{}
\titleformat{\subsection}{\normalsize\bfseries}{\thesubsection}{1em}{}

% Custom commands
\newcommand{\keyword}[1]{\texttt{\textbf{#1}}}
\newcommand{\type}[1]{\textit{#1}}
\newcommand{\term}[1]{\textit{#1}}
\newcommand{\nonterminal}[1]{\textit{#1}}

% Title page
\title{\textbf{The Fulcrum CSP Language}}
\author{Edited by\\
  Mika Nystr\"om \\
  Neuromorphic Computing Laboratory \\
Intel Corporation}
\date{Version 0.0\\
\today}

\begin{document}

\maketitle

\vspace{1em}

\noindent

\vspace{1em}

\noindent
 Copyright \copyright\ 2025 Intel Corporation
 
 Licensed under the Apache License, Version 2.0 (the ``License'');
 you may not use this file except in compliance with the License.
 You may obtain a copy of the License at
 
 {\tt http://www.apache.org/licenses/LICENSE-2.0}
 
 Unless required by applicable law or agreed to in writing, software
 distributed under the License is distributed on an ``AS IS'' BASIS,
 WITHOUT WARRANTIES OR CONDITIONS OF ANY KIND, either express or implied.
 See the License for the specific language governing permissions and
 limitations under the License.

\bigskip\bigskip
 
Portions may be

Copyright \copyright\ 1989 Digital Equipment Corporation, Ing.~C.~Olivetti and~C., SpA.

``This work may not be copied or reproduced in whole or in part for any
commercial purpose.  Permission to photocopy in whole or in part
without payment of fee is granted for nonprofit educational and
research purposes provided that all such copies include the following:
a notice that such copying is by permission of the Systems Research
Center of Digital Equipment Corporation in Palo Alto, California and
the Olivetti Research Center of Ing.~C.~Olivetti and~C., SpA in Menlo
Park, California; an acknowledgment of the authors and individual
contributors to the work; and all applicable portions of the copyright
notice.  All rights reserved.''


\clearpage

\tableofcontents

\clearpage

% ========================================
% INTRODUCTION
% ========================================

\bigskip
\chapter{Introduction}

\bigskip
\begin{center}
\textit{The language designer should be familiar with many alternative\\
features designed by others, and should have excellent judgment in\\
choosing the best and rejecting any that are mutually inconsistent...\\
One thing he should not do is to include untried ideas of his own.\\
His task is consolidation, not innovation.}

---C.A.R. Hoare
\end{center}
\section{History}
Starting in the 1980s, Alain Martin at Caltech started to use a
variant of Hoare's Communicating Sequential Processes to describe
asynchronous circuit designs, and while the basic syntax and semantics of the
language was well established by Martin and his students, the usage
model and many implementation details were left informal.  A detailed
implementation effort followed at Fulcrum Microsystems in the early
2000s, and the resulting language was used to describe the high-level
design of numerous Fulcrum products throughout the 2000s and 2010s,
and later at Intel, in the Neuromorphic Computing Lab of Intel Labs.
The initial implementation was structured as a circuit simulator, but
over time, a direct-to-silicon compiler was also developed at Fulcrum,
Intel, and Timeless Design Automation.  At the time of writing, CSP
has seen over two decades of successful use as a hardware description
and validation language.

CSP, like its 1970s namesake, is a programming language based on a
static graph of many small independent processes, which do not share
internal state but use the rendezvous for communication.  CSP supports
many features that are helpful in hardware design, such as
arbitrary-precision arithmetic by default, bit extraction, packing and
unpacking of structured data, and of course, extremely high levels of
multithreading.

One design goal of CSP has been to as much as possible banish
unpredictable behaviors from the language.  Unpredictable behaviors
that might be of little concern in software development can be a much
bigger concern in hardware description, because CSP is used both as
the input to a silicon compiler and to a software runtime (called a
``simulator''), and the software implementation is used to validate
the design before fabrication.  Unspecified constructs in the language
would allow the two implementations to diverge, meaning that results
from software simulation of the design could not be relied on to
validate the design since the hardware implementation might be
different.  Accordingly, CSP variables have well-defined initial
values, and expression-evaluation orders as well as function-parameter
evaluation orders are strictly specified.

\subsection{CAST : Caltech Asynchronous Tools}
\label{sec:cast}

Like many (but not all) of its predecessors, CSP does not include
facilities for process creation or the establishment of communication
channels between processes.  In the Fulcrum system, these tasks are
handled by an enclosing language environment called CAST (Caltech
ASynchronous Tools).  In other words, when a CSP program starts
executing, the entire process graph has already been created and
connected.  This of course closely matches the distinction between
fabrication and operation that is present in all hardware.
Nevertheless, it is not difficult to imagine a system that uses CSP to
describe processes that will be created and connected by a means other
than CAST: for example a dynamic means, such as might be desired in a
software system.  CAST is not a subject of this report, and mentions
of that language can be read as simply referring to the process
creation and connection that must occur before a CSP process begins
execution.  That being said, one of the main uses of CSP is to serve
as a vehicle for performing process decomposition (``Martin synthesis'')
and it is the CAST framework that establishes the relationship between
different CSP descriptions of the same computational process.

\subsection{The name ``CSP''}
This report describes the syntax and semantics of {\em Fulcrum CSP,} a
programming language developed at Fulcrum Microsystems in the early
2000s, before that company was acquired by Intel in 2011.  The
language takes its name from Hoare's 1978 {\em Communicating
  Sequential Processes,} what the wider programming community knows as
``CSP'', but has seen significant changes from that language, in
particular following the {\em Communicating Hardware Processes\/} (CHP)
described (but not yet named) by Martin in 1986.  In any case, the
name ``CSP'' will be used to mean Fulcrum CSP throughout this report unless
qualified otherwise.


\section{Overview}

The remainder of the introduction covers the main features of Fulcrum
CSP informally and gives some perspective and background information about
the authors of the language.  The next chapter gives a more formal reference manual
for the language.  The final chapter covers a few practical example
programs, showing some of the CSP features in action.  Finally, the
appendix gives the syntax equations for the language, extracted from
the ANTLR grammar used at Fulcrum to construct the CSP parser.

\subsection{Note on syntax}

To avoid confusion, throughout this report we will use the
machine-readable syntax used in Fulcrum CSP, rather than a syntax
``for publication.''  Accordingly, assignment will be written \verb!=!
  (not $:=$), the test for equality will be written \verb!==! (not
$=$), the probe will be written \verb!#A! (not $\overline{A}$), etc.

\section{Features}

CSP has a number of features that may need introduction for those
familiar with the majority of software or hardware languages, and the
remainder of the section will discuss these.

\subsubsection{Process decomposition}

The basic language as developed initially by Martin and his students
was used for {\em Martin synthesis\/} of asynchronous circuits.  This
is a method of program derivation done in the spirit of Edsger
Dijkstra's work, usually on a blackboard, with ``toy'' programs that
still show the essentials of the programs being operated on.  A key
aspect of Martin synthesis is {\em process decomposition,} whereby a
large, complex program is decomposed into a communicating network of
smaller programs, written in a form of Hoare's CSP, with a few changes
to better support hardware-oriented programming. Martin called this
language {\em Communicating Hardware Processes (CHP).}  

\subsubsection{Guarded commands}

Guarded commands were introduced as a reasoning model for
nondeterministic programs by Dijkstra in 1975.  Hoare's CSP and its
descendants have used these for looping and branching.  Following
Dijkstra and unlike if-else statements of the kind introduced in
Algol, guarded commands are symmetric---there is no particular order in
which guards are evaluated (until the \verb!else! keyword is introduced).

In short, instead of
\begin{Verbatim}[samepage=true]
   if e then
     S0
   else if f then
     S1
   else
     S2
   end
\end{Verbatim}
Dijkstra introduced the {\em selection\/}:
\begin{Verbatim}[samepage=true]
   if 
     e -> S0
   []
     f -> S1
   []
     ~e & ~f -> S2
   fi
\end{Verbatim}
with the semantic difference that if both \verb!e! and \verb!f! are
true, the choice between \verb!S0! and \verb!S1! is nondeterministic
and not determined by the syntactic order of the guards.  Activating
the selection with no guards true in Dijkstra was an error (equivalent
to \verb!abort!).  Martin introduced the idea that the selection
instead blocks until one of the guards {\em becomes\/} true, which in
a distributed system can happen owing to events external to the
process.

Similarly, Dijkstra introduced the \verb!do! loop:
\begin{verbatim}
   do
     e -> S0
   []
     f -> S1
   od
\end{verbatim}
which repeats until neither \verb!e! nor \verb!f! is true, and then
terminates, using the same nondeterministic selection rules as for the
selection until then.

Our semantics follow Dijkstra and Martin and use Hoare's syntax of
\begin{Verbatim}[samepage=true]
   [ 
     e -> S0
   []
     f -> S1
   []
     ~e & ~f -> S2
   ]
\end{Verbatim}
for the selection, and
\begin{verbatim}
   *[
     e -> S0
   []
     f -> S1
   ]
\end{verbatim}
for the do-loop.

\subsubsection{The rendezvous}

In CSP, program variables are local to each process in which they are
declared and used.  Processes communicate entirely through the {\em
  rendezvous,} a combined synchronization and communication primitive.
The rendezvous is a distributed assignment statement.  As a simple
example, consider two processes \verb!P! and \verb!Q! where \verb!P!
is the sending process, in which case its text includes a send
operation of a local value \verb!v! on a {\em port\/} \verb!X!:
\begin{Verbatim}[samepage=true]
   X!v
\end{Verbatim}
and further that the receiving process \verb!Q! performs a receive operation
on a port \verb!Y! into a local variable \verb!w!:
\begin{Verbatim}[samepage=true]
   Y?w
\end{Verbatim}
Consider further that (by a mechanism external to CSP itself) the ports
\verb!P.X! and \verb!Q.Y! have been connected together and therefore form
a {\em channel.}  Then the semantics of the rendezvous are as follows:
\begin{itemize}
\item If \verb!P! reaches \verb,X!v, and \verb!Q! has not reached \verb,Y?w,
  then process \verb!P! is suspended until \verb!Q! reaches \verb,Y?w,.

\item If \verb!Q! reaches \verb,Y?w, and \verb!P! has not reached \verb,X!v,
  then process \verb!Q! is suspended until \verb!P! reaches \verb,X!v,.

\item When both \verb!P! has reached \verb,X!v,
  and \verb!Q! has reached \verb,Y?w, then the {\em distributed assignment}
\begin{Verbatim}[samepage=true]
   w = v
\end{Verbatim}
is executed and \verb!P! and \verb!Q! continue execution at their next statements.  See section~\ref{sec:rendezvous}.
  
\end{itemize}
The simple semantics given here can be modified by the addition of {\em slack\/} to the channel.  If the channel has slack $n$, then \verb,P.X!v, can complete
$n$ more times than \verb,Q.Y?w, before \verb,P, is suspended, following
Martin 1981\cite{Martin1981}.  Values entering the channel appear on the
output in FIFO order.  The simple semantics above then illustrates
the case $n=0$, {\em slack-zero\/} communication.  See section~\ref{sec:slack}.

The explicit synchronization provided by the rendezvous allows distributed
systems to be built without an additional synchronization mechanism such as
a clock.  The Martin synthesis method can be applied starting from a high
level specification (such as an architecture manual for a specific computer) and be performed repeatedly until the resulting system
consists of, essentially, communicating processes that represent
single logic gates.  Since at every point, the rendezvous is used for
synchronization, by this method hardware systems can be constructed out
of logic gates entirely without explicit clocks' being added.

\subsubsection{The probe}

The {\em probe\/} was introduced by Martin in 1985 and describes a simple
mechanism for describing reactive programming constructs without introducing
auxiliary variables.  Consider a process that handles inputs from two
unrelated places as these inputs arrive.  Call the ports on the process
\verb!A! and \verb!B!.  If the process is merging inputs from places that
are truly unrelated, there may be no way of knowing in what sequence inputs
will arrive on \verb!A! and \verb!B!.  Simply receiving the inputs is not
a solution, since if we attempt to code this as
\begin{Verbatim}[samepage=true]
   A?a ; B?b
\end{Verbatim}
and only one of the two ports has an input arriving, our system will
deadlock with this process waiting for an input to arrive on the other
ports.  The probe is the solution to this problem.  The probe is written
\verb!#A! and is a boolean value that is true exactly when performing an
action on \verb!A! can proceed without suspension.  The coding of our
example then is
\begin{Verbatim}[samepage=true]
   [  #A -> A?a ; // handle a
   [] #B -> B?b ; // handle b
   ]
\end{Verbatim}

More complex expressions can then be constructed using the usual logical
operators, as needed.

The probe can be used to describe a simple reactive process structure,
where the process waits for inputs and handles each input in turn, doing
nothing else:
\begin{Verbatim}[samepage=true]
   *[[ #A -> A?a ; // handle a
    [] #B -> B?b ; // handle b
    ]]
\end{Verbatim}
See section~\ref{sec:probe}.

\subsubsection{Node data type}

While CSP is based on the idea of independent parallel processes that
communicate across channels using the rendezvous, it also includes a
more basic {\em node\/} data type for unstructured communcation.
While the majority of hardware design can be accomodated by the
channel structures, some algorithms need access to a lower-level,
shared data type.  An example of something that is difficult to accomodate
with channels is global configuration state, which in a hardware system
may be fixed at power-up, or provided to a module as a shared constant
that does not change during operation.

\subsubsection{Arbitrary-precision arithmetic}

In software, there are three common approaches to arithmetic
precision.

\begin{itemize}
\item Some languages (such as Java) define arithmetic to be the same on every
machine, with a limited number of data types and bit widths available,
usually a set chosen based on hardware that is widely available during
a particular time period.  One justification for this approach is that a
given code will run the same everywhere, since the rounding, overflow, etc.,
will occur the same on every machine.

\item Another approach is taken by C, Modula-3, et al.,
and provides the programmer with a few data types whose widths are not
specified precisely but can be tested at compile and run time, although
in practice in this approach, programmers tend to get used to certain
specific types' being available.  As a result, even though the
programmer's reference manual may say that bit widths are unspecified,
in reality, the approach is much more similar to Java's in that everybody
expects certain types to behave in very specific ways.  One justification
for not specifying the data types precisely in the language reference
manual is that this allows the runtime to be optimized for whatever
machine is available.

\item The third approach is to provide data types that are arbitrarily
  large and runtimes that allocate memory to represent arbitrarily large
  numbers as the result of computations.  Many Lisp dialects use this
  approach. 
\end{itemize}

When describing hardware, none of the above approaches is really right.
We are describing a new machine, so there is really no justification for
tying our data types to existing hardware, and we need the flexibility to
describe a wide variety of hardware.  Another issue is that arithmetic
expressions can contain a wide variety of anonymous values.  Most programming
languages (even Verilog) provide only limited widths for these values,
and often very complicated rules for knowing how the values are treated.

CSP takes an approach that expressions use arbitrary precision
arithmetic, but variables can either have arbitrary precision {\em
  or\/} specified precision, in which case the programmer-designer
specifies the precise bit width of the variable (and whether it is
signed or not).  In this way, the rules for expressions are very simple
(arbitrary precision) but it is still possible to specify exactly
the width of registers, etc., which is important for silicon compilation.
Only two's-complement arithmetic is directly
supported.  See section~\ref{sec:integer-types}.


\subsubsection{Structured types}

Structured types (records in Pascal parlance) are common in hardware-oriented
languages.  CSP's structures are called {\em structs\/} and are by default
packed, with a specified packing order, and bitwise control over layout
and data widths.  See section~\ref{sec:struct}.

\subsubsection{Variable initialization and evaluation orders}

To avoid as much as possible unspecified behaviors in the language,
all variables have specific default values.  Furthermore, unlike in most
software languages, the evaluation orders of expressions and
function parameters is fully specified.  See section~\ref{sec:initialization}.

\subsubsection{Functions}

In CSP, functions (procedures in Pascal parlance) have value or
value-result semantics and cannot access any global variables
(excepted predeclared constants from outside the CSP environment).
See section~\ref{sec:functions}.


\subsubsection{Memory management}

CSP has two variable-sized types: strings and the special
arbitrary-precision integer \verb!int!.  There is no memory allocation
or freeing as such.  An implementation is expected to use garbage
collection to manage the memory used for strings and arbitrary-precision
integers, but this is not visible to the programmer-designer working in
CSP.


\section{Synthesizable subset}

This report documents a language intended to describe the behavior of
hardware systems starting at a very high level of description but also
applicable for describing highly decomposed hardware systems, at the
level of gates and simple sequential elements such as latches or flip-flops.
The language itself makes no special distinction between constructs that
are ``easy'' or ``difficult'' to realize  in circuitry.  An implementation
may declare that a certain part of the language is {\em synthesizable,}
meaning that CSP written using only that subset can be automatically
compiled to an efficient hardware realization.  This distinction of a
synthesizable subset is entirely part of a given implementation and
will not be covered in this report.

\section{Perspective}

The vast majority of chip design today is performed in variants of the
Verilog language.  Verilog was introduced in 1984 as a means of
circuit description and simulation, and is subject to an IEEE
standardization process.  Over the years, the simplicity of the
original Verilog idea has been transformed into a veritable
encyclopedia of design, description, and verification capabilities.
Verilog today is used for many tasks throughout a chip-design process
where each task uses only a small subset of the language.  
Verilog shares some fundamental design aspects with CSP; in
particular, Verilog's semantics allow for descriptions with a very
high level of concurrency, and another similarity is that both
languages make it difficult (but not impossible) to allocate storage
space dynamically, as this is not something that has a sensible
implementation in silicon.  At the same time, the wide variety of
computational models supported by Verilog make that language extremely
difficult to describe and extremely demanding of programming
methodology---in particular when it comes to communication between
different system modules.  In contrast, CSP provides specific,
restricted communication constructs representing the rendezvous and
Martin's probe, which leads to highly structured communicating
processes.  Unlike the situation in Verilog, where the
programmer-designer must specify the details of sequencing for every
gate, CSP allows the implementation of sequencing to be abstracted into
the chosen implementation for the rendezvous.

\section{Background}

The design and implementation of Fulcrum CSP was done at Fulcrum Microsystems
in the early 2000s and continuing, after the company was acquired, at Intel
by the Fulcrum/Intel software team, including
Kevin Duncklee,
Mike Davies,
Aaron Denney,
Mika Nystr\"om,
Prasad Joshi,
Yi-hsin Weng,
Don Banh,
Piyush Prakash,
Mark Hesselink,
Kim Wallmark,
Patrick Pelletier,
Robert Karmazin,
Christopher Brichford,
Andrew Lines,
Peter Beerel,
Frederik Eaton,
David Hilvert,
Jesse Rosenstock,
and Harry Liu.

The present report came out of an effort by Mika Nystr\"om to write a
compiler for the Fulcrum CSP language at Intel in early 2025; this compiler
was written in Scheme and targets Modula-3 as the output format, and it
is available for download at \verb!github.com!.  Many dark
corners of the language had to be brought out and examined for the first
time in many years, and this document represents an attempt to capture
what was learned during that effort.

\section{Acknowledgments}
We aspire to be as clear as Greg Nelson's {\em
  Systems Programming with Modula-3\/} and the present editor
gratefully acknowledges the influence of that book, and parts of the
language definition have been shamelessly stolen from that book as
well as the Modula-3 Report from Digital's Systems Research Center (DEC SRC),
with contributions from Luca Cardelli, James Donahue, Lucille
Glassman, Mick Jordan, Bill Kalsow, Paul Rovner, and Greg Nelson.  We
also gratefully acknowledge that text and inspiration has been
borrowed from the Asynchronous Circuit Toolkit (ACT) of Rajit
Manohar's group at Yale.

\chapter{Language Definition}
% ========================================
% DEFINITIONS
% ========================================
\section{Definitions}
\label{sec:definitions}

A CSP program specifies a computation that acts on a sequence of
digital components called {\em locations.}  A CSP program describes
a {\em process\/} within a distributed system.  Within a single process,
a location can be referenced by and shared between different statements, but 
a single location is bound to only one process.

A {\em static error\/} is an error that the implementation must detect
before program execution.  Violations of the language definition are
static errors unless they are explicitly classified as runtime errors.

A {\em checked runtime error\/} is an error that the implementation
must detect and report at runtime.  The method for reporting such
errors is implementation-dependent.  This means that an implementation
is not required to halt execution on every runtime error; however, the
implementation must ensure that the execution of the program is {\em
  safe\/} regardless of the presence of errors.  This means that, even
in the presence of errors, the implementation must guarantee that the
runtime system is not corrupted in a way that makes further execution
of the program not faithful to the language semantics.

An {\em unchecked runtime error\/} is an error that is not guaranteed to
be detected, and can cause the subsequent behavior of the computation to
be arbitrary.  Unchecked runtime errors cannot occur in CSP.

An {\em identifier\/} is a symbol declared as a name for a variable,
type, port, function, etc.  The region of the program over a which a
declaration applies is called the {\em scope\/} of the declaration.
The meaning of an identifier is determined by the enclosing scope.
Declaring an identifier so that it is visible more than once in
a single scope is a static error.

A {\em variable\/} is a
set of locations that represents a mathematical value according to a
convention determined by the variable's {\em type.}  If a value can be
represented by some variable of type \verb!T!, then we say that the
value is a {\em member\/} of \verb!T! and \verb!T! {\em contains\/} the
value.

An {\em expression\/} specifies a computation that produces a value or
variable.  Expressions that produce variables are called {\em
  designators.}  A designator can denote either a variable or the
value of that variable, depending on the context.  Some designators
are {\em readonly,} which means that they cannot be used in contexts
that might change the value of the variable.  A designator that is not
readonly is called {\em writable.}   Expressions whose values can be determined
statically are called {\em constant expressions;} they are never
designators.

A {\em port\/} is a means for a process to communicate with other
processes.  A port can be either an {\em output port,} or an {\em
  inout port,} in which case values written to the port will appear in
another process, or an {\em input port,} in which case values will
appear at the port that were written in another process.

A {\em deadlock\/} is a state in which some number of processes are
permanently blocked from making progress.  An implementation is
permitted but not required to treat a deadlock as a runtime error.  If
an implementation is able to detect a {\em partial deadlock\/}, that
is, a deadlock involving only a subset of processes, it should offer
an option not to treat such a deadlock as an error.

\section{Comments}
Anything within a comment is ignored by the implementation.  Comments
are treated as whitespace, so they can be used to separate language
tokens that are required to be separated by whitespace.

CSP allows two types of comments:

Block comments are written
\begin{Verbatim}[samepage=true]
   /* This is a comment */
\end{Verbatim}
and allow newlines to be present for multi-line comments.  Anything
following \verb!/*!, including \verb!/*! itself, is ignored, until
the next occurrence of \verb!*/! in the file.  (That is, block
comments do not nest.)

Line comments are written
\begin{Verbatim}[samepage=true]
   // This is a comment 
\end{Verbatim}
and anything following \verb!//!, including the \verb!//! itself, until
the following newline, is part of the comment.

\section{Types}
\label{sec:types}

\subsection{Integer types}
\label{sec:integer-types}
There are three kinds of integer types: \verb!int!, {\em signed integers,}
and {\em unsigned integers.} \verb!int! contains zero, all positive integers
(Peano natural numbers), and the negation of the positive integers.

Signed integers are declared \verb!sint(W)! where \verb!W! is the
type's width in bits, which is a constant expression.  A type \verb!sint(W)! contains the values
$[-2^{{\tt W}-1}..2^{{\tt W}-1}-1]$.

Unsigned integers are declared \verb!int(W)! where \verb!W! is the
type's width in bits, which is a constant expression.  A type \verb!int(W)! contains the values
$[0..2^{{\tt W}}-1]$.

Note that all integer types contain the value zero.

\subsection{Boolean}

The type \verb!bool! contains two values: \verb!false! and \verb!true!.

In cases where a boolean variable is implicitly converted to \verb!int!,
\verb!false! is interpreted as 0 and \verb!true! is interpreted as -1.

\subsection{String}

The type \verb!string! contains all sequences of characters, where a
character is one of 256 constant values.  The visual representation of
the characters is system dependent but generally follows 7-bit ASCII
for the lower half (0..127) with a system-dependent upper half.

The shortest string is the empty string.  There is no ``null string'' in CSP.

\subsection{Structure types}
\label{sec:struct}

A {\em structure type\/} is a sequence of named variables, called the
{\em fields\/} of the structure type. Different fields can have
different types.  The name and type of each field is statically
determined by the structure type's declaration.

A structure type declaration has the form:
\begin{Verbatim}[samepage=true]
  structure T =
   (
    FieldList
   );
\end{Verbatim}
where \verb!FieldList! is a list of field declarations, each of which has the
form:
\begin{Verbatim}[samepage=true]
   Type fieldName = initial;
\end{Verbatim}
where \verb!fieldName! is an identifier, \verb!Type! is any type, and
\verb!initial! is a constant expression.  The field names must be
distinct.  Empty structure types are not allowed.


An instance of a structure type is called a {\em structure\/} and is declared
by stating the name of the structure type.  See sections~\ref{sec:declarations}
and~\ref{sec:struct-field-access}.


The constant \verb!initial! is a initial value used when a structure is
constructed.  The \verb!= initial! can be omitted, in which case the default
value for the type will be used for initialization.

\subsection{Array types}\label{sec:array-types}

An {\em array\/} is an indexed collection of component variables,
called the {\em elements\/} of the array.  The indexes are of type
\verb!int!.  The elements all have the same type,
called the {\em element type\/} of the array.

The {\em shape\/} of a multi-dimensional array is the sequence of its
lengths in each dimension.  More precisely, the shape of an array is
its length followed by the shape of any of
its elements; the shape of a non-array is the empty sequence.
The {\em layout\/} of a multi-dimensional array is the sequence of
its lengths and starting indexes in each dimension; the layout of a non-array
is the empty sequence.

All arrays are {\em fixed\/} in CSP, meaning that the layout of every array is
determined before program startup.

An array declaration has the form:
\begin{Verbatim}[samepage=true]
  Element Id Layout;
\end{Verbatim}
where \verb!Element! is the element type and \verb!Layout! represents
the layout of the array.  The layout can be specified in one of
two equivalent ways: either as a comma separated list of ranges \verb!lo..hi!
between brackets or as a list of ranges \verb!lo..hi! within separate brackets:
\begin{Verbatim}[samepage=true]
  T a[0..2,0..2];
  T b[0..2][0..2];
\end{Verbatim}
here \verb!a! and \verb!b! have equivalent declarations.  Any array dimension
can also be specified with a single number \verb!length! in which case the
index ranges over \verb!0..length - 1!.  Hence the following are equivalent:
\begin{Verbatim}[samepage=true]
  T c[0..2,0..5];
  T d[3][6];
\end{Verbatim}

Types that are not array types are called {\em scalar types,} and instances
of such types are called {\em scalars.}



   


\subsection{Packed and fixed-width types}
The following types are of fixed width: signed integers,
unsigned integers, and booleans.

If a structure type contains only fixed-width fields, then the
structure type is a {\em packed structure type.}  A packed structure
type is itself of fixed width.

If the element type of an array is of fixed width, then the array is a
{\em packed array type.}  A packed array type is itself of fixed width.

By elimination, the following types are not of fixed width: strings,
the \verb!int! type, and structure types and arrays containing strings
or the \verb!int! type, whether directly or indirectly.


\subsection{Ports}

Two kinds of ports are provided for processes to communicate with
other processes.

\begin{itemize}
\item {\em Channel ports\/} (section~\ref{sec:chanport})

\item {\em Node ports\/} (section~\ref{sec:nodeport})
\end{itemize}

\subsection{Channels}\label{sec:chanport}

A channel is used for communication between two concurrent processes, using
the rendezvous operations send and receive.
Formally, a channel is an ordered 6-tuple $(S, P_S, R, P_R, W,
\Sigma)$ where $S$ is a process called the {\em sender,} $P_S$ is the
name of the output port in the sender, $R$ is a process called the
{\em receiver,} $P_R$ is the name of the input port in the receiver,
$W$ is an integer called the {\em width\/} of the channel, and
$\Sigma$ is an integer called the {\em slack\/} of the channel.
Channels are not declared within the CSP language but described
externally.  Values transacted over a channel are of type
\verb!int(W)! (see section~\ref{sec:integer-types}.)


Within a CSP process, a channel is visible either as an input port or
as an output port.  Hence, channels are point-to-point and have exactly
one sender and exactly one receiver.  An implementation may permit
ports to ``loop back,'' that is to connect from a process to itself.

Channel ports may be the elements of an array.

\subsection{Nodes}\label{sec:nodeport}

A {\em node\/} is a shared boolean variable that is attached to a node
port of one or more processes.
Formally, a node is a set of associations
\[ \{ (A_0, P_0), (A_1, P_1) \ldots (A_{n-1},P_{n-1}) \} \]
  where $A_i$ is a process with access to the node and $P_0$ is the
  name of the node port used within $A_i$ to access the node.


The port $P_i$ attached to a node can be
read and written as a local variable within $A_i$, and it can be used in any
context where a scalar boolean local variable could be used.  Node
ports may be {\em input node ports, output node ports,} or {\em inout
  node ports,} and like channel ports are declared externally to the CSP
language.  Node ports access their associated nodes: an assignment to a node
port results in an assignment to the node, and inspecting the value of
a node port yields the value of the node.


Node ports may be the elements of an array.

If a node \verb!n! is assigned to (possibly repeatedly) by a single
process \verb!P! (in which the node is attached to an output or inout
node port) and also available to a set of other observing processes
\verb!Q_0..Q_n-1!  (in each of which \verb!n! is attached to an input
or inout node port) then the final assignment within \verb!P! will
eventually be observed as the value of \verb!n!  by \verb!Q_0..Q_n-1!,
but it is not specified in what order the observers will observe the
update.

If a node \verb!n! is assigned to by more than one process
\verb!P_0..P_n-1!, then the observing processes will eventually
observe \verb!n! to have a value determined by some sequential
interleaving of the assignments performed by \verb!P_0..P_n-1!.

Nodes that are members of arrays may be treated by the implementation
as independent of each other; that is, in case of contention, one
node in the array may converge to a value \verb!w[k]! independently of
and notwithstanding the convergence of another node in the array to a
value \verb!w[l]!.  

The implementation may check that nodes are properly reset at the beginning
of execution and treat a failure of a program to do so as a runtime error.
See section~\label{sec:resetfunction}.


\section{Functions}
\label{sec:functions}
A {\em function\/} is a triple consisting of:
\begin{itemize}
\item the {\em name,} which is an identifier.
  
\item the {\em body,} which is a statement.

\item the {\em signature,} which specifies the function's formal parameters and
  result type.
\end{itemize}
A function that returns a result is also called a {\em function procedure\/}; a function that does not return a result is called a {\em proper procedure.}

A function definition has the form:
\begin{Verbatim}[samepage=true]
   function F sig = ( body );
\end{Verbatim}
where \verb!body! is the body, and \verb!sig! is a signature specification,
which has the form:
\begin{Verbatim}[samepage=true]
   (formal0 ; ...; formaln-1) : R
\end{Verbatim}
where each \verb!formali! is a formal parameter declaration, as described below and \verb!R! is the result type, which can be any type.  The ``\verb!: R!'' can
be omitted, making the signature that of a proper procedure.

A formal parameter declaration has the form
\begin{Verbatim}[samepage=true]
   Type Mode Name Layout = Initial
\end{Verbatim}
where
\begin{itemize}
\item \verb!Type! is the type of the parameter if the parameter is scalar, or the element type of the parameter if the parameter type is an array type
  
\item \verb!Mode! is a parameter mode, which can be \verb!-!,
  \verb!+!, or \verb!+-!.  \verb!Mode! can be omitted, in which case
  it is taken to be \verb!+-!.  See section~\ref{sec:funccall} for a
  discussion of the semantics of the parameter modes.

\item \verb!Name! is an identifier that names the parameter.

\item \verb!Layout! specifies the array layout for an array parameter; if it is
  omitted, the parameter is scalar.

\item\verb!Initial! specifies the initial value of a scalar parameter whose mode is \verb!+!.  \verb!= Initial! can be omitted.  
\end{itemize}
When a series of parameters share the same type, \verb!Mode Name Layout = Initial! can be a replaced with a comma-separated list of mode, name, layout, and initial value, in which case the type is repeated for each parameter.  That is:
\begin{Verbatim}[samepage=true]
   Type Mode0 Name0 Layout0 = Initial0, Mode1 Name1 Layout1 = Initial1, ...
\end{Verbatim}
is shorthand for
\begin{Verbatim}[samepage=true]
   Type Mode0 Name0 Layout0 = Initial0; Type Mode1 Name1 Layout1 = Initial1; ...
\end{Verbatim}
A function definition is also a function declaration.

\section{Initialization}\label{sec:initialization}

All declared variables, whether explicitly declared or implicitly
declared as type \verb!int! by being used without a declaration, are
initialized before their first use.

In the following rules (and only in the following rules), the word
``variable'' is taken to encompass all declared variables of the basic
integer, boolean, and string types whether declared as identifiable
variables in their own right or as structure fields or array elements.

\begin{itemize}
\item If a variable declaration carries
  an initializer, that value is used.

\item If a variable is introduced as a
function parameter without \verb!-! in its parameter mode and an initializer
is present, that initializer is used.

\item If a variable is introduced as a function parameter with \verb!-! in its
  parameter mode, the variable is initialized by copying the value of
  the actual parameter at the time that the function is called.
  
\item If a variable appears as a field
  within a struct and the structure type declaration contains an initializer
  for that field, that initializer is used.

\item In all other cases, the default initializer for the variable is used.
\end{itemize}

The default initializers are:
\begin{itemize}
\item For all integer types, zero

\item For the \verb!bool! type, \verb!false!

\item For the \verb!string! type, the empty string, written \verb!""!
\end{itemize}



\section{Statements}
\label{sec:statements}

Executing a statement produces a computation that can 
terminate (halt), cause a checked runtime error, or loop forever.

\subsection{Skip}

The no-op {\em skip\/} statement
\begin{Verbatim}[samepage=true]
   skip
\end{Verbatim}
when activated in any state, terminates in the same state.

\subsection{Error}

The abortion statement {\em error\/} 
\begin{Verbatim}[samepage=true]
   error
\end{Verbatim}
when activated in any state, causes a checked runtime error and does
not terminate.  (It is called {\em abort\/} in most other languages.)

\subsection{Assignment}

We specify assignment rules by enumeration.

\subsubsection{General assignability rule}

A value is assignable to a variable of the same type.  If the type of a
value differs from the target variable, the value is not assignable to
the variable, with the exceptions below.

\subsubsection{Integer assignment}

The three kinds of integer types are all assignable to each other by
virtue of the fact that all integer right-hand-sides are interpreted
as being of type \verb!int!.  Assigning a value of type \verb!int! to a
variable of type \verb!int(W)! or \verb!sint(W)! is done by interpreting
the RHS as a two's-complement number and performing a bitwise
assignment into the locations representing the LHS, also interpreted
as a two's-complement number, discarding any bits that do not fit in
the target locations.


\subsubsection{Array assignment}

Arrays are assignable if they have the same shape and their element types
are assignable.

\subsubsection{Integer to string assignment}

An integer value is assignable to a string variable.  The value assigned
to the string variable is the human-readable string corresponding
to the value of the integer in decimal base.  For example, the assignment
\begin{Verbatim}[samepage=true]
   s = "The answer is " + (4 * 10 + 2)
\end{Verbatim}
will result in \verb!s! containing the string:
\begin{Verbatim}[samepage=true]
   The answer is 42
\end{Verbatim}

\subsubsection{Expression assignment}\label{sec:expr-ass}

An expression \verb!e! is {\em assignable\/} to a type \verb!T! if \verb!e! is
assignable to a variable of type \verb!T!.

An assignment statement has the form:
\begin{Verbatim}[samepage=true]
   v = e
\end{Verbatim}
where \verb!v! is a writable designator and \verb!e! is an expression
assignable to the variable designated by \verb!v!.  The statement sets
\verb!v! to the value of \verb!e!.  \verb!v! is always evaluated before
\verb!e!, and \verb!v! is updated after \verb!e! is evaluated.  In
particular, if \verb!v! and \verb!e! are overlapping subarrays, the
assignment is performed in such a way that no element is used as a
target before it is used as a source.

\subsubsection{Boolean assignment shorthands}

A statement of the form
\begin{verbatim}
   v-
\end{verbatim}
where \verb!v! is a writable designator referencing a boolean variable
is equivalent to the statement \verb!v = false!.

A statement of the form
\begin{verbatim}
   v+
\end{verbatim}
where \verb!v! is a writable designator referencing a boolean variable
is equivalent to the statement \verb!v = true!.

\subsubsection{Assignment by operation shorthands}

CSP provides the assignment-by-operation shorthands
\verb!+=!,
\verb!-=!,
\verb!*=!,
\verb!/=!,
\verb!%=!,
\verb!&=!,
\verb!|=!,
\verb!^=!,
\verb!<<=!,
and
\verb!>>=!.
For each assignment-by-operation statement of the form
\begin{Verbatim}[samepage=true]
   v Op= e
\end{Verbatim}
where \verb!Op=! is the token formed by appending the suffix \verb!=! to
an operation token,
this is equivalent to 
\begin{Verbatim}[samepage=true]
   v = v Op e
\end{Verbatim}
with the exception that \verb!v! is evaluated only once.  For example,
\begin{Verbatim}[samepage=true]
   v *= w
\end{Verbatim}
means 
\begin{Verbatim}[samepage=true]
   v = v * w
\end{Verbatim}
except that \verb!v! is only evaluated once.

See section~\ref{sec:expressions} for a specification of each operation \verb!Op! in this context.

The two special operations \verb!x++! and \verb!x--! are syntactic sugar for
\verb!x += 1! and \verb!x -= 1! respectively.


\subsection{Function call}\label{sec:funccall}

A function call has the form:
\begin{Verbatim}[samepage=true]
   F(Bindings)
\end{Verbatim}
where \verb!F! is the name of a function and \verb!Bindings! is a list
of bindings.  A binding has the form \verb!actual!, where \verb!actual! is
an expression, which is called the {\em actual parameter.}  If the list of bindings is empty, the parentheses are still
required.

Function parameters are passed by value-result.  This means that for
each formal parameter that has \verb!-! in its mode, the binding is turned
into an assignment
\begin{Verbatim}[samepage=true]
   Name = actual
\end{Verbatim}
where \verb!Name! is the name of the formal parameter.  This assignment is evaluated before
the body of the function (during the {\em function prolog\/}).  For a formal that does not have \verb!-! in its
mode, the binding is instead turned into an assignment
\begin{Verbatim}[samepage=true]
   Name = initial
\end{Verbatim}
where \verb!initial! is the initial value from the function signature
for that parameter, or a declaration without an initialization, subject
to the normal initialization rules, if no initial value is provided in the
function signature.
For each formal parameter that has \verb!+! in its mode,
the binding is turned into an assignment
\begin{Verbatim}[samepage=true]
   actual = Name 
\end{Verbatim}
which is evaluated after the body of the function (during the {\em function epilog\/}).

Finally, for function
procedures, the return value of the function is declared using the function's
own name (e.g., \verb!F!) before the
body of the function as having type \verb!R! (the return type of the function),
and upon return of the function the returned value is \verb!F!.

To execute the call, the function's parameters are evaluated
left-to-right,
\begin{itemize}

\item the formal parameters are assigned (the prolog)

\item the body of the procedure is executed

\item and \verb!+! parameters and the
  function result (if present) are assigned (the epilog)
\end{itemize}

For example, consider:
\begin{Verbatim}[samepage=true]
  function F(int +x = 12, -y) : int = ( x = x * y; F = x + y );
  q = 2;
  z = F(q, 3);
\end{Verbatim}
following the rules above, this is equivalent to:
\begin{Verbatim}[samepage=true]
  q = 2;

  // prolog
  int x = 12; 
  int y = 3;
  int F;

  // body
  x = x * y; // x is 36
  F = x + y; // F is 39

  // epilog
  q = x;     // q is 36

  // return site
  z = F;     // z is 39
\end{Verbatim}
where if necessary \verb!x! and \verb!y! are renamed so they do not
clash with existing variables in the current scope.  (\verb!F! does
not need to be renamed since it would already clash with an existing
variable of the same name.)

If \verb!+! parameters are aliased, that is, if following the value-result
rules listed above would lead to a situation in which
two or more \verb!+! parameters reference the same locations, the two
values are independent of each other during execution of the function,
but the order in which the results are assigned in the function epilog
is indeterminate.

\subsection{Sequential composition}\label{sec:sequential-composition}
A statement of the form
\begin{Verbatim}[samepage=true]
   S1; S2
\end{Verbatim}
represents {\em sequential composition,} and is operationally defined that it
executes \verb!S1!, and then if the outcome is normal, executes \verb!S1!.  If
the outcome of \verb!S1! is other than terminating normally, \verb!S2! is ignored.

\subsection{Parallel composition}\label{sec:parallel-composition}
A statement of the form
\begin{Verbatim}[samepage=true]
   S1, S2
\end{Verbatim}
represents {\em parallel composition,} and is operationally defined
that when activated, \verb!S1! and \verb!S2! are executed in some
arbitrary order, or even in such a way that if there are observable
states within \verb!S1! and \verb!S2!, the executions of \verb!S1!
and \verb!S1! may be interleaved in some arbitrary way, or even
executed entirely in parallel.  If either of \verb!S1! or \verb!S2!
loops forever, the other statement still executes.  If either of
\verb!S1!  or \verb!S2! terminates with a runtime error, the
implementation is allowed to terminate the execution of the other
statement.

Syntactically, \verb!,! binds tighter than \verb!;!.

\subsection{Parenthesization}

Parentheses \verb!(! \verb!)! can be used around statements and statement
compositions.  For example,
\begin{Verbatim}[samepage=true]
   S0; S1, S2; S3
\end{Verbatim}
means ``first execute \verb!S0!, then execute \verb!S1! and \verb!S2! in parallel, and finally execute \verb!S3!'' whereas
\begin{Verbatim}[samepage=true]
   (S0; S1), (S2; S3)
\end{Verbatim}
means ``execute \verb!S0! followed by \verb!S1! in parallel with executing \verb!S2! followed by \verb!S3!.''



\subsection{Guarded commands}

A {\em guarded command list\/} is a statement of the form:
\begin{Verbatim}[samepage=true]
  Start G_0 -> S_0 Sep G_1 -> S_1 Sep ... G_n-1 -> S_n-1 Sep else -> S_n End
\end{Verbatim}
where


\begin{itemize}
\item  \verb!Start! is a starting bracket, either \verb!*[!, \verb![!, or \verb!#[!.
\item the \verb!G_i! are boolean expressions called the {\em guards.}
\item the \verb!S_i! are statements called the {\em guarded commands.}
\item \verb!Sep! is a separator, either \verb![]! or \verb!:!.
\item \verb!End! is the ending bracket \verb!]!.
\end{itemize}

\subsubsection{Selection statement}
When \verb!Start! is \verb![!, the statement is a {\em selection statement.}
When the selection statement is activated,
execution is suspended until at least one \verb!G_i! is true, at which time
a \verb!S_j! for which \verb!G_j! is true is executed, after which
the selection statement terminates.

The shorthand \verb![ G ]! is short for \verb![ G -> skip ]! and hence means
``wait until \verb!G! holds.''

\verb!else -> S_n! may be omitted.  If present, \verb!else! is interpreted
as the expression \verb?!G_0 & !G_1 & ... & !G_n-1?.

Activation of a selection statement without an \verb!else! in a state
where none of \verb!G_0..G_n-1! holds and the implementation is able to
detect that none of \verb!G_0..G_n-1! can ever hold (e.g., because they
refer only to local variables of the process containing the statement) may
be treated by the implementation as an error.

\subsubsection{Simple if statement}
When \verb!Start! is \verb!#[!, the statement is a {\em simple if statement.}
When the selection statement is activated,
an arbitrary \verb!Sj! for which \verb!G_j! is true is executed, after which
the selection statement terminates.  If no \verb!G_j! is true, the simple
if statement is equivalent to \verb!skip!.

The simple if statement:
\begin{Verbatim}[samepage=true]
   #[ G_0 -> S_0 [] G_1 -> S_1 [] ... [] G_n-1 -> S_n-1 ]
\end{Verbatim}
is equivalent to the selection:
\begin{Verbatim}[samepage=true]
    [ G_0 -> S_0 [] G_1 -> S_1 [] ... [] G_n-1 -> S_n-1 [] else -> skip ]
\end{Verbatim}

An \verb!else! clause is not permitted in the simple if statement.


\subsubsection{Repetition statement}
When \verb!Start! is \verb!*[!, the statement is {\em a repetition
  statement.}  When the repetition statement is activated, it amounts
to the execution of \verb!skip! if no \verb!G_i! is true; otherwise an
arbitrary \verb!S_j! for which \verb!G_j! is true is executed, after which
the repetition statement is repeated.

The shorthand \verb!*[ S ]! is short for \verb![ true -> S ]! and hence means
``repeat \verb!S! forever.''

\verb!else -> Sn! may be omitted.  If present, \verb!else! is interpreted
as the expression  \verb?!G_0 & !G_1 & ... & !G_n-1?. 

\subsubsection{Deterministic and nondeterministic guarded commands}
When \verb!Sep! is \verb![]!, the selection or repetition statement
is a {\em deterministic\/} selection or repetition statement.  This means
that the programmer-designer guarantees that either exactly zero or
exactly one guard is true at all times.  When \verb!Sep! is \verb!:!,
the selection or repetition statement is a {\em nondeterministic\/} selection
or repetition statement, and there is no restriction on the number
of guards that may be true.  The implementation is permitted to use
a more expensive implementation for nondeterministic than for
deterministic selections and repetitions.


\subsubsection{Evaluation order of guarded commands}
The order in which the the guards are evaluated is indeterminate, and
the number of times that the guards are evaluated is also indeterminate,
but if command \verb!S_i! executes, the guard \verb!G_i! has been evaluated
at least once, and this evaluation has taken place before the evaluation
of \verb!S_i! has begun.


\subsection{The loop statement}\label{sec:loop-statement}
The loop construct is
\begin{Verbatim}[samepage=true]
   < Op Id : Range : Statement >
\end{Verbatim}
where
\begin{itemize}

\item \verb!Op! is either \verb!;! or \verb!,! referencing sequential composition (section~\ref{sec:sequential-composition}) or parallel composition (section~\ref{sec:parallel-composition}),

\item  \verb!Id! is an identifier,
  
\item \verb!Range! is an integer range \verb!Lo .. Hi! or
  an integer value \verb!N!, which is interpreted identically to \verb!0 .. N - 1!,
\item \verb!Statement! is a CSP statement (including a sequential or parallel composition)
\end{itemize}
The loop statement is interpreted as \verb!Hi - Lo + 1! textual
repetitions of \verb!(Statement)! with the name \verb!Id! bound to the
values \verb!Lo! to \verb!Hi! in succession, with \verb!Op! as a
separator.  This means that within \verb!Statement!, \verb!Id! takes
on the constant \verb!int! value of the index of that iteration of the
loop, and \verb!Id! is a readonly designator.  The empty range is
allowed and if present, the loop is equivalent to
\verb!skip!. \verb!Lo! and \verb!Hi! need not be compile-time
constants, but an implementation is allowed to require that \verb!Lo!
and \verb!Hi! be compile-time constants when \verb!Op! is \verb!,!.

\verb!Op! can be omitted, in which case the statement is treated as if
\verb!Op! were \verb!;!.


\subsection{The rendezvous}\label{sec:rendezvous}

The rendezvous is implemented with two kinds of statements: {\em
  send,} and {\em receive,} collectively called {\em communications
  actions.} The rendezvous operations take place on ports and are used
for {\em synchronization\/} and {\em distributed assignment\/} between
{\em communicating processes.}  See section~\ref{sec:chanport} for
details on ports and channels.

\subsubsection{Slack}\label{sec:slack}

Communications actions on the two ports of a channel are synchronized.
Call the number of send actions that have been completed on a channel
$cS$ and the number of receive actions on that channel $cR$.  Since values
cannot be received before they are sent, we have that
\[
cR \leq cS
\]
and the difference $cS - cR$ is the number of messages sent that have not
yet been received.  The slack $\Sigma$ of the channel is the limit on this
difference, so that:
\[
0 \leq cS - cR \leq \Sigma 
\]
The sender and receiver processes will be suspended if a send or receive
action would violate this constraint. 

\subsubsection{Send}

The send statement is
\begin{Verbatim}[samepage=true]
   OPort ! e
\end{Verbatim}
where \verb!OPort! an output port of width \verb!W! and
\verb!e! is an expression assignable to a variable of type \verb!int(W)!.
(Compare to section~\ref{sec:expr-ass}.)

The statement \verb,OPort ! e, results in an assignment to a writable
designator in the receiver associated with the channel of the value
\verb!e!.  If the channel is not ready, the sender is suspended until
the channel is ready, at which time the sender proceeds and $cS$ for
the channel is incremented.  See section~\ref{sec:slack}.


\subsubsection{Receive}\label{sec:receive}

The receive statement is
\begin{Verbatim}[samepage=true]
   IPort ? v
\end{Verbatim}
where \verb!IPort! is an input port of width \verb!W!
and \verb!v! is a writable designator assignable from an expression
of type \verb!int(W)!.
(Compare to section~\ref{sec:expr-ass}.)

The statement \verb,IPort ? v, results in the assignment to \verb!v!
of the value \verb!e! sent on the associated port in the sender.  If the
value has not yet been sent by the sender, the receiver is suspended
until the sender has sent the value, at which time the receiver proceeds
and $cR$ for the channel is incremented.
See section~\ref{sec:slack}.

See also section~\ref{sec:peek}.



\section{Declarations}
\label{sec:declarations}
CSP declarations in a program occur in a specified order.  Certain
constants, functions, and structure types can be pre-declared by the
CAST environment of a CSP program.  Within a single CSP program,
the sequence of declarations is fixed:
\begin{Verbatim}[samepage=true]
  structure type declarations;
  function declarations;
  program body
\end{Verbatim}
The program body is evaluated when the system starts to run, but
after initialization (see sections~\ref{sec:initialization} and~\ref{sec:resetport}).  The evaluation
order of different CSP programs within a system is unspecified.
CSP does not allow forward references: therefore, an implication of the
declaration order is that functions cannot access variables declared in
the program body.  Functions can however access structure type declarations
and constants declared in CAST.

Declarations of structure types are described in
section~\ref{sec:struct}, and declarations and definitions of
functions are described in section~\ref{sec:functions}.

Variables are declared anywhere in the program body.  The program body is a
scope, and a guarded command is a scope.  A variable declaration has
the format
\begin{Verbatim}[samepage=true]
   Type Id Layout ;
\end{Verbatim}
\begin{itemize}
\item \verb!Type! is the type of the variable, if the variable is scalar, or the type of an element, if the variable is an array.
\item \verb!Id! is an identifier that becomes bound to the variable
\item \verb!Layout! is optional, and if present, declares that the variable is an array of \verb!Type! with the specified layout (see section~\ref{sec:array-types}.)
\end{itemize}

Variables that appear in the program text without being declared are implicitly
declared to have type \verb!int! (section~\ref{sec:integer-types}).


% ========================================
% EXPRESSIONS
% ========================================
\section{Expressions}
\label{sec:expressions}

An expression prescribes a computation that produces a value or
variable.  Syntactically, an expression is an operand, or an operation
applied to arguments, which are themselves expressions.  Operands are
identifiers or literals.  An expression is evaluated by recursively
evaluating its arguments and performing the operation.  The order of
argument operation is left-to-right for every subexpression at equal
precedence, and subexpressions are evaluated exactly once, except
for subexpressions of \verb!&&! and \verb!||! (see section~\ref{sec:shortcircuit}).

\subsection{Conventions for describing operations}\label{op-conventions}

To describe the argument and result types of operations, we use a
notation like function declarations (section~\ref{sec:functions}).
But since most operations are too general to be described by a true
function signature, we extend the notation in several ways.

The argument to an operation can be required to have a type in a
particular class, such as an integer type.  In this case the formal parameter
specifies a type class instead of a type.
Type classes we use, as opposed to regular CSP types, are
\verb!Integer! for an integer of any type, and \verb!Structure! for any
structure type.

For example to describe the
signature of the addition operator \verb!+! we use a function-style
declaration:
\begin{Verbatim}[samepage=true]
   + (Integer -a, -b) : int
\end{Verbatim}

A single operation name can be overloaded, which means that it denotes
more than one operation.  In this case, we write a separate signature
for each of the operations, each alternative signature on a line of
its own.  For example:
\begin{Verbatim}[samepage=true]
   + (Integer -a, -b) : int
   + (string  -a, -b) : string
\end{Verbatim}
The particular operation will be elected so that each actual argument type
is of the corresponding formal parameter type or a member of the corresponding
formal parameter type class.


\subsection{Literals}

Numeric, boolean, and string literals.

\subsubsection{Numeric literals}

Numeric literals denote constant non-negative integers.  Numeric literals
may not contain spaces: the spaces we use below serve to separate components
of the numeric literals for discussion.
Numeric literals can be specified in a four different ways:
\begin{itemize}

\item A sequence of digits is interpreted in base 10 (decimal).

\item The sequence \verb!0x! followed by hexadecimal digits (\verb!0!--\verb!9! and \verb!a!--\verb!f! and \verb!A!--\verb!F! ) is interpreted as a base 16 (hexadecimal) number.  Example: the number forty-two can be written \verb!0x2a!,
  \verb!0x002a!, or \verb!0x2A!
  
\item The sequence \verb!0b! followed by binary digits (\verb!0!--\verb!1!) is interpreted as a base 2 (binary) number.  Example: \verb!0b101010!

\item The sequence \verb!Base _ Number! where \verb!Base! is a base itself expressed in decimal and \verb!Number! is interpreted as a number in that base, expressed in digits
  \verb!0!--\verb!9! and \verb!a!--\verb!z! and \verb!A!--\verb!Z!.
  Example: \verb!21_20!
\end{itemize}

\subsubsection{Boolean literals}

The boolean literals are \verb!false! and \verb!true!.  The type of boolean
literals is \verb!bool!.

\subsubsection{String literals}

A string literal is a pair of double quotes enclosing a sequence of
ISO-Latin-1 printing charcters (excluding double quote) and escape sequences.
The type of a string literal is \verb!string!.
\begin{Verbatim}[samepage=true]
   " String "
\end{Verbatim}
The legal escape sequences are: \verb!\n!, \verb!\r!, \verb!\t!,
\verb!\b!, \verb!\f!, \verb!\"!, \verb!\'!, \verb!\\!, and arbitrary
8-bit characters can be written \verb!\xHexNum!, where \verb!HexNum!
is a hexadecimal number from 0--255, or as \verb!\OctNum!, where
\verb!OctNum! is an octal number from 0--255.  For example, the
ASCII character \verb!*! can be written \verb!\02a! or \verb!\052!.

\subsection{Integer expressions}

Every integer expression has type \verb!int! regardless of the types
of its operands.

\subsection{Operation syntax}\label{sec:opsyntax}

The operators that have special syntax are classified and listed in order
of decreasing binding power in the following table:

\begin{tabular}{ll}
  \verb,x::a,                 & structure field \\
  \verb,f(x) a[i] b{hi:lo},   & applicative \verb!(!, \verb![!, \verb!{! \\
  \verb,- ~,                  & prefix negation, prefix not \\
  \verb,++ --,                & postfix arithmetic \\
  \verb,**,                   & infix exponentiation \\
  \verb,* / %,                & infix arithmetic \\
  \verb,+ -,                & infix arithmetic \\
  \verb,<< >>,                & infix bitwise shifts \\
  \verb,< <= >= >,            & infix relations \\
  \verb,== !=,                & infix relations \\
  \verb,&,                    & infix bitwise AND \\
  \verb,^,                    & infix bitwise XOR \\
  \verb,|,                    & infix bitwise OR \\
  \verb,&&,                   & infix logical AND \\
  \verb,||,                   & infix logical OR \\
\end{tabular}

All infix operators are left associative except exponentiation \verb!**!, which
is right associative.

\subsection{Designators}\label{sec:designators}

An identifier is a writable designator if it is declared as a
variable, as an output node or inout node, an array of output nodes or
inout nodes, , or is a function parameter.  An identifier is a
readonly designator if it is an input node, an array of input nodes, the index of a loop
statement (section~\ref{sec:loop-statement}), or the index of a loop
expression (section~\ref{sec:loop-expression}).  CAST constants are
not designators: they are constant expressions.

The only operations that produce designators are subscripting, array slicing,
bit accessing, and structure-field accessing.


\subsection{Function application}

A function call is an expression if the function returns a result (is
a function procedure).  The type of the expression is the result type
of the function.

\subsection{Bit, array, and struct accessors}

The only expressions that may create designators are bit-slicing,
array, and struct accessor expressions.

\subsubsection{Bit access}
If \verb!e! is an integer expression, a range of bits of \verb!e! can
be accessed using the syntax
\begin{Verbatim}[samepage=true]
   e { Range }
\end{Verbatim}
where \verb!Range! is a range \verb!Hi : Lo! where \verb!Hi! and \verb!Lo!
are of type \verb!int!, or a single number \verb!k!, which is interpreted as
\verb!k : k!.  If the range is empty (that is, if $\verb!Hi! < \verb!Lo!$) then, the value of the expression is zero.
The bit-access expression is a designator if \verb!e! is a designator; it is
a writable designator if \verb!e! is a writable designator.  The type of
\verb!e { Range }! is \verb!uint(Hi - Lo + 1)!.


\subsubsection{Array access}
If \verb!e! is an array expression of any type, a {\em slice\/} of the array
is denoted by
\begin{Verbatim}[samepage=true]
   e [ I_0 , I_1, I_2 , ... , I_n-1]
\end{Verbatim}
where \verb!I_0!..\verb!I_n-1! are indexes into the first \verb!n! dimensions
of the array; \verb!I_0!..\verb!I_n-1! must be integers.

If \verb!n! is equal to the dimensionality of \verb!e!, the value of
the expression is an element of \verb!e!.  If \verb!n! is less than
the dimensionality of \verb!e!, the value of the expression is the
indicated subarray of \verb!e!.  It is a static error for \verb!n! to
be greater than the dimensionality of \verb!e!, and it is a checked
runtime error for any of the indexes to overrun the length of any
corresponding dimension of the array.  The array-access expression is
a designator if \verb!e! is a designator; it is a writable designator
if \verb!e! is a writable designator.  The type of
\verb!e [ I_0 , I_1, I_2 , ... , I_n-1]! is the element type of \verb!e!
  if \verb!n! is equal to the dimensionality of \verb!e!, else it is
  an array with the shape of the remaining unspecified dimensions of
  \verb!e!.

Note that ranges
cannot be used as array accessors even though they can be used to
declare arrays: this implies that ``partial array slices'' are not
available in CSP.

\subsubsection{Struct field access}\label{sec:struct-field-access}
If \verb!e! is a struct, the expression
\begin{Verbatim}[samepage=true]
  e :: Field
\end{Verbatim}
where \verb!Field! names a field of the structure type of \verb!e!, denotes
the value of the corresponding field.  The struct-access expression is a designator if \verb!e! is a designator; it is
a writable designator if \verb!e! is a writable designator.  The type of
\verb! e :: Field! is the declared type of the field.


\subsection{Arithmetic and string operators}

The basic arithmetic operators are built into the language.  

\begin{tabular}{rl}
{\em infix} & \verb!+  (Integer -x, -y)         : int! \\
{\em infix} & \verb!+  (string  -x, -y)         : int! \\
{\em infix} & \verb!+  (Integer -x, string  -y) : int! \\
{\em infix} & \verb!+  (string  -x, Integer -y) : int! \\
\end{tabular}

As an infix operator on integer arguments, \verb!+! denotes addition.

As an infix operator on strings, \verb!+! denotes string concatenation.

As an infix operator where one argument is a string and the other an integer, \verb!+! denotes string concatenation with the integer argument converted to a
string denoting the number in decimal.

\begin{tabular}{rl}
{\em prefix} & \verb!-  (Integer -x)           : int! \\
{\em infix}  & \verb!-  (Integer -x, -y)       : int! \\
\end{tabular}

As a prefix operator, \verb!-! denotes integer negation.
As an infix operator on integer arguments, \verb!-! denotes subtraction.

\begin{tabular}{rl}
{\em infix}  & \verb!*  (Integer -x, -y)          : int! \\
{\em infix}  & \verb!/  (Integer -x, -y)          : int! \\
{\em infix}  & \verb!%  (Integer -x, -y)          : int! \\
\end{tabular}

\verb!*! denotes integer multiplication. For \verb!x! and \verb!y! both
positive or both negative, \verb!x / y! is the floor of the quotient
of \verb!x! and \verb!y!.  For one of \verb!x! and \verb!y! negative
and the other positive, \verb!x / y! is the ceiling of the quotient of
\verb!x! and \verb!y!.  For \verb!y! equal to zero, \verb!x / y! is zero.
For any integers \verb!x! and \verb!y!, \verb!x % y! is defined to be
equal to \verb!x - y * (x / y)!.

\begin{tabular}{rl}
{\em infix}  & \verb!>>  (Integer -x, -y)          : int! \\
{\em infix}  & \verb!<<  (Integer -x, -y)          : int! \\
\end{tabular}

The \verb!<<! and \verb!>>! operators interpret the first operand as
being a two's-complement representation of a number.  \verb!x << y!
denotes left shift, that is, when \verb!y! is positive, the value \verb!x!
shifted \verb!y! bit positions to the left, with the bottom \verb!y!
bits of the result being zero. \verb!x >> y! with \verb!y! positive
denotes right shift of \verb!x!, \verb!y! bit positions to the right.
If \verb!x! is negative, it is sign extended.  If \verb!y! is negative
then \verb!x >> y! is defined as \verb!x << -y!, and \verb!x << y! is defined
as \verb!x >> -y!.

\begin{tabular}{rl}
  {\em infix}  & \verb!**  (Integer -x, -y)          : int! \\
 \end{tabular}

As an infix operator on integers, \verb!**! denotes exponentiation:
\verb!x ** y! is equal to \verb!x! raised to the \verb!y!th power.
Unlike all the other operators, \verb!**! is right-associative.

\subsection{Relations}

Relations are operators that have \verb!bool! as their range.

\begin{tabular}{ll}
{\em infix} & \verb!<  (Integer -x, -y)     : bool! \\
{\em infix} & \verb!<= (Integer -x, -y)     : bool! \\
{\em infix} & \verb!>  (Integer -x, -y)     : bool! \\
{\em infix} & \verb!>= (Integer -x, -y)     : bool! \\
\end{tabular}

The regular less than, less than or equal to, greater than, greater than
or equal to operators operate on two integers \verb!x! and \verb!y! and
compute a boolean according to the normal rules.  As with all other
expressions in CSP, the operands are interpreted as \verb!int! regardless
of type.

\begin{tabular}{ll}
{\em infix} & \verb!== (Integer -x, -y)     : bool! \\
{\em infix} & \verb!== (bool -x, -y)        : bool! \\
{\em infix} & \verb,!= (Integer -x, -y)     : bool, \\
{\em infix} & \verb,!= (bool -x, -y)        : bool, \\
\end{tabular}

\verb!==! and \verb,!=, return \verb!true! if the operands are
equal, \verb!false! if they are not.  They can be applied to two
integers or to two booleans.

\subsection{Logical and bitwise operators}

\begin{tabular}{ll}
  {\em prefix} & \verb!~ (x: Integer)     : int! \\
  {\em prefix} & \verb!~ (x: bool)        : bool! \\
\end{tabular}

\verb!~! represents logical inversion.  If \verb!x! is an integer,
\verb!~x! is the bitwise inversion of \verb!x!, interpreted as a
two's-complement number.  If \verb!x! is a boolean, and the value of
\verb!x! is \verb!true!, then the value of \verb!~x! is \verb!false!
and if the value of \verb!x! is \verb!false! then the value of
\verb!~x! is \verb!true!.

\begin{tabular}{ll}
{\em infix} & \verb!& (Integer -x, -y)     : int! \\
{\em infix} & \verb!& (bool -x, -y)        : bool! \\
{\em infix} & \verb!| (Integer -x, -y)     : int! \\
{\em infix} & \verb!| (bool -x, -y)        : bool! \\
{\em infix} & \verb!^ (Integer -x, -y)     : int! \\
{\em infix} & \verb!^ (bool -x, -y)        : bool! \\
\end{tabular}

For boolean arguments \verb!x! and \verb!y!, \verb!x & y! is \verb!true! if
both arguments are \verb!true!, otherwise it is \verb!false!.
For integer arguments \verb!x! and \verb!y!, \verb!x & y! is the bitwise AND
of \verb!x! and \verb!y! interpreted as two's-complement numbers.

For boolean arguments \verb!x! and \verb!y!, \verb!x | y! is \verb!false! if
both arguments are \verb!false!, otherwise it is \verb!true!.  
For integer arguments \verb!x! and \verb!y!, \verb!x | y! is the bitwise OR
of \verb!x! and \verb!y! interpreted as two's-complement numbers.

For boolean arguments \verb!x! and \verb!y!, \verb!x ^ y! is
equivalent to \verb,x != y,. For integer arguments \verb!x! and
\verb!y!, \verb!x ^ y! is the bitwise XOR of \verb!x! and \verb!y!
interpreted as two's-complement numbers.

\subsection{Logical short-circuit operators}\label{sec:shortcircuit}
\begin{tabular}{ll}
{\em infix} & \verb!&& (bool -x, -y)        : bool! \\
{\em infix} & \verb!&& (Integer -x, -y)     : bool! \\
{\em infix} & \verb!|| (bool -x, -y)        : bool! \\
{\em infix} & \verb!|| (Integer -x, -y)     : bool! \\
\end{tabular}

For boolean \verb!x! and \verb!y!, \verb!x && y! is \verb!false! if
either \verb!x! or \verb!y! is \verb!false!, and \verb!true!
otherwise.  If \verb!x! is \verb!false!, \verb!y! is not evaluated.
For integer \verb!x! and \verb!y!, \verb!x && y! is equivalent to
\verb,(x != 0) && (y != 0),; this means that if \verb!x! is zero, then
\verb!y! is not evaluated. 

For boolean \verb!x! and \verb!y!, \verb!x || y! is \verb!true! if
either \verb!x! or \verb!y! is \verb!true!, and \verb!false!
otherwise.  If \verb!x! is \verb!true!, \verb!y! is not evaluated.
For integer \verb!x! and \verb!y!, \verb!x || y! is equivalent to
\verb,(x != 0) || (y != 0),; this means that if \verb!x! is nonzero, then
\verb!y! is not evaluated.

\subsection{Loop expressions}\label{sec:loop-expression}
A loop expression is
\begin{Verbatim}[samepage=true]
   < Op Id : Range : Expr >
\end{Verbatim}
where
\begin{itemize}

\item \verb!Op! is \verb!+!, \verb!*!, \verb!&!, \verb!|!, \verb!^!

\item  \verb!Id! is an identifier,
  
\item \verb!Range! is an integer range \verb!Lo .. Hi! or
  an integer value \verb!N!, which is interpreted identically to \verb!0 .. N - 1!,
\item \verb!Expr! is an expression
\end{itemize}
The loop expression is interpreted as \verb!Hi - Lo + 1! textual repetitions of
 \verb!(Expr)! with the name \verb!Id! bound to the values
\verb!Lo! to \verb!Hi! in succession, with \verb!Op! as a separator.
This means that within \verb!Expr!, \verb!Id! takes on the constant
\verb!int! value of the index of that iteration of the loop, and \verb!Id! is
a readonly designator.
The empty range is allowed and if present, the loop expression is
equal to the zero element of \verb!Op! as it applies to the type of \verb!Expr!, according to the following table

\begin{tabular}{llr}
\verb!+!  & \verb!int! & 0 \\
\verb!*!  & \verb!int! & 1 \\
\verb!&! & \verb!int! & -1 \\
\verb!&! & \verb!bool! & \verb!true! \\
\verb!|! & \verb!int! & 0 \\
\verb!|! & \verb!bool! & \verb!false! \\
\verb!^! & \verb!int! & 0 \\
\verb!^! & \verb!bool! & \verb!false! \\
\end{tabular}

\subsection{Probe}\label{sec:probe}

If \verb!X! is an input port or an output port, the expression
\begin{verbatim}
   #X
\end{verbatim}
is called the probe of \verb!X!.  \verb!#X! is true if \verb!X! is an
input port and the receive statement \verb!X?x! would execute without
blocking, or if \verb!X! is an output port and the send statement
\verb,X!x, would execute without blocking.  

\subsection{Receive expression}\label{sec:receive-expression}

The {\em receive expression\/} is
\begin{Verbatim}[samepage=true]
   IPort ?
\end{Verbatim}
where \verb!IPort! is an input port of width \verb!W!.  If no value
has yet been sent by the sender, the receiver is suspended until the
sender has sent the value, at which time the receive expression
evaluates by performing a receive as in section~\ref{sec:receive} into
an anonymous location, and returning that as a value.

For example, the assignment of the receive expression \verb!IPort?! to
\verb!v!:
\begin{verbatim}
   v = IPort? 
\end{verbatim}
is equivalent to the receive statement:
\begin{verbatim}
   IPort?v 
\end{verbatim}
The receive expression can be used in complex expressions without
introducing variable names for the values received, for example to
receive two numbers on \verb!A! and \verb!B! and send the product on
\verb!C!:
\begin{verbatim}
   C!(A? * B?)
\end{verbatim}

\subsection{Peek}\label{sec:peek}

The peek is
\begin{Verbatim}[samepage=true]
   # IPort ?
\end{Verbatim}
where \verb!IPort! is an input port of width \verb!W!.  If no value
has yet been sent by the sender, the receiver is suspended until the
sender has sent the value, at which time the peek evaluates to the
value that would have been written by executing a receive on the
channel associated with \verb!IPort!.  $cR$ is not incremented for the
channel.  The peek is thus a receive expression that does not
increment $cR$, and allows the same value to be inspected more than
once without using a variable to store that value between uses.  See
sections~\ref{sec:receive} and~\ref{sec:receive-expression}.

\section{Special functions}

We use the format of section~\ref{op-conventions}.

\subsection{Printing}

\begin{tabular}{l}
  \verb!print  (string  -x)! \\
  \verb!print  (Integer -x)! \\
  \verb!print  (bool    -x)! \\
\end{tabular}

\verb!print! is a proper procedure that can take a parameter of several
types.  If \verb!print(x)! is activated and \verb!x! is a \verb!string!, a
representation of that string will be printed to the terminal
using the normal localization rules for character sets, etc.

If \verb!print(x)! is activated and \verb!x! is an integer of any type, 
an ASCII representation of the absolute value of \verb!x! in hexadecimal digits, drawn
from the set \verb!0..9,a..f! will be printed to the terminal, preceded by the prefix \verb!0x! if \verb!x! is positive and preceded by the prefix \verb!-0x! if \verb!x! is negative.


If \verb!print(x)! is activated and \verb!x! is \verb!false!, \verb!0x0! will
be printed to the terminal, and if \verb!x! is \verb!true!, \verb!-0x1! will
be printed to the terminal.

\subsection{Packing and unpacking of structs}

\subsubsection{Packing structs}

\begin{tabular}{l}
  \verb!pack(Structure -x) : int! \\
\end{tabular}

\verb!pack! packs the value of the fields of an instance of any structure
type into an integer.  Structures are packed into higher-order bits first,
lower-order bits later.  Arrays within structures are packed into lower-order
bits first and higher-order bits later.  Multi-dimensional arrays are
packed with the earlier indexes changing slowest.

\subsubsection{Examples}

Consider
\begin{Verbatim}[samepage=true]
    structure s = (
      int(8) a[4];
      int(8) b;
    );

    s ss;

    ss::a[0] = 0;
    ss::a[1] = 1;
    ss::a[2] = 2;
    ss::a[3] = 3;

    ss::b    = 0xbb;

    sss = pack(ss)  
\end{Verbatim}
\verb!sss! will contain \verb!0x03020100bb! because \verb!a! gets packed to
higher-order bits, \verb!b! gets packed to lower-order bits, and within
\verb!b! the higher indexes are packed to higher-order bits.

Now consider
\begin{Verbatim}[samepage=true]
    structure t = (
      int(8) a[2][2];
    );

    t tt;

    tt::a[0][0] = 0x00;
    tt::a[0][1] = 0x01;
    tt::a[1][0] = 0x10;
    tt::a[1][1] = 0x11;

    ttt = pack(tt)
\end{Verbatim}
\verb!ttt! will contain \verb!0x11100100! because \verb!a[1]! is packed
to higher-order-bits and within \verb!a[1]!, \verb!a[1][1]! is packed
higher than \verb!a[1][0]!.

\subsubsection{Unpack}
\begin{tabular}{l}
  \verb!unpack(Structure +x, Integer -y)! \\
\end{tabular}

If \verb!x! and \verb!t! are of the same structure type, and
\verb!y! is equal to \verb!pack(t)! then
\begin{Verbatim}[samepage=true]
   unpack(x, y)
\end{Verbatim}
is equivalent to the statement
\begin{Verbatim}[samepage=true]
   x = t
\end{Verbatim}

\subsection{Assertions}

\begin{tabular}{l}
  \verb!assert  (bool -x)! \\
  \verb!assert  (bool -x, string -y)! \\
\end{tabular}

For both forms of \verb!assert!, activation of \verb!assert(x)! or
\verb!assert(x, y)! when \verb!x! is \verb!true! is equivalent to
\verb!skip!.  When \verb!x! is \verb!false!, activation of either
version of the statement is a checked runtime error, the handling of
which is determined by the implementation.  The implementation may use
the argument \verb!y! to determine how to handle the error; \verb!y!
may be used to generate an error message or for any other purpose.
If \verb!x! is \verb!true!, the implementation is permitted, but not
required, to not evaluate \verb!y!.

\subsection{String from integer}
\begin{tabular}{l}
  \verb!string  (Integer -x, -y) : string! \\
\end{tabular}

If \verb!x! is any integer and \verb!y! is in the range \verb!2..36!,
\verb!string(x, y)! converts the number \verb!x! to an ASCII string
representation in the base \verb!y!, where the digits are drawn from the
set \verb!0..9,a..z!.

\subsection{Random numbers}
\begin{tabular}{l}
  \verb!random  (Integer -x) : int(x)! \\
\end{tabular}

If \verb!x! is a nonnegative integer, \verb!random(x)! evaluates to
a number whose \verb!x! least-significant bits are randomly chosen to
be \verb!0! or \verb!1!.  \verb!random(0)! evaluates to \verb!0!.
Activating \verb!random(x)! with \verb!x! a negative number is a checked
runtime error.

\subsection{Timing functions}
\begin{tabular}{l}
  \verb!simtime   () : int! \\
  \verb!walltime  () : int! \\
\end{tabular}

\verb!simtime! and \verb!walltime! are timing functions.
\verb!simtime()! returns an integer that increases at some rate with
respect to the execution of the system of processes.  Repeated calls
to \verb!simtime! are not required to be strictly increasing but are
guaranteed to be nondecreasing as long as the calls are occuring within
the same CSP process.  An implementation is permitted to violate this
requirement if the calls to \verb!simtime! occur in different CSP processes.

\verb!walltime()! returns an integer that increases at a nominal rate
of one billion per second of real time, with some unspecified
granularity.  Its offset to TAI (Temps Atomique International,
International Atomic Time) is unspecified but should be constant at
least across a single execution of a system of CSP processes.

The implementation will provide a function 

\begin{tabular}{l}
  \verb!time   () : int! \\
\end{tabular}

that is related to \verb!simtime! and \verb!walltime! in some
implementation-dependent way.

\subsection{Wait}
\begin{tabular}{l}
  \verb!wait (Integer -x)! \\
\end{tabular}

A process's calling \verb!wait! pauses the execution of that process
for a given amount of simulation time.




\subsection{File input}

An implementation will provide a rudimentary interface for file input.

\begin{tabular}{l}
  \verb!readHexInts   (string -path; Integer -n, +data[]) : int! \\
\end{tabular}

\verb!readHexInts(path,n,data)! reads at most \verb!n! integers from a
file in the local filesystem that can be reached through the path
\verb!path!, using local system conventions for naming, filesystem
links, access rights, etc.  These integers are expected to be provided
in ASCII hexadecimal notation (without a \verb!0x! or any other
prefix) within this file.  \verb!//! line comments are respected
within this file: if the strong \verb!//! appears in the file, any
text from, and including, \verb!//! to the end of that line is
ignored.

It is a checked runtime error for the referenced file to contain anything
other than ASCII hexadecimal data and proper line comments.

\subsection{Miscellaneous required functions}

The implementation will provide the following required functions.

\begin{tabular}{l}
  \verb!min   (Integer -x, -y) : int! \\
  \verb!max   (Integer -x, -y) : int! \\
\end{tabular}

\verb!min(x,y)! returns the value of the less positive of the two arguments \verb!x! and \verb!y!; \verb!max(x,y)! returns the value of the more positive of the two arguments \verb!x! and \verb!y!.

\begin{tabular}{l}
  \verb!b2i   (bool -x) : int(1)! \\
\end{tabular}

\verb!b2i(x)! returns 1 if \verb!x! is equal to \verb!true! and 0 if \verb!x! is equal to \verb!false!.

\begin{tabular}{l}
  \verb!abs   (Integer -x) : int! \\
\end{tabular}

\verb!abs(x)! returns the value of \verb!x! if \verb!x! is greater than or
equal to zero; \verb!-x! otherwise.

\begin{tabular}{l}
  \verb!choose   (bool -x; Integer -y, -z) : int! \\
\end{tabular}

\verb!choose(x, y, z)! returns \verb!y! if \verb!x! is equal to
\verb!true! and \verb!z! otherwise; \verb!choose! adheres to the
normal function-call rules; in particular, all three actuals are
always evaluated.

\begin{tabular}{l}
  \verb!log2   (Integer -x) : int! \\
  \verb!log4   (Integer -x) : int! \\
\end{tabular}

\verb!log2(x)! returns the number of binary digits required to represent
any number in the range \verb!0..abs(x - 1)!.  If \verb!x! is positive,
this is equal to the mathematical expression $\lceil \log_2 \verb!x! \rceil$;
if \verb!x! is negative, it is equal to $\lceil \log_2 (\verb!-x - 1!) \rceil$.

\verb!log4(x)! returns the number of quaternary digits required to represent
any number in the range \verb!0..abs(x - 1)!.  If \verb!x! is positive,
this is equal to the mathematical expression $\lceil \log_4 \verb!x! \rceil$;
if \verb!x! is negative, it is equal to $\lceil \log_4 (\verb!-x - 1!) \rceil$.


\subsection{Reset function}\label{sec:resetport}

The proper procedure \verb!resetNodes! may be declared within a CSP program.
If it is present, \verb!resetNodes()! will be called before the CSP program
begins execution.  Writing initialization within \verb!resetNodes!
allows the implementation to determine that nodes have been reset
properly (see section~\ref{sec:nodeport}).

\begin{tabular}{l}
  \verb!resetNodes ()! \\
\end{tabular}






\begin{thebibliography}{99}

\bibitem{spwm3}{G.~Nelson, ed.  {\it Systems Programming with
    Modula-3.}  Prentice-Hall Series in Innovative Technology.
  Prentice-Hall, 1991.}

\bibitem{Martin1981}{
  A. J. Martin.  An axiomatic definition of synchronization primitives.
  Acta Informatica {\bf 16}, 1981, pp.~219--235.}

\bibitem{Martin1985}{
  A. J. Martin.  The probe: An addition to communication primitives.
  Information Processing Letters {\bf 20}, 1985, pp.~125--130.}

\bibitem{Martin1986}{
A. J. Martin.  Compiling communicating processes into
delay-insensitive VLSI circuits.  Distributed Computing 1, 1986.}

\bibitem{Martin1991}{
A. J. Martin.  Synthesis of Asynchronous VLSI Circuits.  Caltech Computer Science Technical Report CS-TR-93-28.  California Institute of Technology, 1991.}

\bibitem{Dijkstra1975a}{
E. W. Dijkstra.  Guarded Commands, Nondeterminacy, and Formal
Derivation of Programs.  1975.}

\bibitem{Dijkstra1975b}{
E. W. Dijkstra.  A Discipline of Programming.  1975.}

\bibitem{Hoare1974}{
C. A. R. Hoare.  Monitors : An operating system structuring concept.
CACM 17(10), 1974.}

\bibitem{Hoare1978}{
C. A. R. Hoare.  Communicating sequential processes.  CACM 21(8), 1978.}

\bibitem{Pike2009}{
  R. Pike et al.  The Go Programming Language.  2009.}

\bibitem{Kahn}{G.~Kahn.  The semantics of a simple language for parallel processing.  Information Processing {\bf 74}.  North-Holland, 1974.}

\bibitem{Mutex}{E.~W.~Dijkstra.  Solution of a problem in concurrent programming control.  {\it CACM,} {\bf 8}(9), p.~559, 1965.}

  \bibitem{euclid}{Euclid. (\textgreek{Στοιχεῖα}); {\it The Thirteen Books of the Elements,}
  vol.~2, Book~VII.  Translated by Sir~Thomas Heath.  Second edition,
  unabridged.  Dover Publications, 1956.  Reprint of edition by
  Cambridge University Press, 1908.}
  
\
\end{thebibliography}

\chapter{Example Programs}
\label{app:examples}

We look at some concrete example programs.  As CSP is not a complete
language once ports, channels, and nodes are involved, the example
programs will be in a notation we can call CAST-CSP; in other words,
we will look at CSP programs enclosed inside the CAST language (see
section~\ref{sec:cast}).

\section{Single-process programs}

\subsection{The simplest program}
\begin{Verbatim}[samepage=true]
define SKIP()()
{
  csp {
    skip
  }
}
\end{Verbatim}


\subsection{Hello, world}

Obligatory ``hello, world'' program.

\begin{Verbatim}[samepage=true]
define HELLOWORLD()()
{
  csp {
    print("hello, world!")
  }
}
\end{Verbatim}

\newpage
\subsection{The Algorithm}

Euclid's algorithm.\cite{euclid}

\begin{Verbatim}[samepage=true]
define EUCLID(int A; int B)()
{
  csp {
    x = A;
    y = B;

    *[  x > y -> x = x - y
     [] x < y -> y = y - x
     ];

    print ("gcd(" + A + "," + B + ") = " + x)
  }
}
\end{Verbatim}


\newpage
\section{Multi-process programs}

\subsubsection{Simple incrementer}
A first process \verb!P0! sending an ever-increasing integer
over a channel of width \verb!WIDTH! bits to a second process \verb!P!, which
prints them to the terminal.


\begin{Verbatim}[samepage=true]
define P0(int WIDTH)(bd(WIDTH) +R)
{

  csp {
    int x;
    *[ R!x; x++ ]
  }
}

define P1(int WIDTH)(bd(WIDTH) -L)
{
  csp {
    *[ L?x ; print(x) ]
  }
}

define P0P1(int WIDTH)()
{
  subcells {
    bd(WIDTH) c;
    P0(WIDTH) p0(c);
    P1(WIDTH) p1(c);
  }
}
\end{Verbatim}

Here \verb!bd(WIDTH)! represents a bundled-data channel of width
\verb!WIDTH! bits, one of the many possible hardware realizations of a CSP
\verb!WIDTH!-width channel.

\subsection{A bigger example: the Collatz conjecture}

\subsubsection{Common definitions}
\begin{Verbatim}[]
int W        = 61; // 2^62 < (2 ^ 61 - 1) * 3 + 1  < 2^63
                   // we need an extra bit for the sign bit

int STATUSW  = 128 + 2 * W + (W + 1); 

define common attributes {
  csp {
    
    structure longest_so_far =
      (
       int(W) num;
       int(W) length;
       );

    structure status =
      (
       bool     tok;
       sint(W)  id;
       int(128) ops;
       longest_so_far l;
       );

    function fmtStatus(status -s) : string =
      (
       fmtStatus =
       "{id=" + s::id +
       " tok=" + s::tok +
       " len=" + s::ops +
       " num=" + s::l::num  +
       " length=" + s::l::length +
       "}"
      );
    
    function sendStatus(bool -tok; int -id, -len, -num, -steps) =
      (
       status s;
       s::tok       = tok;
       s::id        = id;
       s::l::length = len;
       s::l::num    = num;
       s::ops       = steps;

       int p = pack(s);

       status t;
       unpack(t, p);

       STATUS!(p);
       steps = 0;
       );

    function even(int -n) : bool = ( even = ((n & 1) == 0) );

    function odd(int -n) : bool =  ( odd = ~even(n) );

  }
} 
\end{Verbatim}

\subsubsection{Collatz worker}
\begin{Verbatim}[]
define WORKER (int NWORKERS) (bd(W) -STRT; bd(STATUSW) +STATUS) <+ common
{
  csp {

    bool assert_inrange=false;
    
    int(W-log2(NWORKERS)) i;
    int(W) steps  = 0;
    int(W) maxlen = 0;

    int(W/2) id;
    
    STRT?id;
    print ("id = " + id);
    
    *[int(W) idx = i * NWORKERS + id; // this is the index of the odd number
      int(W) num = 2 * idx + 1;       // this is the actual number we shall work on
      int(W) len = 0;
      int    snum = num;
      int    newval;
      
      // do the Collatz loop
      *[ num != 1 ->
         [  odd(num)  -> newval = 3 * num + 1
         [] even(num) -> newval = num / 2
         ];
         num = newval;

         #[ assert_inrange -> assert(num == newval, "overflow!") ];
         
         len++
       ]
      ;
      assert(num == 1);
      steps += len;
      
     #[  len > maxlen ->
         maxlen = len;
         sendStatus(false, id, len, snum, steps);
         steps = 0
      ];
      
     #[ (i % 1000 == 0)    ->
        sendStatus(true, id, 0, 0, steps);
        steps = 0;

        int dummy;
        STRT?dummy;
      ];
      i++
    ]
    }
}
\end{Verbatim}

\subsubsection{Collatz manager}
\begin{Verbatim}[]
int Billion = 1000**3;
int printstep = 4;

define MANAGER(int NWORKERS)(bd(W) +STRT; bd(STATUSW) -STATUS) <+ common
{
  csp {
      
    int ops;

    int(W) bigiter=0;
    int(log2(NWORKERS)) outstanding;
    STRT!0; outstanding = NWORKERS; bigiter++;
    
    status save;

    int start = walltime();
    int last  = start; // last time for a print
    
    *[ STATUS?x;
          
       status s;
       bool doprint = false;

       unpack(s,x);

       #[ s::tok -> // a return token from the STRT
         outstanding--;
       //         print("Token!  outstanding=" + outstanding);
         #[ outstanding == 0 ->
            // restart workers once all tokens have been collected
       print("phase " + bigiter + " complete, re-launching...");
            STRT!bigiter ; outstanding=NWORKERS ; bigiter++
          ];
       ];

       #[ s::l::num != save::l::num -> save = s ; doprint = true ];
       
       //              print ("STATUS? " + fmtStatus(s) + " ( ops = " + s::ops +
       //            " ; num = " + save::l::num +
       //            " ; chain = " + save::l::length + " )");

       ops += s::ops;

       int now       = walltime();
       int runtime   = now - start;
       int runtime_s = runtime / Billion + 1;

       doprint |= (now > (lastprint + printstep * Billion));
       
       #[ doprint ->

          print("num = " + save::l::num + " ; chain = " + save::l::length + 
          " : total steps = " + ops + " wall = " + runtime_s +
          " ksteps/s = " + (ops/runtime_s/1000));

          lastprint = now
       ]
       
     ]
  }
}
\end{Verbatim}

\subsubsection{Start tree}
\begin{Verbatim}[samepage=true]
define STARTSPLIT(int level)(bd(W) -L, +R[0..1]) <+ common
{
  csp {
    *[ L?i; R[0]!i, R[1]!(i + 2**level) ]
  }
}

define STARTBUFF()(bd(W) -L, +R) <+ common
{
  subcells { L = R; }  // this makes the workers start a bit out of phase
}

define START_TREE(int level, N)(bd(W) -L, +R[0..N-1]) <+ common
// level is the level of the output of the tree
{
  [ N==1 -> subcells { STARTBUFF b(L, R[0]); } ]
      [ N==2 -> subcells { STARTSPLIT(level) s(L, R); } ]
    [ N > 2 ->
      subcells {
        bd(W) X[0..(N+1)/2 - 1];
        <i : 0.. N / 2 - 1 : STARTSPLIT(level)
           s[i](X[i], {R[2*i], R[2*i + 1]}); >
           
        START_TREE(level + 1, (N+1)/2) tree(L, X);

        <i: N/2 .. (N+1)/2 - 1 : STARTBUFF b[i](X[i],R[2*i]);>
           }
    ]
}
\end{Verbatim}

\subsubsection{Status tree}
\begin{Verbatim}[]
define merge_attr attributes <+ common
{
  csp {
    function recv(int -which; status +-s) =
      (
       int oldops = s::ops;
       int oldlen = s::l::length;
       
       L[which]?xs;
       unpack(s, xs);
       s::l::length  = max(oldlen, s::l::length);
       s::ops       += oldops;
       );

    function checksend(status +-os, +-s) =
      (
       #[ (os::l::length != s::l::length) || s::tok ->
             R!s; s::ops = 0; os = s;
        ]
       );
  }
}

define SMERGE()(bd(STATUSW) -L[0..1], +R) <+ merge_attr
{
  csp {
    status s;
    
    *[
      status os = s;
      [
       #L[0] -> recv(0, s)
       :
       #L[1] -> recv(1, s)
       ];
       checksend(os, s);
       ]
  }
}

define STATUS_TREE(int N)(bd(STATUSW) -L[0..N-1], +R)  <+ common
{
  [ N == 1 -> subcells { L[0] = R; } ]
  [ N == 2 -> subcells { SMERGE m(L, R); } ]
  [ N > 2  ->
      subcells {
        bd(STATUSW) X[0..(N+1)/2 - 1];
        <i : 0.. N / 2 - 1 : SMERGE m[i]({L[2*i], L[2*i + 1]}, X[i]); >
        STATUS_TREE((N+1)/2) tree(X, R);

        <i: N/2 .. (N+1)/2 - 1 : L[2*i] = X[i];>
    }
  ]
}
\end{Verbatim}

\subsubsection{Putting it all together}
\begin{Verbatim}[samepage=true]
define COLLATZ(int NWORKERS_ARG)()  <+ common
{
  int NWORKERS;
  [ NWORKERS_ARG > 0 -> NWORKERS = NWORKERS_ARG; ]
  [ NWORKERS_ARG < 0 -> NWORKERS = -NWORKERS_ARG; ]
    
  subcells {
    bd(W) S, STRT[0..NWORKERS-1];
    bd(STATUSW) WSTATUS[0..NWORKERS-1], STATUS;
    MANAGER(NWORKERS) mgr(S, STATUS);
    START_TREE(0, NWORKERS) start_tree(S, STRT);

    [ NWORKERS_ARG > 0 ->
      <i:0..NWORKERS-1: WORKER(NWORKERS) wrkr[i](STRT[i], WSTATUS[i]);>]

    STATUS_TREE(NWORKERS) status_tree(WSTATUS, STATUS);
  }
}
\end{Verbatim}

\newpage
\subsection{Four-phase handshaking in nodes}
Normally, the four-phase protocol would be implemented inside the channel
types, so the processes could use the rendezvous.  Here, we do it explicitly,
so as to exercise the node type.

\begin{verbatim}
define S()(node -ri, +ro)
{
  csp {
    function resetNodes() = (ro-);

    *[ [ri]; ro+; [~ri]; ro- ]
  }
}

define T()(node +-li, +-lo)
{
  csp {
    function resetNodes() = (lo-);
    *[ lo+; [li]; print(li); lo-; [~li]; print(li) ]
  }
}

define FOURPHASE()()
{
  subcells {
    node d, a;

    S() s(a, d);
    T() t(d, a);
  }
}
\end{verbatim}


\newpage
\subsection{Dijkstra's solution to the mutual exclusion problem}
Dijkstra's generalization of Dekker's algorithm for mutual exclusion\cite{Mutex}.
Applied to two processes since assignments to node arrays in CSP are
non-atomic.

\begin{verbatim}
int N = 2;
define COMPUTER(int i)(node b[0..N-1], c[0..N-1], k, busy)
{
  csp {
    function resetNodes() = (b[i]+, c[i]+, k+, busy-);
    
    bool Li0 = true;
    bool Li1 = false;
    
    *[ Li0 -> b[i] = false;
       Li0-, Li1+

     [] Li1 ->
       wait(random(4));
       print ("Li1 k=" + k);
       
      [ b2i(k) != i ->
        c[i] = true;
        #[ b[b2i(k)] -> k = i && i ]
      [] else ->
         c[i] = false;
         bool fail = false;
        <;j:N: #[ j != i & ~c[j] -> fail = true]> ;

        #[ ~fail ->
           print ("enter critical section " + i);
           assert(~busy);
           busy+;
           wait(random(4));
           busy-;
           print ("leave critical section " + i);
           c[i] = true; b[i] = true;
           print ("enter non-critical section " + i);
           wait(random(4));
           print ("leave non-critical section " + i);
           Li0+, Li1- 
         ]
        ]
      ]
   }
}
\end{verbatim}
\newpage
\begin{verbatim}
define DIJKSTRA()()
{
  subcells {
    node b[0..N-1], c[0..N-1], k, busy;

    COMPUTER(0) c0(b, c, k, busy);
    COMPUTER(1) c1(b, c, k, busy);
  }
}
\end{verbatim}


\appendix

\chapter{Syntax}

\section{CSP Lexer}
% Copyright (c) 2025 Intel Corporation.  All rights reserved.  See the file COPYRIGHT for more information.
% SPDX-License-Identifier: Apache-2.0

\begin{longtable}[]{@{}l@{}}
\toprule
\endhead
\begin{minipage}[t]{0.97\columnwidth}\raggedright
Grammar CspLexer\\
\href{http://www.ANTLR.org}{ANTLR}-generated HTML file from Csp.g

Terence Parr, \href{http://www.magelang.com}{MageLang Institute}\\
ANTLR Version 2.7.5 (20050128); 1989-2005\strut
\end{minipage}\tabularnewline
\bottomrule
\end{longtable}

\begin{verbatim}
Definition of lexer CspLexer, which is a subclass of CharScanner.

mWS 
|   mIDENT 
|   mINTEGER 
|   mLPAREN 
|   mRPAREN 
|   mLBRACK 
|   mRBRACK 
|   mLCURLY 
|   mRCURLY 
|   mBOX 
|   mLOOP 
|   mPLOOP 
|   mMLOOP 
|   mALOOP 
|   mOLOOP 
|   mSLOOP 
|   mBLOOP 
|   mCLOOP 
|   mXLOOP 
|   mBOXLOOP 
|   mCLLOOP 
|   mLANGL 
|   mRANGL 
|   mLSHIFT 
|   mLEQ 
|   mGEQ 
|   mARROW 
|   mEQUAL 
|   mNEQ 
|   mASSIGN 
|   mPASSIGN 
|   mMASSIGN 
|   mTASSIGN 
|   mDASSIGN 
|   mRASSIGN 
|   mAASSIGN 
|   mOASSIGN 
|   mLSASSIGN 
|   mXASSIGN 
|   mINC 
|   mDEC 
|   mTILDE 
|   mSTAR 
|   mSLASH 
|   mCENT 
|   mPLUS 
|   mMINUS 
|   mEXP 
|   mDOT 
|   mCOMMA 
|   mSEMI 
|   mCOLON 
|   mCOLON2 
|   mDOTDOT 
|   mQUEST 
|   mBANG 
|   mPIPE 
|   mPIPE2 
|   mAMP 
|   mAMP2 
|   mCARET 
|   mHASH 
|   mAT 
|   mCOMMENT_1 
|   mCOMMENT_2 
|   mSTRING_LITERAL 
mWS
    :   (   ' ' 
        |   '\t' 
        |   '\n' 
        |   '\r' 
        ) 
        
    ;

protected mALPHA
    :   (   (               'a'..'z' ) 
        |   (               'A'..'Z' ) 
        |   '_' 
        ) 
    ;

protected mDIGIT
    :   (           '0'..'9' ) 
    ;

protected mHEX_DIGIT
    :   mDIGIT 
    |           'a'..'f' 
    |           'A'..'F' 
    ;

protected mBIN_DIGIT
    :   '0' 
    |   '1' 
    ;

mIDENT
    :   mALPHA 
        (   mALPHA 
        |   mDIGIT 
        )* 
    ;

mINTEGER
    :   ( mDIGIT )+ 
        (   '_' 
            (   (                   'a'..'z' ) 
            |   (                   'A'..'Z' ) 
            |   mDIGIT 
            )+ 
        |   
        ) 
    |   "0x" 
        (   '_' 
        |   
        ) 
        ( mHEX_DIGIT )+ ( '_' ( mHEX_DIGIT )+ )* 
    |   "0b" 
        (   '_' 
        |   
        ) 
        ( mBIN_DIGIT )+ ( '_' ( mBIN_DIGIT )+ )* 
    ;

mLPAREN
    :   '(' 
    ;

mRPAREN
    :   ')' 
    ;

mLBRACK
    :   '[' 
    ;

mRBRACK
    :   ']' 
    ;

mLCURLY
    :   '{' 
    ;

mRCURLY
    :   '}' 
    ;

mBOX
    :   "[]" 
    ;

mLOOP
    :   "*[" 
    ;

mPLOOP
    :   "<+" 
    ;

mMLOOP
    :   "<*" 
    ;

mALOOP
    :   "<&" 
    ;

mOLOOP
    :   "<|" 
    ;

mSLOOP
    :   "<;" 
    ;

mBLOOP
    :   "<||" 
    ;

mCLOOP
    :   "<," 
    ;

mXLOOP
    :   "<^" 
    ;

mBOXLOOP
    :   "<[]" 
    ;

mCLLOOP
    :   "<:" 
    ;

mLANGL
    :   '<' 
    ;

mRANGL
    :   '>' 
    ;

mLSHIFT
    :   "<<" 
    ;

mLEQ
    :   "<=" 
    ;

mGEQ
    :   ">=" 
    ;

mARROW
    :   "->" 
    ;

mEQUAL
    :   "==" 
    ;

mNEQ
    :   "!=" 
    ;

mASSIGN
    :   '=' 
    ;

mPASSIGN
    :   "+=" 
    ;

mMASSIGN
    :   "-=" 
    ;

mTASSIGN
    :   "*=" 
    ;

mDASSIGN
    :   "/=" 
    ;

mRASSIGN
    :   "%=" 
    ;

mAASSIGN
    :   "&=" 
    ;

mOASSIGN
    :   "|=" 
    ;

mLSASSIGN
    :   "<<=" 
    ;

mXASSIGN
    :   "^=" 
    ;

mINC
    :   "++" 
    ;

mDEC
    :   "--" 
    ;

mTILDE
    :   '~' 
    ;

mSTAR
    :   '*' 
    ;

mSLASH
    :   '/' 
    ;

mCENT
    :   '%' 
    ;

mPLUS
    :   '+' 
    ;

mMINUS
    :   '-' 
    ;

mEXP
    :   "**" 
    ;

mDOT
    :   '.' 
    ;

mCOMMA
    :   ',' 
    ;

mSEMI
    :   ';' 
    ;

mCOLON
    :   ':' 
    ;

mCOLON2
    :   "::" 
    ;

mDOTDOT
    :   ".." 
    ;

mQUEST
    :   '?' 
    ;

mBANG
    :   '!' 
    ;

mPIPE
    :   '|' 
    ;

mPIPE2
    :   "||" 
    ;

mAMP
    :   '&' 
    ;

mAMP2
    :   "&&" 
    ;

mCARET
    :   "^" 
    ;

mHASH
    :   '#' 
    ;

mAT
    :   '@' 
    ;

mCOMMENT_1
    :   "/*" 
        (   '*' 
        |   '\n' 
        |   (   '*' 
            |   '\n' 
            ) 
        )* 
        "*/" 
    ;

mCOMMENT_2
    :   "//" ( ~'\n' )* '\n' 
    ;

mSTRING_LITERAL
    :   '"' 
        (   mESC 
        |   '\n' 
        |   (   '"' 
            |   '\\' 
            |   '\n' 
            ) 
        )* 
        '"' 
    ;

protected mESC
    :   '\\' 
        (   'n' 
        |   'r' 
        |   't' 
        |   'b' 
        |   'f' 
        |   '"' 
        |   '\'' 
        |   '\\' 
        |   'x' mHEX_DIGIT mHEX_DIGIT 
        |   (               '0'..'3' ) 
            (   (                   '0'..'7' ) 
                (                       '0'..'7' 
                |   
                ) 
            |   
            ) 
            
        ) 
    ;
\end{verbatim}


\section{CSP Parser}
% Copyright (c) 2025 Intel Corporation.  All rights reserved.  See the file COPYRIGHT for more information.
% SPDX-License-Identifier: Apache-2.0

\begin{longtable}[]{@{}l@{}}
\toprule
\endhead
\begin{minipage}[t]{0.97\columnwidth}\raggedright
Grammar CspParser\\
\href{http://www.ANTLR.org}{ANTLR}-generated HTML file from Csp.g

Terence Parr, \href{http://www.magelang.com}{MageLang Institute}\\
ANTLR Version 2.7.5 (20050128); 1989-2005\strut
\end{minipage}\tabularnewline
\bottomrule
\end{longtable}

\begin{verbatim}
Definition of parser CspParser, which is a subclass of LLkParser.


program
    :   (   function_decl 
        |   structure_decl 
        )* 
        (   sequential_statement 
        |   
        ) 
        (   DIRECTIVES LCURLY 
        |   
        ) 
        EOF 
    ;


function_decl
    :   FUNCTION IDENT LPAREN 
        (   declaration_list 
        |   
        ) 
        RPAREN 
        (   COLON type 
        |   
        ) 
        ASSIGN statement SEMI 
    ;


structure_decl
    :   STRUCTURE IDENT ASSIGN LPAREN declaration_list RPAREN SEMI 
    ;


sequential_statement
    :   sequential_part ( SEMI sequential_part )* 
        (   SEMI 
        |   
        ) 
    ;


declaration_list
    :   declaration ( SEMI declaration )* 
        (   SEMI 
        |   
        ) 
    ;


type
    :   (   CONST 
        |   
        ) 
        (   INT 
            (   LPAREN expression RPAREN 
            |   
            ) 
            
        |   SINT LPAREN expression RPAREN 
        |   BOOLEAN 
        |   BOOL 
        |   STRING 
        |   IDENT 
        ) 
    ;


statement
    :   LPAREN sequential_statement RPAREN 
    |   selection_statement 
    |   repetition_statement 
    |   assignment_statement 
    |   incdec_statement 
    |   communication_statement 
    |   bool_assignment_statement 
    |   lvalue 
    |   loop_statement 
    |   ERROR 
    |   SKIP 
    ;


var_statement
    :   declaration 
    ;


declaration
    :   type declarator_list 
    ;


loop_statement
    :   (   LANGL 
        |   SLOOP 
        |   BLOOP 
        |   CLOOP 
        ) 
        IDENT COLON range COLON sequential_statement RANGL 
    ;


range
    :   expression 
        (   DOTDOT expression 
        |   
        ) 
    ;


sequential_part
    :   var_statement 
    |   parallel_statement 
    ;


parallel_statement
    :   statement ( COMMA statement )* 
    ;


selection_statement
    :   LBRACK 
        (   guard_commands 
        |   expression 
        ) 
        RBRACK 
    |   HASH LBRACK det_guard_noelse_commands RBRACK 
    ;


repetition_statement
    :   LOOP 
        (   guard_commands 
        |   sequential_statement 
        ) 
        RBRACK 
    ;


lvalue
    :   (   identifier 
        |   STRING 
        ) 
        (   ( LBRACK expression ( COMMA expression )* RBRACK ) 
        |   LCURLY expression 
            (   COLON expression 
            |   
            ) 
            RCURLY 
        |   LPAREN 
            (   expression ( COMMA expression )* 
            |   
            ) 
            RPAREN 
        |   DOT IDENT 
        |   DOT INTEGER 
        |   COLON2 IDENT 
        )* 
    ;


assignment_statement
    :   lvalue 
        (   ASSIGN 
        |   PASSIGN 
        |   MASSIGN 
        |   TASSIGN 
        |   DASSIGN 
        |   RASSIGN 
        |   AASSIGN 
        |   OASSIGN 
        |   XASSIGN 
        |   LSASSIGN 
        |   ( RANGL GEQ ) 
        ) 
        expression 
    ;


incdec_statement
    :   lvalue 
        (   INC 
        |   DEC 
        ) 
        
    ;


communication_statement
    :   HASH lvalue QUEST lvalue 
    |   lvalue 
        (   QUEST 
            (   lvalue 
            |   
            ) 
            
        |   BANG 
            (   expression 
            |   
            ) 
            
        ) 
    ;


bool_assignment_statement
    :   lvalue 
        (   PLUS 
        |   MINUS 
        ) 
        
    ;


expression
    :   conditional_or_expression 
    ;


guard_commands
    :   non_det_guard_commands 
    |   det_guard_commands 
    ;


det_guard_noelse_commands
    :   det_guard_command ( BOX det_guard_command )* 
    ;


non_det_guard_commands
    :   non_det_guard_simple 
    |   non_det_guard_linked 
    ;


det_guard_commands
    :   det_guard_noelse_commands 
        (   BOX guard_else_command 
        |   
        ) 
    ;


guard_else_command
    :   ELSE ARROW sequential_statement 
    ;


det_guard_command
    :   guard_command_simple 
    |   BOXLOOP IDENT COLON range COLON 
        (   LPAREN det_guard_command ( BOX det_guard_command )* RPAREN 
        |   det_guard_command ( BOX det_guard_command )* 
        ) 
        RANGL 
    ;


non_det_guard_linked
    :   linked_guard_command ( COLON linked_guard_command )* COLON linkage_specifier 
    ;


linked_guard_command
    :   expression linkage_specifier ARROW sequential_statement 
    |   CLLOOP IDENT COLON range COLON 
        (   LPAREN linked_guard_command ( COLON linked_guard_command )* RPAREN 
        |   linked_guard_command ( COLON linked_guard_command )* 
        ) 
        RANGL 
    ;


linkage_specifier
    :   AT LPAREN linkage_terms RPAREN 
    ;


non_det_guard_simple
    :   non_det_guard_command ( COLON non_det_guard_command )* 
    ;


non_det_guard_command
    :   guard_command_simple 
    |   CLLOOP IDENT COLON range COLON 
        (   LPAREN non_det_guard_command ( COLON non_det_guard_command )* RPAREN 
        |   non_det_guard_command ( COLON non_det_guard_command )* 
        ) 
        RANGL 
    ;


guard_command_simple
    :   expression ARROW sequential_statement 
    ;


linkage_terms
    :   linkage_term ( COMMA linkage_term )* 
    ;


linkage_term
    :   (   TILDE 
        |   
        ) 
        IDENT 
        (   DOT IDENT 
        |   DOT INTEGER 
        |   LBRACK expression ( COMMA expression )* RBRACK 
        )* 
        
    |   CLOOP IDENT COLON range COLON 
        (   linkage_term 
        |   LPAREN linkage_term RPAREN 
        ) 
        RANGL 
    ;


conditional_or_expression
    :   conditional_and_expression ( PIPE2 conditional_and_expression )* 
    ;


loop_expression
    :   (   PLOOP 
        |   MLOOP 
        |   ALOOP 
        |   OLOOP 
        |   XLOOP 
        ) 
        IDENT COLON range COLON expression RANGL 
    ;


conditional_and_expression
    :   or_expression ( AMP2 or_expression )* 
    ;


or_expression
    :   xor_expression ( PIPE xor_expression )* 
    ;


xor_expression
    :   and_expression ( CARET and_expression )* 
    ;


and_expression
    :   equal_expression ( AMP equal_expression )* 
    ;


equal_expression
    :   rel_expression 
        (   NEQ rel_expression 
        |   EQUAL rel_expression 
        )* 
    ;


rel_expression
    :   shift_expression 
        (   LEQ shift_expression 
        |   GEQ shift_expression 
        |   LANGL shift_expression 
        |   RANGL shift_expression 
        )* 
    ;


shift_expression
    :   add_expression 
        (   LSHIFT add_expression 
        |   RANGL RANGL add_expression 
        )* 
    ;


add_expression
    :   mult_expression 
        (   PLUS mult_expression 
        |   MINUS mult_expression 
        )* 
    ;


mult_expression
    :   unary_expression 
        (   STAR unary_expression 
        |   SLASH unary_expression 
        |   CENT unary_expression 
        )* 
    ;


unary_expression
    :   TILDE unary_expression 
    |   MINUS unary_expression 
    |   exponential_expression 
    ;


exponential_expression
    :   primary_expression 
        (   EXP unary_expression 
        |   
        ) 
    ;


primary_expression
    :   INTEGER 
    |   TRUE 
    |   FALSE 
    |   STRING_LITERAL 
    |   LPAREN expression RPAREN 
    |   communication_expression 
    |   lvalue 
    |   probe_expression 
    |   loop_expression 
    ;


communication_expression
    :   lvalue QUEST 
    ;


probe_expression
    :   HASH lvalue 
        (   QUEST 
        |   
        ) 
    ;


identifier
    :   IDENT 
    ;


startLvalue
    :   lvalue EOF 
    ;


declarator_list
    :   declarator ( COMMA declarator )* 
    ;


declarator
    :   (   (   PLUS 
                (   MINUS 
                |   
                ) 
            |   MINUS 
                (   PLUS 
                |   
                ) 
            ) 
        |   
        ) 
        identifier 
        (   ( LBRACK range RBRACK )* 
        |   LBRACK range ( COMMA range )* RBRACK 
        |   
        ) 
        (   ASSIGN expression 
        |   
        ) 
    ;


identifier_list
    :   identifier ( COMMA identifier )* 
    ;
\end{verbatim}


\section{CAST Lexer}
\begin{longtable}[]{@{}l@{}}
\toprule
\endhead
\begin{minipage}[t]{0.97\columnwidth}\raggedright
Grammar CastTwoLexer\\
\href{http://www.ANTLR.org}{ANTLR}-generated HTML file from CastTwo.g

Terence Parr, \href{http://www.magelang.com}{MageLang Institute}\\
ANTLR Version 2.7.5 (20050128); 1989-2005\strut
\end{minipage}\tabularnewline
\bottomrule
\end{longtable}

\begin{verbatim}
Definition of lexer CastTwoLexer, which is a subclass of CharScanner.

/** Lexer nextToken rule:
 *  The lexer nextToken rule is synthesized from all of the user-defined
 *  lexer rules.  It logically consists of one big alternative block with
 *  each user-defined rule being an alternative.
 */
mARROW 
|   mEXP 
|   mPLUS 
|   mMINUS 
|   mTIMES 
|   mDIV 
|   mMOD 
|   mAND 
|   mOR 
|   mXOR 
|   mNOT 
|   mCOMMA 
|   mSEMI 
|   mCOLON 
|   mASSIGN 
|   mEQ 
|   mNE 
|   mLT 
|   mLE 
|   mGE 
|   mGT 
|   mLPAREN 
|   mRPAREN 
|   mLCURLY 
|   mRCURLY 
|   mLBRACK 
|   mRBRACK 
|   mDOT 
|   mDOTDOT 
|   mPOUND 
|   mCELEM_ARROW 
|   mCOMB_ARROW 
|   mLT_AND 
|   mLT_OR 
|   mLT_COMMA 
|   mWS 
|   mCOMMENT_1 
|   mCOMMENT_2 
|   mIDENT 
|   mNUM_INT 
|   mDOT_IDENT 
|   mQuotedString 
|   mESCAPED_VERILOG_IDENT 
mARROW
    :   "->" 
    ;

mEXP
    :   "**" 
    ;

mPLUS
    :   '+' 
    ;

mMINUS
    :   '-' 
    ;

mTIMES
    :   '*' 
    ;

mDIV
    :   '/' 
    ;

mMOD
    :   '%' 
    ;

mAND
    :   '&' 
    ;

mOR
    :   '|' 
    ;

mXOR
    :   '^' 
    ;

mNOT
    :   '~' 
    ;

mCOMMA
    :   ',' 
    ;

mSEMI
    :   ';' 
    ;

mCOLON
    :   ':' 
    ;

mASSIGN
    :   '=' 
    ;

mEQ
    :   "==" 
    ;

mNE
    :   "!=" 
    ;

mLT
    :   '<' 
    ;

mLE
    :   "<=" 
    ;

mGE
    :   ">=" 
    ;

mGT
    :   '>' 
    ;

mLPAREN
    :   '(' 
    ;

mRPAREN
    :   ')' 
    ;

mLCURLY
    :   '{' 
    ;

mRCURLY
    :   '}' 
    ;

mLBRACK
    :   '[' 
    ;

mRBRACK
    :   ']' 
    ;

mDOT
    :   '.' 
    ;

mDOTDOT
    :   ".." 
    ;

mPOUND
    :   "#" 
    ;

mCELEM_ARROW
    :   "#>" 
    ;

mCOMB_ARROW
    :   "=>" 
    ;

mLT_AND
    :   "<&" 
    ;

mLT_OR
    :   "<|" 
    ;

mLT_COMMA
    :   "<," 
    ;

mWS
    :   (   ' ' 
        |   '\t' 
        |   '\f' 
        |   ( '\n' ) 
        |   '\r' 
        ) 
        
    ;

mCOMMENT_1
    :   "/*" 
        (   '*' 
        |   '\n' 
        |   (   '*' 
            |   '\n' 
            ) 
        )* 
        "*/" 
    ;

mCOMMENT_2
    :   "//" ( ~'\n' )* '\n' 
    ;

mIDENT
    :   mREAL_IDENT 
    |   mESCAPED_IDENT 
    ;

protected mREAL_IDENT
    :   (   mLETTER 
        |   '_' 
        ) 
        (   mLETTER 
        |   '_' 
        |   mDIGIT 
        )* 
        
    ;

protected mESCAPED_IDENT
    :   '"' 
        (   mLETTER 
        |   mDIGIT 
        |   '_' 
        |   '!' 
        )+ 
        '"' 
    ;

protected mLETTER
    :           'a'..'z' 
    |           'A'..'Z' 
    ;

protected mDIGIT
    :           '0'..'9' 
    ;

mNUM_INT
    :   mDIGIT mLETTER 
        (   mDIGIT 
        |   'A' 
        |   'B' 
        |   'C' 
        |   'D' 
        |   'E' 
        |   'F' 
        |   'a' 
        |   'b' 
        |   'c' 
        |   'd' 
        |   'e' 
        |   'f' 
        )+ 
        
    |   ( mDIGIT )+ 
        (   mIDENT 
        |   
        ) 
        (   '.' ( mDIGIT )+ 
            (   mEXPONENT 
            |   
            ) 
            
        |   
        ) 
    ;

protected mEXPONENT
    :   (   'e' 
        |   'E' 
        ) 
        (   '+' 
        |   '-' 
        |   
        ) 
        ( mDIGIT )+ 
    ;

mDOT_IDENT
    :   '.' 
        (   mIDENT 
        |   mNUM_INT 
        ) 
        
    ;

mQuotedString
    :   '\'' ( ~'\'' )* '\'' 
    ;

mESCAPED_VERILOG_IDENT
    :   '\\' ( (  ' ' 
            |   '\t' 
            |   '\n' 
            |   '\r' 
            ) )+ 
    ;
\end{verbatim}


\section{CAST Parser}
\begin{longtable}[]{@{}l@{}}
\toprule
\endhead
\begin{minipage}[t]{0.97\columnwidth}\raggedright
Grammar CastTwoParser\\
\href{http://www.ANTLR.org}{ANTLR}-generated HTML file from CastTwo.g

Terence Parr, \href{http://www.magelang.com}{MageLang Institute}\\
ANTLR Version 2.7.5 (20050128); 1989-2005\strut
\end{minipage}\tabularnewline
\bottomrule
\end{longtable}

\begin{verbatim}
Definition of parser CastTwoParser, which is a subclass of LLkParser.


goal
    :   compilationUnit EOF 
    ;


compilationUnit
    :   moduleDeclaration ( importDeclaration )* ( fileLevelBodyStatement )* 
    ;


moduleDeclaration
    :   MODULE moduleIdent SEMI 
    |   
    ;


importDeclaration
    :   IMPORT importIdent SEMI 
    ;


fileLevelBodyStatement
    :   typeDeclaration 
    |   channelDeclaration 
    |   aliasDeclaration 
    |   typeBody 
    |   assignmentStatement 
    |   variableDeclarationStatement 
    |   loopStatement 
    |   ifStatement 
    |   prsBlock 
    ;


moduleIdent
    :   IDENT 
        (   moduleMetaParamList 
        |   
        ) 
        ( DOT_IDENT 
            (   moduleMetaParamList 
            |   
            ) )* 
    ;


importIdent
    :   IDENT 
        (   moduleMetaParamList 
        |   
        ) 
        ( DOT_IDENT 
            (   moduleMetaParamList 
            |   
            ) )* 
        (   DOT_IDENT 
        |   DOT TIMES 
        ) 
        
    ;


moduleMetaParamList
    :   LPAREN moduleMetaParams RPAREN 
    ;


moduleMetaParams
    :   moduleMetaParam ( COMMA moduleMetaParam )* 
    ;


moduleMetaParam
    :   (   MINUS 
        |   
        ) 
        integer 
    |   TRUE 
    |   FALSE 
    |   LCURLY moduleMetaParams RCURLY 
    ;


integer
    :   NUM_INT 
    |   NUM_HEX 
    ;


cellTypeIdent
    :   IDENT cellTypeTail 
    ;


cellTypeTail
    :   (   moduleMetaParamList 
        |   
        ) 
        DOT_IDENT cellTypeTail 
    |   
    ;


dottedIdent
    :   IDENT ( DOT_IDENT )+ 
    ;


bodyStatements
    :   (   ( fileLevelBodyStatement )* 
        |   ( internalBodyStatement )* 
        ) 
        
    ;


internalBodyStatement
    :   prsBlock 
    |   cspBlock 
    |   subcellsBlock 
    |   envBlock 
    |   javaBlock 
    |   netlistBlock 
    |   directiveBlock 
    |   typeBody 
    |   topLevelConstantDeclaration 
    |   assignmentStatement 
    |   variableDeclarationStatement 
    |   loopStatement 
    |   ifStatement 
    |   subtypesBlock 
    |   verilogBlock 
    ;


typeDeclaration
    :   typeHeader typeBody 
    ;


channelDeclaration
    :   DEFCHAN IDENT 
        (   metaParameterList 
            (   portParameterList 
            |   
            ) 
        |   
        ) 
        (   inheritanceList 
        |   
        ) 
        (   refinement 
        |   
        ) 
        channelBody 
    ;


aliasDeclaration
    :   DEFALIAS IDENT 
        (   metaParameterList 
        |   
        ) 
        refinement SEMI 
    ;


typeBody
    :   LCURLY bodyStatements RCURLY 
    ;


assignableExpression
    :   IDENT 
        (   DOT_IDENT 
        |   arraySelector 
        )* 
    ;


assignmentStatement
    :   assignment SEMI 
    ;


variableDeclarationStatement
    :   variableDeclaration SEMI 
    ;


loopStatement
    :   LT IDENT COLON range COLON bodyStatements GT 
    ;


ifStatement
    :   LBRACK expression ARROW bodyStatements RBRACK 
    ;


prsBlock
    :   (   FRAGMENT 
        |   
        ) 
        PRS LCURLY 
    ;


cspBlock
    :   CSP 
        (   cspPorts 
        |   
        ) 
        LCURLY 
    ;


subcellsBlock
    :   (   INLINE 
        |   FRAGMENT 
        |   
        ) 
        SUBCELLS LCURLY 
    ;


envBlock
    :   ENV 
        (   ( LCURLY ( namedEnv )* RCURLY ) 
        |   namedEnv 
        ) 
    ;


javaBlock
    :   JAVA LCURLY 
        (   oldJavaStatements 
        |   newJavaStatements 
        ) 
        RCURLY 
    ;


netlistBlock
    :   NETLIST LCURLY 
    ;


directiveBlock
    :   DIRECTIVE LCURLY 
    ;


topLevelConstantType
    :   INT 
    |   FLOAT 
    |   BOOL 
    ;


topLevelConstantDeclaration
    :   topLevelConstantType IDENT 
        (   arraySelector 
        |   
        ) 
        (   ASSIGN expression 
        |   
        ) 
        SEMI 
    ;


subtypesBlock
    :   SUBTYPES LCURLY 
    ;


verilogBlock
    :   VERILOG 
        (   ( LCURLY ( verilogNamed )* RCURLY ) 
        |   verilogNamed 
        ) 
    ;


metaParameterList
    :   LPAREN parameterList RPAREN 
    ;


portParameterList
    :   LPAREN parameterList RPAREN 
    ;


inheritanceList
    :   inheritance ( inheritance )* 
    ;


refinement
    :   LT COLON cellTypeIdent 
        (   typeParameterList 
        |   
        ) 
        (   COLON cellTypeIdent 
            (   typeParameterList 
            |   
            ) 
        |   
        ) 
    ;


channelBody
    :   LCURLY channelStatements RCURLY 
    ;


channelStatements
    :   ( channelStatement )* 
    ;


channelStatement
    :   aliasBlock 
    |   assertBlock 
    |   directiveBlock 
    ;


aliasBlock
    :   ALIAS LCURLY 
    ;


assertBlock
    :   ASSERT LCURLY 
    ;


typeHeader
    :   DEFINE IDENT 
        (   metaParameterList 
            (   portParameterList 
                (   impliedParameterList 
                |   
                ) 
            |   
            ) 
        |   
        ) 
        inheritanceAndRefinement 
    ;


impliedParameterList
    :   LPAREN parameterList RPAREN 
    ;


inheritanceAndRefinement
    :   ATTRIBUTES 
        (   inheritanceList 
        |   
        ) 
    |   (   inheritanceList 
        |   
        ) 
        (   refinement 
        |   
        ) 
    ;


inheritance
    :   LT PLUS cellTypeIdent 
        (   typeParameterList 
        |   
        ) 
        (   COLON cellTypeIdent 
            (   typeParameterList 
            |   
            ) 
        |   
        ) 
    ;


expressionList
    :   (   
        |   expression ( COMMA expression )* 
            (   COMMA 
            |   
            ) 
            
        ) 
    ;


typeParameterList
    :   LPAREN expressionList RPAREN 
    |   
    ;


parameterList
    :   (   variableDeclaration ( SEMI variableDeclaration )* 
            (   SEMI 
            |   
            ) 
        |   
        ) 
        
    ;


envExtraParameterList
    :   LPAREN parameterList RPAREN 
    ;


variableDeclaration
    :   (   INLINE 
        |   FLATTEN 
        |   
        ) 
        type channelWidth variableDeclarators 
    ;


arraySelector
    :   LBRACK ranges RBRACK 
    ;


expression
    :   conditionalInclusiveOrExpression 
    ;


instantiation
    :   variableDeclaration 
    ;


type
    :   primitiveType 
    |   cellTypeIdent typeParameterList 
    ;


channelWidth
    :   LBRACK expression RBRACK 
    |   
    ;


variableDeclarators
    :   variableDeclarator ( COMMA variableDeclarator )* 
    ;


variableDeclarator
    :   (   (   PLUS 
                (   MINUS 
                |   
                ) 
            |   MINUS 
                (   PLUS 
                |   
                ) 
            ) 
        |   
        ) 
        IDENT 
        (   arraySelector 
        |   
        ) 
        (   COLON IDENT 
        |   
        ) 
        (   COLON assignableExpression 
        |   
        ) 
        (   variableInitializer 
        |   
        ) 
        
    ;


variableInitializer
    :   LPAREN expressionList RPAREN 
        (   LPAREN expressionList RPAREN 
        |   
        ) 
    |   ASSIGN expression 
    ;


ranges
    :   range ( COMMA range )* 
    ;


range
    :   expression 
        (   DOTDOT expression 
        |   
        ) 
        
    ;


startRange
    :   range EOF 
    ;


cellType
    :   cellTypeIdent typeParameterList EOF 
    ;


primitiveType
    :   BOOL 
    |   FLOAT 
    |   INT 
    |   NODE 
    ;


assignment
    :   assignableExpression assignmentRhs 
    ;


assignmentRhs
    :   ( ASSIGN expression )+ 
    ;


startExpressionList
    :   expressionList EOF 
    ;


arrayComprehension
    :   arrayComprehensionItem ( COMMA arrayComprehensionItem )* 
    ;


arrayComprehensionItem
    :   (   expression 
        |   LT_COMMA IDENT COLON range COLON arrayComprehension GT 
        ) 
    ;


startExpression
    :   expression EOF 
    ;


conditionalInclusiveOrExpression
    :   conditionalExclusiveOrExpression ( OR conditionalExclusiveOrExpression )* 
    ;


conditionalExclusiveOrExpression
    :   conditionalAndExpression ( XOR conditionalAndExpression )* 
    ;


conditionalAndExpression
    :   relationalExpression ( AND relationalExpression )* 
    ;


relationalExpression
    :   additiveExpression ( (  EQ 
            |   NE 
            |   LT 
            |   LE 
            |   GT 
            |   GE 
            ) 
            additiveExpression )* 
    ;


additiveExpression
    :   multiplicativeExpression ( (  PLUS 
            |   MINUS 
            ) 
            multiplicativeExpression )* 
    ;


multiplicativeExpression
    :   unaryExpression ( (    TIMES 
            |   DIV 
            |   MOD 
            ) 
            unaryExpression )* 
    ;


additiveExpr
    :   additiveExpression 
    ;


unaryExpression
    :   PLUS unaryExpression 
    |   MINUS unaryExpression 
    |   NOT unaryExpression 
    |   exponentialExpression 
    ;


exponentialExpression
    :   selectionExpression 
        (   EXP exponentialExpression 
        |   
        ) 
    ;


selectionExpression
    :   primaryExpression 
        (   DOT_IDENT 
        |   arraySelector 
        )* 
    ;


primaryExpression
    :   IDENT LPAREN expressionList RPAREN 
    |   IDENT 
    |   integer 
    |   NUM_REAL 
    |   TRUE 
    |   FALSE 
    |   anonymousArray 
    |   QuotedString 
    |   LPAREN conditionalInclusiveOrExpression RPAREN 
    |   primitiveType LPAREN conditionalInclusiveOrExpression RPAREN 
    |   LT 
        (   PLUS 
        |   TIMES 
        |   XOR 
        ) 
        IDENT COLON range COLON additiveExpr GT 
    |   LT_OR IDENT COLON range COLON additiveExpr GT 
    |   LT_AND IDENT COLON range COLON additiveExpr GT 
    ;


anonymousArray
    :   LCURLY arrayComprehension RCURLY 
    ;


namedEnv
    :   IDENT 
        (   metaParameterList 
            (   envExtraParameterList 
            |   
            ) 
        |   
        ) 
        inheritanceAndRefinement 
        (   ( LCURLY envBlockStatements RCURLY ) 
        |   singleEnvBlockStatement 
        ) 
        
    ;


envBlockStatements
    :   ( internalBodyStatement )* 
    ;


singleEnvBlockStatement
    :   envBlockStatement 
    ;


envBlockStatement
    :   prsBlock 
    |   subcellsBlock 
    |   subtypesBlock 
    |   directiveBlock 
    |   cspBlock 
    |   netlistBlock 
    |   javaBlock 
    |   verilogBlock 
    ;


startPrsNodeExpression
    :   prsNodeExpression EOF 
    ;


prsNodeExpression
    :   IDENT 
        (   DOT_IDENT 
        |   arraySelector 
        )* 
    ;


cspPorts
    :   LPAREN cspPort ( COMMA cspPort )* RPAREN 
    ;


cspPort
    :   IDENT 
    ;


oldJavaStatements
    :   ( oldJavaStatement )+ 
    ;


newJavaStatements
    :   ( newJavaStatement )+ 
    ;


newJavaStatement
    :   newJavaClassStatement 
    |   javaChannelStatement 
    |   directiveBlock 
    ;


newJavaClassStatement
    :   (   IDENT 
        |   dottedIdent 
        ) 
        (   javaClassMetaParameters 
        |   
        ) 
        IDENT 
        (   javaClassParameters 
        |   
        ) 
        (   javaClassInitializers 
        |   
        ) 
        SEMI 
    ;


javaChannelStatement
    :   CHANNEL IDENT 
        (   arraySelector 
        |   
        ) 
        (   javaChannelTimingInfo 
        |   
        ) 
        SEMI 
    ;


javaClassMetaParameters
    :   LPAREN javaClassMetaParameter ( COMMA javaClassMetaParameter )* RPAREN 
    ;


javaClassParameters
    :   LPAREN javaClassParameter ( COMMA javaClassParameter )* RPAREN 
    ;


javaClassInitializers
    :   COLON javaClassInitializer ( COMMA javaClassInitializer )* 
    ;


javaClassMetaParameter
    :   expression 
    ;


javaClassParameter
    :   ( selectionExpression ) 
        (   javaChannelTimingInfo 
        |   
        ) 
        
    ;


javaChannelTimingInfo
    :   COLON 
        (   integer 
        |   LPAREN integer COMMA integer COMMA integer COMMA 
                   integer COMMA integer RPAREN 
        ) 
    ;


javaClassInitializer
    :   TRUE 
    |   FALSE 
    |   integer 
    |   NUM_REAL 
    |   QuotedString 
    ;


oldJavaStatement
    :   oldJavaClassStatement 
    |   inputChannelStatement 
    |   outputChannelStatement 
    |   internalChannelStatement 
    ;


oldJavaClassStatement
    :   JAVACLASS QuotedString 
    ;


inputChannelStatement
    :   INPUTCHAN QuotedString 
    ;


outputChannelStatement
    :   OUTPUTCHAN QuotedString 
    ;


internalChannelStatement
    :   INTERNALCHAN QuotedString 
    ;


verilogNamed
    :   IDENT LCURLY verilogStatements RCURLY 
    ;


verilogStatements
    :   ( verilogStatement )* 
    ;


verilogStatement
    :   verilogIdent 
        (   POUND LPAREN expressionList RPAREN 
        |   
        ) 
        LPAREN 
        (   verilogParameters 
        |   
        ) 
        RPAREN 
        (   COLON verilogFiles 
        |   
        ) 
        SEMI 
    |   COLON verilogFiles SEMI 
    |   directiveBlock 
    ;


verilogIdent
    :   IDENT 
    |   dottedIdent 
    ;


verilogParameters
    :   verilogImplicitParameter ( COMMA verilogImplicitParameter )* 
    |   verilogExplicitParameter ( COMMA verilogExplicitParameter )* 
    ;


verilogFiles
    :   QuotedString ( COMMA QuotedString )* 
    ;


verilogImplicitParameter
    :   selectionExpression 
    ;


verilogExplicitParameter
    :   (   DOT_IDENT 
        |   DOT ESCAPED_VERILOG_IDENT 
        ) 
        LPAREN 
        (   selectionExpression 
        |   
        ) 
        RPAREN 
    ;


partialExtraction
    :   cellTypeIdent 
        (   typeParameterList 
        |   
        ) 
        ( ( PLUS 
            |   MINUS 
            ) 
            (   PLUS 
            |   MINUS 
            |   
            ) 
            assignableExpression )* 
        (   COLON 
            (   cellTypeIdent 
                (   typeParameterList 
                |   
                ) 
            |   TIMES 
            ) 
        |   
        ) 
        EOF 
    ;


alintFanin
    :   prsNodeExpression 
        (   ASSIGN NUM_INT 
        |   PLUS 
        |   MINUS 
        ) 
        
    ;


alintScenario
    :   alintScenarioItem ( COMMA alintScenarioItem )* 
    ;


alintScenarioItem
    :   (   alintFanin 
        |   LT_COMMA IDENT COLON range COLON alintScenario GT 
        ) 
    ;


startAlintScenario
    :   LCURLY 
        (   alintScenario 
        |   
        ) 
        RCURLY EOF 
    ;
\end{verbatim}


\section{CAST Tree Parser}
% Copyright (c) 2025 Intel Corporation.  All rights reserved.  See the file COPYRIGHT for more information.
% SPDX-License-Identifier: Apache-2.0

\begin{longtable}[]{@{}l@{}}
\toprule
\endhead
\begin{minipage}[t]{0.97\columnwidth}\raggedright
Grammar CastTwoTreeParser\\
\href{http://www.ANTLR.org}{ANTLR}-generated HTML file from
CastTwoTree.g

Terence Parr, \href{http://www.magelang.com}{MageLang Institute}\\
ANTLR Version 2.7.5 (20050128); 1989-2005\strut
\end{minipage}\tabularnewline
\bottomrule
\end{longtable}

\begin{verbatim}

Definition of tree parser CastTwoTreeParser, which is a subclass of TreeParser.

    
    
goal
        :   compilationUnit EOF 
        ;
    
    
compilationUnit
        :   ( moduleDeclaration ) ( importDeclaration )* ( globalBodyStatement )* 
        ;
    
    
moduleDeclaration
        :                #( m:MODULE id:IDENT ) 
        ;
    
    
importDeclaration
        :                #( IMPORT id:IMPORT_IDENT ) 
        ;
    
    
globalBodyStatement
        :   bodyStatement 
        ;
    
    
bodyStatementList
        :                #( BODY_STATEMENT_LIST 
                              ( bodyStatementenv, globalP, cell, envContainer, 
                                paramEnv, prsEnv,
                                subcellsEnv, subtypesEnv )* ) 
        ;
    
    
bodyStatement
        :   prsBlock 
        |   subcellsBlock 
        |   cspBlock 
        |   envBlock 
        |   javaBlock 
        |   netlistBlock 
        |   directiveBlock 
        |   block 
        |   variableDeclarationStatement 
        |   loopStatement 
        |   ifStatement 
        |   assignmentStatement 
        |   subtypesBlock 
        |   verilogBlock 
        |   typeDeclaration 
        |   channelDeclaration 
        |   aliasDeclaration 
        ;
    
    
prsBlock
        :                #( prs:PRS ( FRAGMENT  isFragment=true;  | ) ) 
        ;
    
    
subcellsBlock
        :                #( subcells:SUBCELLS 
                            ( inline:INLINE | fragment:FRAGMENT | ) ) 
        ;
    
    
cspBlock
        :                #( csp:CSP 
          ( cspPortsenv, cell | ) ) 
        ;
    
    
envBlock
        :                #( ENV ( envNameBlocknamedEnvBindings )* ) 
        ;
    
    
javaBlock
        :   oldJavaBlock 
        |   newJavaBlock 
        ;
    
    
netlistBlock
        :                #( netlist:NETLIST 
          ) 
        ;
    
    
directiveBlock
        :                #( directive:DIRECTIVE 
          ) 
        ;
    
    
block
        :                #( BLOCK bodyStatementListenv, false, cell, 
                                 envContainer, paramEnv, prsEnv,
                                 subcellsEnv, subtypesEnv ) 
        ;
    
    
variableDeclarationStatement
        :                #( VAR_DECL id:IDENT    
                             ( INLINE  inlineP = true;  | ) ( FLATTEN  flattenP = true;  | ) typeenv, cell, envContainer, HierName.makeHierName(varName),
                globalP, inlineP, flattenP, id, LOCAL_VAR ( variableInitializerenv | ) ) 
        ;
    
    
loopStatement
        :                #( LOOP id:IDENT rangeenv, true slist:BODY_STATEMENT_LIST ) 
        ;
    
    
ifStatement
        :                #( IF expressionenv, false slist:BODY_STATEMENT_LIST ) 
        ;
    
    
assignmentStatement
        :                #( ASSIGN exprenv, false ( expressionenv, false  l.add(vv); eList.add(ee); )* ) 
        ;
    
    
subtypesBlock
        :   SUBTYPES 
        ;
    
    
verilogBlock
        :                #( VERILOG ( verilogNamedenv, vb )* ) 
        ;
    
    
typeDeclaration
        :                #( TYPE_DEFINITION id:IDENT ( m:FORMALS ( p:FORMALS ( implied:FORMALS | ) | ) | ) ( ATTRIBUTES  isAttributeCell = true;  ( inheritanceList1:INHERITANCE_LIST | ) | ( inheritanceList2:INHERITANCE_LIST | ) ( refinement:REFINEMENT | ) ) b:BLOCK ) 
        ;
    
    
channelDeclaration
        :                #( DEFCHAN name:IDENT ( metas:FORMALS ( netGroup:FORMALS | ) | ) ( inheritanceList:INHERITANCE_LIST | ) ( refinement:REFINEMENT | ) body:BODY_STATEMENT_LIST ) 
        ;
    
    
aliasDeclaration
        :                #( DEFALIAS name:IDENT ( metas:FORMALS | ) refinement:REFINEMENT ) 
        ;
    
    
type
        :                #( TYPE baseTypeenv, env, cell, envContainer, varName, globalP,
                             declKind == META_PARAM ( arraySelectorenv, declKind != LOCAL_VAR
                             ? (ParamTypeInterface) pb.getSecond()
                             : (ParamTypeInterface) null,
                            declKind == META_PARAM | ) ( setFinalWidthenv, pb, ps ) ) 
        ;
    
    
variableInitializer
        :                #( ASSIGN expressionenv, false ) 
        |                #( PORT_LIST expressionListenv, true ( expressionListenv, true  v = new TupleGroupValue((TupleValue)v, implied);  | ) ) 
        ;
    
    
expression
        :                #( EXPRESSION exprenv, anonymousAllowedP, deepP ) 
        ;
    
    
expressionList
        :                #( EXPRESSION_LIST ( expressionenv, anonymousAllowedP, deepP  l.add(ve);  | ( id:IDENT rangeenv, true expr:EXPRESSION_LIST ) 
            )* ) 
        ;
    
    
setFinalWidth
        :                #( errorAST:CHANNEL_WIDTH ( expressionenv,false |  channelWidthValue = null;  ) ) 
        ;
    
    
baseType
        :                #( USER_TYPE id:IDENT expressionListmetaEnv, false ) 
        |   primitiveType 
        ;
    
    
arraySelector
        :                #( arrayExpr:ARRAY_SUB ( Range r; rangeenv, false 
            )* ) 
        ;
    
    
primitiveType
        :   BOOL 
        |   INT 
        |   FLOAT 
        |   NODE 
        ;
    
    
channelBody
        :                #( BODY_STATEMENT_LIST ( channelStatementenv, cell, aliasesEnv )* ) 
        ;
    
    
channelStatement
        :   aliasBlock 
        |   assertBlock 
        |   directiveBlock 
        ;
    
    
aliasBlock
        :   ALIAS 
        ;
    
    
assertBlock
        :   ASSERT 
        ;
    
    
assertStatements
        :                #( BODY_STATEMENT_LIST ( assertStatementenv, cell )* ) 
        ;
    
    
assertStatement
        :   assertPrsStatement 
        |   assertExclStatement 
        |   assertLoopStatement 
        |   assertIfStatement 
        ;
    
    
assertPrsStatement
        :   prsExpression 
        ;
    
    
assertExclStatement
        :   (                    #( hi:EXCLHI assertExclNodesenv  ) 
            |                    #( lo:EXCLLO assertExclNodesenv  ) 
            |                    #( cc:EXCLCC assertExclNodesenv  ) 
            |                    #( nocc:NOCC assertExclNodesenv  ) 
            ) 
            
        ;
    
    
assertLoopStatement
        :                #( LOOP id:IDENT rangeenv, true assertList:BODY_STATEMENT_LIST ) 
        ;
    
    
assertIfStatement
        :                #( IF expressionenv, false assertList:BODY_STATEMENT_LIST ) 
        ;
    
    
prsExpression
        :                #( PRS_EXPRESSION exprenv, false, deepP ) 
        ;
    
    
assertExclNodes
        :   ( expr )+ 
        ;
    
    
expr
        :                #( o01:OR exprenv, false, deepP exprenv, false, deepP ) 
        |                #( o02:AND exprenv, false, deepP exprenv, false, deepP ) 
        |                #( o03:NOT exprenv, false, deepP ) 
        |                #( oXOR:XOR exprenv, false, deepP exprenv, false, deepP ) 
        |                #( oExp:EXP exprenv, false, deepP exprenv, false, deepP ) 
        |                #( o04:PLUS exprenv, false, deepP exprenv, false, deepP ) 
        |                #( o05:MINUS exprenv, false, deepP exprenv, false, deepP ) 
        |                #( o06:TIMES exprenv, false, deepP exprenv, false, deepP ) 
        |                #( o07:DIV exprenv, false, deepP exprenv, false, deepP ) 
        |                #( o08:MOD exprenv, false, deepP exprenv, false, deepP ) 
        |                #( o09:UNARY_PLUS exprenv, false, deepP ) 
        |                #( o10:UNARY_MINUS exprenv, false, deepP ) 
        |                #( o11:LT exprenv, false, deepP exprenv, false, deepP ) 
        |                #( o12:LE exprenv, false, deepP exprenv, false, deepP ) 
        |                #( o13:EQ exprenv, false, deepP exprenv, false, deepP ) 
        |                #( o14:NE exprenv, false, deepP exprenv, false, deepP ) 
        |                #( o15:GE exprenv, false, deepP exprenv, false, deepP ) 
        |                #( o16:GT exprenv, false, deepP exprenv, false, deepP ) 
        |                #( o17:FIELD_ACCESS exprenv, false, deepP fi:FIELD_IDENT ) 
        |                #( o18:ARRAY_ACCESS exprenv, false, deepP arraySelectorenv, null, false ) 
        |   IDENT 
            (   expressionList 
            |   
            ) 
            
        |   NUM_INT 
        |   NUM_HEX 
        |   NUM_REAL 
        |   TRUE 
        |   FALSE 
        |                #( o21:ARRAY expressionListenv, anonymousAllowedP ) 
        |                #( o22:PRS_AND exprenv, false, deepP exprenv, false, deepP ) 
        |                #( o23:PRS_OR exprenv, false, deepP exprenv, false, deepP ) 
        |                #( o24:PRS_NOT exprenv, false, deepP ) 
        |                #( o25:LOOP_PRS_OR id2:IDENT rangeenv, true e2:PRS_EXPRESSION ) 
        |                #( o26:LOOP_PRS_AND id3:IDENT rangeenv, true e3:PRS_EXPRESSION ) 
        |                #( o27:LOOP_TIMES id4:IDENT rangeenv, true e4:EXPRESSION ) 
        |                #( o28:LOOP_PLUS id5:IDENT rangeenv, true e5:EXPRESSION ) 
        |                #( o28x:LOOP_XOR id6:IDENT rangeenv, true e6:EXPRESSION ) 
        |                #( o28o:LOOP_OR id7:IDENT rangeenv, true e7:EXPRESSION ) 
        |                #( o28a:LOOP_AND id8:IDENT rangeenv, true e8:EXPRESSION ) 
        |                #( o29:BOOL exprenv, false, deepP ) 
        |                #( o30:INT exprenv, false, deepP ) 
        |                #( o31:FLOAT exprenv, false, deepP ) 
        |                #( o32:NODE exprenv, false, deepP ) 
        |                #( o33:ALINT_FANIN alintFaninenv ) 
        ;
    
    
range
        :                #( RANGE expressionenv, false ( expressionenv, false | ) ) 
        ;
    
    
aliasStatements
        :                #( BODY_STATEMENT_LIST ( aliasStatementenv, chan, aliasesEnv )* ) 
        ;
    
    
aliasStatement
        :   variableDeclarationStatement 
        |   aliasIfStatement 
        |   aliasLoopStatement 
        ;
    
    
aliasIfStatement
        :                #( IF expressionenv, false aliasList:BODY_STATEMENT_LIST ) 
        ;
    
    
aliasLoopStatement
        :                #( LOOP id:IDENT rangeenv, true aliasList:BODY_STATEMENT_LIST ) 
        ;
    
    
metaParamList
        :   (                    #( f:FORMALS ( 
        
         metaParamenv, metaParams, tupleVal.accessTuple(i++) )* ) 
            |   
            ) 
            
        ;
    
    
metaParam
        :                #( VAR_DECL id:IDENT typeenv, null, null, HierName.makeHierName(id.getText()), false,
                false, false, id, META_PARAM ) 
        ;
    
    
portParamList
        :                #( FORMALS ( portParamenv, cell, portList, impliedPortParentNodes,
                             reverseDirection )* ) 
        ;
    
    
portParam
        :                #( VAR_DECL id:IDENT ( PLUS | ) ( MINUS | ) typeenv, cell, null, HierName.makeHierName(id.getText()), false,
                false, false, id, PORT_PARAM ( COLON id2:IDENT | ) ( envExtraPortIdentenv, buf | ) ) 
        ;
    
    
envExtraPortIdent
        :                #( id:IDENT 
          buf.append(id.getText());
       ) 
        |                #( FIELD_ACCESS envExtraPortIdentenv, buf fi:FIELD_IDENT ) 
        |                #( ARRAY_ACCESS envExtraPortIdentenv, buf   subtypesArraySelectorenv, buf  ) 
        ;
    
    
subtypesArraySelector
        :                #( ARRAY_SUB subtypesRangeenv, buf (   subtypesRangeenv, buf )* ) 
        ;
    
    
arrayAccessExpr
        :                #( o18:ARRAY_ACCESS exprenv, false arraySelectorenv, null, false ) 
        ;
    
    
alintFanin
        :   expr 
            (   NUM_INT 
            |   PLUS 
            |   MINUS 
            ) 
            
        ;
    
    
prsStatements
        :                #( BODY_STATEMENT_LIST ( prsStatementenv, cell )* ) 
        ;
    
    
prsStatement
        :   prsIfStatement 
        |   prsLoopStatement 
        |   prsAction 
        |   directiveBlock 
        |   variableDeclarationStatement 
        |   assignmentStatement 
        |   assertBlock 
        ;
    
    
prsIfStatement
        :                #( IF expressionenv, false prslist:BODY_STATEMENT_LIST ) 
        ;
    
    
prsLoopStatement
        :                #( LOOP id:IDENT rangeenv, true prslist:BODY_STATEMENT_LIST ) 
        ;
    
    
prsAction
        :   (                    #( ARROW prsActionBodyenv, cell, deepP ) 
            |                    #( celem:CELEM_ARROW prsActionBodyenv, cell, deepP ) 
            |                    #( comb:COMB_ARROW prsActionBodyenv, cell, deepP ) 
            ) 
            
        ;
    
    
prsActionBody
        :   (   ISOCHRONIC 
            |   
            ) 
            (   UNSTAB 
            |   METASTABLE 
            |   
            ) 
            (   TIMED 
                (   expression 
                    (   expression 
                    |   
                    ) 
                |   
                ) 
            |   
            ) 
            (   (   AFTER 
                |   AFTER_PS 
                ) 
                expression 
            |   
            ) 
            prsExpression expr 
            (   PLUS 
            |   MINUS 
            ) 
            
        ;
    
    
subcellsStatementsList
        :                #( BODY_STATEMENT_LIST subcellsStatementsenv, cell ) 
        ;
    
    
subcellsStatements
        :   ( subcellsStatement )* 
        ;
    
    
subcellsStatement
        :   variableDeclarationStatement 
        |   assignmentStatement 
        |   subcellsIfStatement 
        |   subcellsLoopStatement 
        |   assertBlock 
        |   directiveBlock 
        ;
    
    
subcellsIfStatement
        :                #( IF expressionenv, false subcellsList:BODY_STATEMENT_LIST ) 
        ;
    
    
subcellsLoopStatement
        :                #( LOOP id:IDENT rangeenv, true subcellsList:BODY_STATEMENT_LIST ) 
        ;
    
    
subtypesStatementsList
        :                #( BODY_STATEMENT_LIST subtypesStatementsenv, copied, cell ) 
        ;
    
    
subtypesStatements
        :   ( subtypesStatement )* 
        ;
    
    
subtypesStatement
        :   subtypesRefinement 
        |   subtypesIfStatement 
        |   subtypesLoopStatement 
        |   directiveBlock 
        ;
    
    
subtypesRefinement
        :                #( v:VAR_DECL ( inline:INLINE | ) subtypesIdentenv, copied, buf 
            typeenv, throwawayCell, null,
                         HierName.makeHierName("fake"),
                         false, false, false, null, LOCAL_VAR typeenv, throwawayCell, null,
                        HierName.makeHierName("fake"),
                        false, false, false, null, LOCAL_VAR ) 
        ;
    
    
subtypesIfStatement
        :                #( IF expressionenv, false subtypeStmts:BODY_STATEMENT_LIST ) 
        ;
    
    
subtypesLoopStatement
        :                #( LOOP id:IDENT rangeenv, true subtypeStmts:BODY_STATEMENT_LIST ) 
        ;
    
    
subtypesRange
        :   range 
        ;
    
    
subtypesIdent
        :                #( id:IDENT 
          final String s = id.getText();
          buf.append(s);

          next = new SimpleBinder(env, Symbol.create(s));
       ) 
        |                #( FIELD_ACCESS subtypesIdentenv, copied, buf fi:FIELD_IDENT ) 
        |                #( ARRAY_ACCESS subtypesIdentenv, copied, buf  buf.append('[');  subtypesArraySelectorenv, buf  buf.append(']');  ) 
        ;
    
    
cspPorts
        :                #( CSP_PORTS ( cspPortenv, cell )+ ) 
        ;
    
    
cspPort
        :   IDENT 
        ;
    
    
envNameBlock
        :                #( NAMED_ENV typeDeclarationnamedEnvBindings, true ) 
        ;
    
    
oldJavaBlock
        :                #( JAVA oldJavaStatementsenv, cell ) 
        ;
    
    
newJavaBlock
        :                #( j:NEW_JAVA newJavaStatements
           env,
           cell.getBlockInterface().iterator(BlockInterface.JAVA).next(),
           (JavaCoSimInfo) cell.getCoSimInfo(CoSimParameters.JAVA),
           isEnv
          ) 
        ;
    
    
newJavaStatements
        :                #( BODY_STATEMENT_LIST ( newJavaStatementenv, javaBlock, cosimInfo, isEnv )* ) 
        ;
    
    
newJavaStatement
        :   newJavaClassStatement 
        |   javaChannelStatement 
        |   directiveBlock 
        ;
    
    
newJavaClassStatement
        :                #( NEW_JAVA_CLASS className:IDENT ( javaClassMetaParametersenv | ) instanceName:IDENT ( javaClassParametersenv, channelAndNodeNames | ) ( javaClassInitializersenv | ) ) 
        ;
    
    
/**
 * Adds the channel to the environment and the cosimInfo.
 **/
javaChannelStatement
        :                #( CHANNEL id:IDENT ( arraySelectorenv, null, false | ) ( javaChannelTimingInfoenv | ) ) 
        ;
    
    
javaClassMetaParameters
        :                #( JAVA_CLASS_META_PARAMETERS ( javaClassMetaParameterenv  accum.add(metaParam);  )+ ) 
        ;
    
    
javaClassParameters
        :                #( JAVA_CLASS_PARAMETERS ( javaClassParameterenv  params.add(param);  )+ ) 
        ;
    
    
javaClassInitializers
        :   JAVA_CLASS_INITIALIZERS 
        ;
    
    
javaClassMetaParameter
        :                #( EXPRESSION exprenv, false 
             final String name;
             if (v.getInstanceName() == null)
                 name = null;
             else
                 name = v.getInstanceName().getAsString('.');
             metaParam = new MetaParamDefinition(
                     name,
                     ((MetaParamValueInterface) v).toMetaParam());
          ) 
        ;
    
    
javaClassParameter
        :                #( JAVA_CLASS_PARAMETER exprenv, false ( javaChannelTimingInfoenv | ) 
          ) 
        ;
    
    
javaChannelTimingInfo
        :                #( c:CHANNEL_TIMING_INFO exprenv, false ( 
          | exprenv, false exprenv, false exprenv, false exprenv, false 
          ) ) 
        ;
    
    
oldJavaStatements
        :                #( BODY_STATEMENT_LIST ( oldJavaStatementenv, cell )* ) 
        ;
    
    
oldJavaStatement
        :   oldJavaClassStatement 
        |   inputChannelStatement 
        |   outputChannelStatement 
        |   internalChannelStatement 
        ;
    
    
oldJavaClassStatement
        :                #( JAVACLASS cls:QuotedString ) 
        ;
    
    
inputChannelStatement
        :                #( INPUTCHAN cls:QuotedString ) 
        ;
    
    
outputChannelStatement
        :                #( OUTPUTCHAN cls:QuotedString ) 
        ;
    
    
internalChannelStatement
        :                #( INTERNALCHAN cls:QuotedString ) 
        ;
    
    
verilogNamed
        :   IDENT verilogStatements 
        ;
    
    
verilogStatements
        :                #( BODY_STATEMENT_LIST ( verilogStatementenv, nb )* ) 
        ;
    
    
verilogStatement
        :   IDENT 
            (   expressionList 
            |   
            ) 
            (   verilogParameters 
            |   
            ) 
            (   verilogFiles 
            |   
            ) 
            SEMI 
        |   verilogFiles SEMI 
        |   directiveBlock 
        ;
    
    
verilogParameters
        :   ( verilogParameter )+ 
        ;
    
    
verilogFiles
        :   ( QuotedString )+ 
        ;
    
    
verilogParameter
        :   verilogImplictParameter 
        |   verilogExplictParameter 
        ;
    
    
verilogImplictParameter
        :   expr 
        ;
    
    
verilogExplictParameter
        :   (   DOT_IDENT 
            |   ESCAPED_VERILOG_IDENT 
            ) 
            (   expr 
            |   
            ) 
            RPAREN 
        ;
    
    
partialExtract
        :   IDENT 
            (   expressionList 
            |   
            ) 
            ( 
                (   PLUS 
                |   MINUS 
                ) 
                (   PLUS 
                |   MINUS 
                |   
                ) 
                ( partialExtractExpr ) )* 
            (   IDENT 
                (   expressionList 
                |   
                ) 
            |   TIMES 
            |   
            ) 
            
        ;
    
    
partialExtractExpr
        :                #( FIELD_ACCESS partialExtractExprsb fi:FIELD_IDENT ) 
        |                #( ARRAY_ACCESS partialExtractExprsb  ) 
        |   IDENT 
        ;
    
    
partialExtractArraySelector
        :                #( arrayExpr:ARRAY_SUB partialExtractRangesb ( partialExtractRangesb )* ) 
        ;
    
    
partialExtractRange
        :                #( RANGE #( EXPRESSION n:NUM_INT   ) ) 
        ;
    
    
getHeaderAST
        :                #( VAR_DECL IDENT ( INLINE | ) ( FLATTEN | ) #( TYPE #( USER_TYPE id:IDENT metas:EXPRESSION_LIST ) ) ) 
        ;
    
    
getCellData
        :   IDENT 
            (   EXPRESSION_LIST 
            |   
            ) 
            (   IDENT 
                (   EXPRESSION_LIST 
                |   
                ) 
            |   
            ) 
            
        ;
\end{verbatim}



% ========================================
% INDEX (optional)
% ========================================
% \printindex


\end{document}
