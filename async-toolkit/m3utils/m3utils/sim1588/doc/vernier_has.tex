\documentclass{article}
\usepackage{graphicx}
\usepackage{epsf}

\usepackage{floatflt}
\usepackage{fancyhdr}
\usepackage{array}

\usepackage{sectsty}
%\allsectionsfont{\mdseries\sffamily}
\allsectionsfont{\mdseries\sc}

\usepackage{caption}
\captionsetup{margin=2pc,font=small,labelfont=bf}

%%%%%%%%%%%%%%Mika's figure macro%%%%%%%%%%%%%%%%%%%%%%%%%%%
\def\ins_epsfig#1#2#3{
  \begin{figure}[!tbph!]
  \bigskip
  \begin{center}
  \includegraphics[width=#1]{#2}
  \end{center}
  \caption{#3}\label{fig:#2}
  \bigskip
  \end{figure}
}

\def\rotins_epsfig#1#2#3#4{
  \begin{figure}[!tbph!]
  \bigskip
  \begin{center}
  \includegraphics[angle=#4,width=#1]{#2}
  \end{center}
  \caption{#3}\label{fig:#2}
  \bigskip
  \end{figure}
}

%%%%%%%%%%%%%%%%%%%%%%%%%%%%%%%%%%%%%%%%%%%%%%%%%%%%%%%%%%%%%%%%%%%%%%
\oddsidemargin=0.15in
\evensidemargin=0.15in
\topmargin=0.2in
\textwidth=6.2truein


\pagestyle{fancy}
\lhead{\scriptsize\bfseries\sffamily Vernier Timestamping Unit HAS 0.49}
\chead{}\rhead{\thepage}
\lfoot{}\cfoot{}\rfoot{}
\renewcommand{\headrulewidth}{0pt}
%%%%%%%%%%%%%%%%%%%%%%%%%%%%%%%%%%%%%%%%%%%%%%%%%%%%%%%%%%%%%%%%%%%%%%

\title{Vernier Timestamping Unit \\ High-Level Architecture Specification (HAS) 0.49}

%\date{January 22, 2018}
\date{\today}

\begin{document}

\maketitle


\parindent=0pt
\parskip=1.5ex

\arraycolsep=1.4pt\def\arraystretch{1.5}

\newpage

\section{Revision History}

\section{Contact}

Contact Mika Nystr\"om, {\tt mika.nystroem@intel.com} or 
Jin Yan,  {\tt jin2.yan@intel.com} for more information.

\section{Overview}

The Vernier Timestamping Unit (TSU) is a module that is used to
forward time measurements relative to a clock edge in one clock domain
$B$ to a time measurement relative to a clock edge in another clock
domain $A$.  The accuracy and precision of the TSU can be much better
than one clock cycle, given appropriate frequency (period) relationships
between the clock speeds.

The TSU is a soft macro and parameterizable.  It contains no memory or
register-file hard macros.  It relies on having synchronizer elements
available in the cell library, but otherwise is delivered as normal
SystemVerilog to be compiled with Design Compiler or similar.

\section{DFx Support}

Since the TSU is entirely a soft macro (apart from a number of
synchronizers), no support for DFx is provided.

\section{Timing Constraints}

Numerous clock-domain crossings are present in the TSU RTL.  Timing constraint
files are not provided by the RTL team as of the writing of this document.

\section{Theory of Operation}

See the report~\cite{too}.

\begin{thebibliography}{9}

\bibitem{nbs}{B.~E.~Blair, ed.  {\it Time and Frequency: Theory and
      Fundamentals.}  National Bureau of Standards Monograph 140.
    Boulder, Colorado:\ United States Department of Commerce, National
    Bureau of Standards, May 1974.}

\bibitem{too}{M.~Nystr\"om and J.~Yan.  Precise time-interval measurement using vernier clocks.  DCG DSD Mount Steller team, 2018.  Available from the authors.}
  
\end{thebibliography}

\end{document}
