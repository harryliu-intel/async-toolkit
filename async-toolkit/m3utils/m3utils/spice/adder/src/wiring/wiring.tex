% Copyright (c) 2025 Intel Corporation.  All rights reserved.  See the file COPYRIGHT for more information.
% SPDX-License-Identifier: Apache-2.0

\documentclass{article}
\usepackage{graphicx}
\usepackage{epsf}
\usepackage{epstopdf}

\usepackage{floatflt}
\usepackage{fancyhdr}
\usepackage{array}

\usepackage{xeCJK}
%\setCJKmainfont{SimSun}

\usepackage{sectsty}
%\allsectionsfont{\mdseries\sffamily}
\allsectionsfont{\mdseries\sc}

\usepackage{caption}
\captionsetup{margin=2pc,font=small,labelfont=bf}

%%%%%%%%%%%%%%Mika's figure macro%%%%%%%%%%%%%%%%%%%%%%%%%%%
\def\listcaption_ins_epsfig#1#2#3#4{
  \begin{figure}[!tbph!]
  \bigskip
  \begin{center}
  \includegraphics[width=#1]{#2}
  \end{center}
  \caption[#4]{#3}\label{fig:#2}
  \bigskip
  \end{figure}
}

\def\ins_epsfig#1#2#3{
  \begin{figure}[!tbph!]
  \bigskip
  \begin{center}
  \includegraphics[width=#1]{#2}
  \end{center}
  \caption{#3}\label{fig:#2}
  \bigskip
  \end{figure}
}

\def\rotins_epsfig_listcaption#1#2#3#4#5{
  \begin{figure}[!tbph!]
  \bigskip
  \begin{center}
  \includegraphics[angle=#4,width=#1]{#2}
  \end{center}
  \caption[#5]{#3}\label{fig:#2}
  \bigskip
  \end{figure}
}

%%%%%%%%%%%%%%%%%%%%%%%%%%%%%%%%%%%%%%%%%%%%%%%%%%%%%%%%%%%%%%%%%%%%%%
\oddsidemargin=0.15in
\evensidemargin=0.15in
\topmargin=0.2in
\textwidth=6.2truein


\pagestyle{fancy}
\lhead{\scriptsize\bfseries\sffamily DRAFT---INTEL CONFIDENTIAL---DRAFT}
\chead{}\rhead{\thepage}
\lfoot{}\cfoot{}\rfoot{}
\renewcommand{\headrulewidth}{0pt}
%%%%%%%%%%%%%%%%%%%%%%%%%%%%%%%%%%%%%%%%%%%%%%%%%%%%%%%%%%%%%%%%%%%%%%

\title{Wiring}
%\date{January 22, 2018}
\date{\today}

\begin{document}

\maketitle
\parindent=0pt
\parskip=1.5ex

\arraycolsep=1.4pt\def\arraystretch{1.5}

%\begin{abstract}
%Abstract.
%\end{abstract}

%\tableofcontents
%\listoffigures
%\listoftables


\newpage

Consider capacitance scaling with uniform scaling of all linear distances.

Assume chip/layer area is $ B $.

Wire capacitance

\[ C = k { A \over d } \]

where $A$ is the surface area of the wire and $d$ is the distance to
adjacent metal.

The area taken by the wire on the surface of the chip also goes as $A$.

The number of wires on a metal layer will be

\[ N = { B \over A } \quad . \]

Therefore the total capacitance of the wires on the layer goes as

\[ C_\mathrm{tot} = N C_\mathrm{wire} = { B \over A } k { A \over d } = k { B \over d } \]

Now replace all dimensions
\[ (w, s, l, h) \Rightarrow (\alpha w, \alpha s, \alpha l, \alpha h) \quad . \]

The chip area does not change, so the total capacitance of the layer becomes


\[ C'_\mathrm{tot} = k { B \over \alpha d } = { C_\mathrm{tot} \over \alpha } \quad . \]

So as $\alpha$ decreases, $C'_\mathrm{tot}$ increases linearly, or

\[ C'_\mathrm{tot} \propto {1 \over \alpha} \quad . \]





\end{document}
