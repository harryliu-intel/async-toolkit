\documentclass{article}
\usepackage{graphicx}
\usepackage{epsf}

\usepackage{floatflt}
\usepackage{fancyhdr}
\usepackage{array}

\usepackage{amsmath}
\usepackage{amssymb}

\usepackage{sectsty}
%\allsectionsfont{\mdseries\sffamily}
\allsectionsfont{\mdseries\sc}

\usepackage{caption}
\captionsetup{margin=2pc,font=small,labelfont=bf}

\usepackage{multicol}

\usepackage{pdflscape}

\usepackage{listings}
\usepackage{xcolor}
\usepackage{svg}

\usepackage[hidelinks]{hyperref}

\usepackage{underscore}


%%%%%%%%%%%%%%Mika's figure macro%%%%%%%%%%%%%%%%%%%%%%%%%%%
\def\listcaption_ins_epsfig#1#2#3#4{
  \begin{figure}[!tbph!]
  \bigskip
  \begin{center}
  \includegraphics[width=#1]{#2}
  \end{center}
  \caption[#4]{#3}\label{fig:#2}
  \bigskip
  \end{figure}
}

\def\ins_epsfig#1#2#3{
  \begin{figure}[!tbph!]
  \bigskip
  \begin{center}
  \includegraphics[width=#1]{#2}
  \end{center}
  \caption{#3}\label{fig:#2}
  \bigskip
  \end{figure}
}

\def\rotins_epsfig_listcaption#1#2#3#4#5{
  \begin{figure}[!tbph!]
  \bigskip
  \begin{center}
  \includegraphics[angle=#4,width=#1]{#2}
  \end{center}
  \caption[#5]{#3}\label{fig:#2}
  \bigskip
  \end{figure}
}

%%%%%%%%%%%%%%%%%%%%%%%%%%%%%%%%%%%%%%%%%%%%%%%%%%%%%%%%%%%%%%%%%%%%%%
\oddsidemargin=0.15in
\evensidemargin=0.15in
\topmargin=0.2in
\textwidth=6.2truein

\pagestyle{fancy}
\lhead{\scriptsize\bfseries\sffamily DRAFT---INTEL CONFIDENTIAL---DRAFT}
\lhead{}\chead{}\rhead{\thepage}
\lfoot{}\cfoot{}\rfoot{}
\renewcommand{\headrulewidth}{0pt}
%%%%%%%%%%%%%%%%%%%%%%%%%%%%%%%%%%%%%%%%%%%%%%%%%%%%%%%%%%%%%%%%%%%%%%



%%%%%%%%%%%%%%%%%%%%%%%%%%%%%%%%%%%%%%%%%%%%%%%%%%%%%%%%%%%%%%%%%%%%%%
\definecolor{codegreen}{rgb}{0,0.6,0}
\definecolor{codegray}{rgb}{0.5,0.5,0.5}
\definecolor{codepurple}{rgb}{0.58,0,0.82}
\definecolor{backcolour}{rgb}{0.95,0.95,0.92}

\lstdefinestyle{codeblock}{
    backgroundcolor=\color{backcolour},   
    commentstyle=\color{codegreen},
    keywordstyle=\color{magenta},
    numberstyle=\tiny\color{codegray},
    stringstyle=\color{codepurple},
    basicstyle=\ttfamily\footnotesize,
    breakatwhitespace=false,         
    breaklines=true,                 
    captionpos=b,                    
    keepspaces=true,                 
    numbers=left,                    
    numbersep=5pt,                  
    showspaces=false,                
    showstringspaces=false,
    showtabs=false,                  
    tabsize=2
}
%%%%%%%%%%%%%%%%%%%%%%%%%%%%%%%%%%%%%%%%%%%%%%%%%%%%%%%%%%%%%%%%%%%%%%












\graphicspath{{fig/}}

\title{CM3 Development}
\author{
  Mika Nystr\"om \\ {\tt mika.nystroem@intel.com} \\ 
  Kaixin Yang \\ {\tt kaixin.yang@intel.com} \\ 
}
%\date{January 22, 2018}
\date{\today}

\begin{document}

\maketitle
\parindent=0pt
\parskip=1.5ex

\arraycolsep=1.4pt\def\arraystretch{1.5}

% \begin{abstract}
% \end{abstract}

\tableofcontents
\listoffigures
\listoftables
\newpage



\section{Introduction}
The target is to migrate features implemented in the m3-utils repository 
to the public CM3 repository.

%\subsection{Modula-3}
%\subsection{CM3}



\section{Environment Setup}


\subsection{Compile Using Shell Scripts}

Release version: d5.11.4

Reference: 
\href{https://github.com/modula3/cm3/wiki/Getting-Started}
{GitHub Wiki - CM3 - Get Started}

This environment setup is based on the CM3 release version d5.11.4. 
The compiling steps introduced in the Wiki page, as shown above as the 
reference, work well in my WSL, which operates on Ubuntu 22.04.4 LTS.
However, I got many issues when compiling it on the Santa Clara EC 
running on SUSE Linux Enterprise Server 12 SP5. Below are the issue 
reports and solutions.

The default Shell on the EC is csh. Any user specified Shell scripts 
can be added to ``~/.cshrc.\{username\}''.

\subsubsection{technical issue \#1: cmake version}
The first issue I got is related to the version of cmake. 
Most of the executable binary files on the EC are under the paths 
/usr/bin/ and /usr/intel/bin/. CM3 cannot be compiled with the cmake 
under those paths. The cmake version is 3.5.2, which might be out-of-date 
for CM3. Therefore, I downloaded the source code of cmake-3.29.3, 
and compiled it locally. The compiled cmake is under the path 
/nfs/site/disks/zsc9_fwr_lib_char_010/kaixinya/local/cmake. 

\subsubsection{technical issue \#2: gcc linker}
I got an error during compilation: "collect2: fatal error: cannot find 'ld'".
This issue is related to gcc, g++, and ld. The compilation script uses 
the option ``-fuse-ld'' to specify the linker used during compilation. 
However, the default gcc on the EC is either too old (gcc 4.7.2 under 
the path ``/usr/intel/bin'') or broken and it does not support this option. 
So I have to turn to use gcc and g++ to the ones under the path 
``/usr/bin'' (version 4.8.5). There soft links are under a local folder 
``/nfs/site/disks/zsc9_fwr_lib_char_010/kaixinya/local/bin''.
The ``-fuse-ld'' option is set in the file 
``\{CM3\_root\}/m3-sys/cminstall/src/config-no-install/LINUX.common''. 
In the line 17, the ``LD_PREF'' is set as `` -fuse-ld=gold''.
To make the compilation successful, line 17 should be commented out 
and line 16 with an empty value should be enabled.

\subsubsection{environment variables}
Here is my personal cshrc configuration:

\lstset{style=codeblock}
\begin{lstlisting}
sccc16341901> tail -n2 ~/.cshrc.kaixinya
setenv PATH /nfs/site/disks/zsc9_fwr_lib_char_010/kaixinya/local/bin:$PATH
setenv PATH /nfs/site/disks/zsc9_fwr_lib_char_010/kaixinya/local/cmake/bin:$PATH

sccc16341901> ls -al /nfs/site/disks/zsc9_fwr_lib_char_010/kaixinya/local/bin
total 8
drwxr-s--- 2 kaixinya bzmi_fe 4096 Aug  5 16:48 .
drwxr-s--- 5 kaixinya bzmi_fe 4096 Aug 12 15:22 ..
lrwxrwxrwx 1 kaixinya bzmi_fe   12 Aug  5 16:20 g++ -> /usr/bin/g++
lrwxrwxrwx 1 kaixinya bzmi_fe   12 Aug  5 15:54 gcc -> /usr/bin/gcc
lrwxrwxrwx 1 kaixinya bzmi_fe   11 Aug  5 16:48 ld -> /usr/bin/ld

sccc16341901> ls -al /nfs/site/disks/zsc9_fwr_lib_char_010/kaixinya/local/cmake/bin
total 64044
drwxr-s--- 2 kaixinya bzmi_fe     4096 Jun  5 08:46 .
drwxr-s--- 5 kaixinya bzmi_fe     4096 Jun  5 08:46 ..
-rwxr-xr-x 1 kaixinya bzmi_fe 15571744 Jun  4 15:34 ccmake
-rwxr-xr-x 1 kaixinya bzmi_fe 15886960 Jun  4 15:34 cmake
-rwxr-xr-x 1 kaixinya bzmi_fe 16340512 Jun  4 15:34 cpack
-rwxr-xr-x 1 kaixinya bzmi_fe 17487912 Jun  4 15:34 ctest
\end{lstlisting}

When you run the installation command 
``../cm3-dist-AMD64_LINUX-d5.11.4/scripts/concierge.py install 
--prefix \{installation\_path\}'', it may remind you ``environment 
variable CM3_INSTALL not set AND cm3 not found in PATH.'' 
The reason is that the script finds both ``CM3_INSATLL'' and ``CM3'' are 
empty or not working. If you already solve the issues above, 
then you will find a binary file cm3 is already compiled under the path 
``\{installation\_path\}/bin'' specified by your ``--prefix'' option.
You can set the environment variables ``CM3_INSATLL'' and ``CM3'' 
to the folder ``\{installation\_path\}'' and the the binary file
``\{installation\_path\}/bin/cm3'' separately, and run the installation 
command again.

\lstset{style=codeblock}
\begin{lstlisting}
# set them in the command line or in the cshrc configuration file
sccc16341901> setenv CM3 {installation_path}/bin/cm3
sccc16341901> setenv CM3_INSTALL {installation_path}
\end{lstlisting}


\subsection{Compile Using Python Scripts}
I followed the instruction on the following page and worked with 
the version d5.11.4:
\href{https://github.com/modula3/cm3/releases/tag/cm3-d5.11.0-20210608}
{cm3 d5.11.0 release instructions}
There are basically four steps:
\lstset{style=codeblock}
\begin{lstlisting}
# Step 1: compile and generate a working cm3 copmiler
# but g++ failed for me, so I used another locally compiled cm3 compiler
wget https://github.com/modula3/cm3/releases/download/cm3-d5.11.0-20210608/cm3-boot-unix64le-d5.11.0-20210608.cpp.gz
gunzip cm3-boot-unix64le-d5.11.0-20210608.cpp.gz
g++ -g -pthread -c cm3-boot-unix64le-d5.11.0-20210608.cpp
g++ -g -pthread -o cm3 cm3-boot-unix64le-d5.11.0-20210608.o

# Step 2: optionally remove previous
# this should be related to the environment cleanup and setup, and I didn't do this
sudo rm -rf /cm3 
sudo mkdir -p /cm3/bin
sudo chown -R $USER /cm3
mv cm3 /cm3/bin

# Step 3: unzip the source code
wget https://github.com/modula3/cm3/archive/refs/tags/cm3-d5.11.0-20210608.tar.gz 
tar xf cm3-d5.11.0-20210608.tar.gz 
cd cm3-cm3-d5.11.0-20210608
rm -rf m3-sys/m3cc 
rm -rf m3-sys/m3gdb 
cd scripts/python
sudo apt-get install python2
export PATH=/cm3/bin:$PATH

# Step 4: build cm3
./boot2min.py c
./do-cm3-all.py buildship c 
\end{lstlisting}

Mika used the following test script to check whether some mathematical
functions defined as external functions from math.h, including 
cabs, frexp, modf, signgam, have been compiled correctly or not.
Here is the \href{https://github.com/modula3/cm3/issues/1178}
{issue report in GitHub}. We got an error mainly saying 
``undefined reference to `m3_modf' ''.

\lstset{style=codeblock}
\begin{lstlisting}
MODULE Main;
IMPORT IO;
IMPORT Math;
IMPORT Fmt;

VAR
  x, y : LONGREAL := 1.0d0;
  z : LONGREAL := Math.modf(x, y);
  t : LONGREAL := Math.cosh(x);

BEGIN

  IO.Put("Hello, World!\n");

  IO.Put("x = " & Fmt.LongReal(x) & "\n");

END Main.
\end{lstlisting}

Mika solved this issue by adding glue code into the 
``m3-libs/libm3/src/arith/POSIX'' folder.



\section{Feature Migration}

I already compiled cm3 with packages in the headless, caltech-parser, 
caltech-other categories.


\subsection{Yield/diesplit}

\subsubsection{Dependence}

\lstset{style=codeblock}
\begin{lstlisting}
diesplit
|-- mscheme (cm3/m3-scheme)
|   |-- scheme-lib (cm3/m3-scheme)
|-- modula3scheme (cm3/m3-scheme)
|   |-- rdwr (cm3; headless)
|-- sstubgen (m3utils)
|   |-- schemereadline (cm3/m3-scheme)
|   |   |-- schemesig (cm3/m3-scheme)
|-- integrate (m3utils)
|-- simpletoken (m3utils)
|-- mpfr (m3utils)
|-- minimize (m3utils)
|   |-- matrix (cm3; caltech-other)
\end{lstlisting}

Installation order:
\begin{enumerate}
  \item	Migrate simpletoken, integrate, mpfr, minimize, yield to 
        cm3/caltech-new.
  \item Add simpletoken, integrate, mpfr, minimize, diesplit to 
        pkginfo.txt.
  \item mscheme(contains sstubgen) is already under m3-scheme folder, 
        but sstubgen is not in the pkginfo.txt.
  \item Upgrade cm3 with scheme-lib, schemesig, mscheme, modula3scheme, 
        schemereadline, integrate, simpletoken, mpfr, minimize, sstubgen,
        diesplit in this order. (I already have rdwr and matrix compiled. 
        If not, please include them.)
\end{enumerate}

\subsubsection{Technical Issues}
\begin{enumerate}
  \item simpletoken: No .m3x and .M3WEB files. Solved by adding .m3 and 
        .i3 placeholder files.
  \item mpfr: Library mpfr is missing. Solved by removing the c file 
        from m3makefile.
  \item sstubgen: Missing schedulerwrap.tmpl. Solved by copying it from 
        m3tuils.
  \item sstubgen: Missing schemestubs_pll.tmpl and schemestubs_ser.tmpl 
        in the lib folder. Only one file is required based on the value
        of a environment variable CM3\_VERSION. Solved by adding 
        schemestubs_ser.tmpl as another template in the m3makefile,
        since currently CM3_VERSION is empty.
  \item sstubgen: The paths defined in the template file does not 
        match when compiling yield.
\end{enumerate}

% \section{Scripts}
% \subsection{concierge.py}
% \begin{itemize}
%   \item Functions \texttt{_load_package_sets()} and \texttt{_load_package_index()} in 
%   the \texttt{class PackageDatabase} are used to load package dependency and find 
%   available packages.
% \end{itemize}



% \begin{thebibliography}{99}  
% \end{thebibliography}

\end{document}
