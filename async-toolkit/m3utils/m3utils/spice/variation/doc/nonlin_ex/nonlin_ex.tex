\documentclass{article}
\usepackage{graphicx}
\usepackage{epsf}

\usepackage{floatflt}
\usepackage{fancyhdr}
\usepackage{array}

\usepackage{sectsty}
%\allsectionsfont{\mdseries\sffamily}
\allsectionsfont{\mdseries\sc}

\usepackage{caption}
\captionsetup{margin=2pc,font=small,labelfont=bf}

%%%%%%%%%%%%%%Mika's figure macro%%%%%%%%%%%%%%%%%%%%%%%%%%%
\def\listcaption_ins_epsfig#1#2#3#4{
  \begin{figure}[!tbph!]
  \bigskip
  \begin{center}
  \includegraphics[width=#1]{#2}
  \end{center}
  \caption[#4]{#3}\label{fig:#2}
  \bigskip
  \end{figure}
}

\def\ins_epsfig#1#2#3{
  \begin{figure}[!tbph!]
  \bigskip
  \begin{center}
  \includegraphics[width=#1]{#2}
  \end{center}
  \caption{#3}\label{fig:#2}
  \bigskip
  \end{figure}
}

\def\rotins_epsfig_listcaption#1#2#3#4#5{
  \begin{figure}[!tbph!]
  \bigskip
  \begin{center}
  \includegraphics[angle=#4,width=#1]{#2}
  \end{center}
  \caption[#5]{#3}\label{fig:#2}
  \bigskip
  \end{figure}
}

%%%%%%%%%%%%%%%%%%%%%%%%%%%%%%%%%%%%%%%%%%%%%%%%%%%%%%%%%%%%%%%%%%%%%%
\oddsidemargin=0.15in
\evensidemargin=0.15in
\topmargin=0.2in
\textwidth=6.2truein


\pagestyle{fancy}
\lhead{\scriptsize\bfseries\sffamily DRAFT---INTEL CONFIDENTIAL---DRAFT}
\chead{}\rhead{\thepage}
\lfoot{}\cfoot{}\rfoot{}
\renewcommand{\headrulewidth}{0pt}
%%%%%%%%%%%%%%%%%%%%%%%%%%%%%%%%%%%%%%%%%%%%%%%%%%%%%%%%%%%%%%%%%%%%%%

\title{An Example in Statistics}
\author{Mika Nystr\"om \\ {\tt mika.nystroem@intel.com} }

\date{\today}

\begin{document}

\maketitle
\parindent=0pt
\parskip=1.5ex

\arraycolsep=1.4pt\def\arraystretch{1.5}

\begin{abstract}
Abstract.
\end{abstract}

\tableofcontents
\listoffigures
\listoftables

\section{Background}

Consider the following problem, which is related to an important problem in
circuit design under manufacturing-process variability.

Let it be given that two variables, let us call them $x_0$ and $x_1$, are i.i.d.\ normal with $\mu=0$ and $\sigma=1$, so
\begin{equation}
   \begin{array}{rl}
    x_0 &~ N(0,1) \\
    x_1 &~ N(0,1) \\
           &= n_A   p_A - n_B   p_B + \mathcal{C}\quad .
    \end{array}
\end{equation}
where we write $N(0,1)$ for the p.d.f.\ of the standard 0,1-normal
distribution.

If we now imagine that $x_0$ and $x_1$ represent some physical
parameters of an electronic circuit that we are building in our fab,
$x_0$ and $x_1$ will affect the performance of the circuit in some
way.  Let us make this dependence fairly general and abstract, e.g.,
let us write for the performance simply $q$ so that
\begin{equation}
q : {\bf R \cross R} \rightarrow {\bf R}   
\end{equation}
so we can write the performance as the scalar function $q(x_0,x_1)$.
Without loss of generality, we require that lower values of $q$
represent better (more desirable) performance characteristics.

Let us make the following assumption on $q$: if we consider any
circular region around the origin. the extrema of $q(x_0,x_1)$ over
the region will be achieved on the boundary of the region.  This is
likely to be the case in practice, in particular for the {\em
  pessimum\/} (least desirable) value of $q$ over such a region.  Let us also
assume that $q$ is at least continuous over the region.

Beyond this we do not require much of $q$.  In particular, we do not
require that $q$ be convergent (to a finite value) over the entire
$x_0,x_1$ space, nor do we require that $q$ be differentiable.

Our question is how to characterize the distribution of $q$ under the
foregoing assumptions and distribution of $x_0$ and $x_1$.  Since
$q(x_0.x-1)$ is not required to be convergent, it is likely that the
mean of $q$ will also not converge.  However, metrics such as the
mode, median, and $n$th percentile will likely converge and be useful
for the practitioner.


We will generally show how to compute the p.d.f.\ of $q$, $\phi_q(q)$
or the corresponding c.d.f.\ $\Phi_q(q)$ where we have, as usual,
\begin{equation}
  \Phi_q(q_0) = \int_{-\infty}^{q_0} phi_q(x) \, dx \quad .
\end{equation}

\section{The Cumulative Distribution Function and the Maximum of Random Variables}

The c.d.f.\ is of particular interest when $q$ represents a delay
because in digital circuits a performance metric of much interest is the maximum
over a number of delays, since clock speeds (or even asynchronous maximum
operating speeds) are generally constrained by expressions of this form.

We have that if $\Phi_{r_i}(x)$ is the c.d.f.\ of one random variable,
and we have the c.d.f.s of a number of random variables $r_i$ where
say $i \in { 0, 1, \ldots , M - 1}$ and the variable $s$ is the max
over these $r_i$s so that
\begin{equation}
  s = \max ( r_0, r_1, \ldots r_{M-1} )
\end{equation}
then the c.d.f.\ of the maximum over these random variables is simply
the product of the variables' individual c.d.f.s:
\begin{equation}
  \Phi_s(x) = \Pi_0^{M-1} \Phi_{r_i}(x) \quad .
\end{equation}
This justifies our concentrating our efforts on computing the
c.d.f.\ of the distribution of $q$.

\subsection{Example: median of equal distributions}

Consider the restriction to $r_0, r_1, \ldots , r_{M-1} ~ \phi_r$, a single distribution with c.d.f.\ $\Phi_r$.  In this case,
\begin{equation}

  \max ( r_0, r_1, \ldots , r_{M-1}) ~ E
\end{equation}
where
\begin{equation}
  \Phi_E(x) = \Phi_r^M(x) \quad .
\end{equation}
Now consider the behavior of the median of $E$, $e^*$:
\begin{equation}
     \begin{array}{rl}
       {1 \over 2} &= \Phi_E(e^*) \\
       &= \Phi_r^M(e^*)
     \end{array}
\end{equation}
and therefore we have
\begin{equation}
  \Phi_r(e^*) = {1 \over \sqrt^M{2}} \quad .
\end{equation}

For a concrete example, assume $\phi_r = N(0,1)$.  In this case we have
\begin{equation}
  e^* = \erf^{-1} \left( { 1 \over \sqrt^{M-2}{2} } - 1\right) \quad .
  \end{equation}

     

\section{Approach and Formulation}

We introduce the complement of the c.d.f.\, called the {\em survival function\} as
  \begin{equation}
   \begin{array}{rl}
     S_q(x) &= 1 - \Phi_q(x) \\
           &= 1 - P(q \le x) \\
           &= P(q > x) \qquad .
     \end{array}
  \end{equation}
  




\begin{thebibliography}{99}

\bibitem{newuoas}{Zaikun Zhang.  On derivative-free optimization methods (in Chinese).  Ph.D. thesis, Chinese Academy of Sciences. Beijing, China: 2012.}

\bibitem{newuoas-http}{{\tt https://www.zhangzk.net/docs/talks/20160806-icnaao-newuoas.pdf}}

\bibitem{newuoa}{M.~J.~D.~Powell.  The NEWUOA software for unconstrained optimization without derivatives.  In {\it Nonconvex Optimization and Its Applications\/} (book series), NOIA {\bf 83}.  Springer-Verlag, 2006.}

\bibitem{brent}{R.~P.~Brent.  {\it Algorithms for Minimization Without Derivatives.} Prentice-Hall, 1972.}

\end{thebibliography}


\end{document}
