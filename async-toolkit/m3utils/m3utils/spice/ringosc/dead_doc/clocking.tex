\documentclass{article}
\usepackage{graphicx}
\usepackage{epsf}

\usepackage{floatflt}
\usepackage{fancyhdr}
\usepackage{array}

\usepackage{sectsty}
%\allsectionsfont{\mdseries\sffamily}
\allsectionsfont{\mdseries\sc}

\usepackage{caption}
\captionsetup{margin=2pc,font=small,labelfont=bf}

%%%%%%%%%%%%%%Mika's figure macro%%%%%%%%%%%%%%%%%%%%%%%%%%%
\def\listcaption_ins_epsfig#1#2#3#4{
  \begin{figure}[!tbph!]
  \bigskip
  \begin{center}
  \includegraphics[width=#1]{#2}
  \end{center}
  \caption[#4]{#3}\label{fig:#2}
  \bigskip
  \end{figure}
}

\def\ins_epsfig#1#2#3{
  \begin{figure}[!tbph!]
  \bigskip
  \begin{center}
  \includegraphics[width=#1]{#2}
  \end{center}
  \caption{#3}\label{fig:#2}
  \bigskip
  \end{figure}
}

\def\rotins_epsfig_listcaption#1#2#3#4#5{
  \begin{figure}[!tbph!]
  \bigskip
  \begin{center}
  \includegraphics[angle=#4,width=#1]{#2}
  \end{center}
  \caption[#5]{#3}\label{fig:#2}
  \bigskip
  \end{figure}
}

%%%%%%%%%%%%%%%%%%%%%%%%%%%%%%%%%%%%%%%%%%%%%%%%%%%%%%%%%%%%%%%%%%%%%%
\oddsidemargin=0.15in
\evensidemargin=0.15in
\topmargin=0.2in
\textwidth=6.2truein


\pagestyle{fancy}
\lhead{\scriptsize\bfseries\sffamily DRAFT---INTEL CONFIDENTIAL---DRAFT}
\chead{}\rhead{\thepage}
\lfoot{}\cfoot{}\rfoot{}
\renewcommand{\headrulewidth}{0pt}
%%%%%%%%%%%%%%%%%%%%%%%%%%%%%%%%%%%%%%%%%%%%%%%%%%%%%%%%%%%%%%%%%%%%%%

\title{Generating Multiphase Clocks \\ for Ultra-Low-Energy Computing}
\author{Mika Nystr\"om \\ {\tt mika.nystroem@intel.com} \\
Ganesh Iyer \\ {\tt ganesan.iyer@intel.com}}
%\date{January 22, 2018}
\date{\today}

\begin{document}

\maketitle
\parindent=0pt
\parskip=1.5ex

\arraycolsep=1.4pt\def\arraystretch{1.5}

\begin{abstract}
\end{abstract}

\tableofcontents
\listoffigures
\listoftables

\section{Background}

Many highly parallel algorithms imply a demand for what we can term
lowest-total-cost computing, where the designer's objective is to
minimize the sum of the silicon and energy costs of computing a
certain function.  Some such algorithms have the further property that
they are very dense in logic and therefore operate at relatively very
high power levels.  The optimal operating point of such algorithms in
CMOS technology is at a lower-than-standard operating voltage using
relatively short critical paths between state elements and also
specially designed low-power state elements.  Clock generation for
such hardware is difficult, partly because the headroom between the
required logic performance to be competitive and the potential
performance of clock logic is not that great (owing to the shallow
critical paths between state elements).  System considerations can add
further requirements, such as synchronous operation across large
silicon areas, where communications are kept local for simplicity
(e.g., systolic array computation).  The work described in this report
is in the context of cryptocurrency (Bitcoin) mining, which is
probably the most extreme application of cost-sensitive highly
parallel computing architectures with high compute intensity, but the
techniques are expected to be applicable broadly to problems with
similar requirements, such as custom AI training or inference hardware.

\subsection{Latch-based design}

A compelling design approach for minimum-cost computing
microarchitectures is the use of latch-based logic with multiphase
clocks, which is in fact a classic design style for CMOS
circuits\cite{m+c,g+d}.  In classic designs, clocks were arranged to
be non-overlapping but with as short inactive periods as possible.
This approach leads to static timing calculations that are difficult
to perform because clock borrowing is a natural feature of such
arrangements: when there are more than two clock phases, in
particular, all clocks except consecutive ones are generally set to
overlap.  More recent practice is more amenable to static timing
tools: these approaches use relatively narrow active clock pulses that
are exclusive of each other.  With narrow clock pulses, clock borrowing still
happens in practice, of course, but the static timing problem can be closed
in a conservative mode where we assume that no borrowing occurs; this makes
the static-timing problem feasible in practice while leaving some extra
performance margin in the fabricated chip.

(Insert some figures here: 1. multiphase clocking from Glasser and Dobberpuhl; 2. timing constraints in pulsed logic in FIN/FON/FWR.)

\subsection{Timing constraints for pulsed latches}

To review, (now-standard) flip-flop timing is based on the concept of
an {\em active clock edge,} from which setup and hold timing is
defined.  The clock edge itself is modeled to occur at an instant in
time, and inputs to sequential elements are required to remain stable
from the setup time instant to the hold time instant: we can call the
interval between the setup time instant and the clock edge the setup
interval and the interval between the clock edge and the hold time
instant the hold interval.  As a convention, the setup time is
positive for a setup time instant before the clock edge, and the hold
time is positive for a hold time instant after the clock edge.  A
negative setup time therefore means that the input data is allowed to
change up to and including a time after the active clock edge (and
will be registered), and a negative hold time conversely means that
the input data is allowed to change before the active clock edge
(without being registered).

Pulsed-latch timing is similar to flip-flop timing except that the
clock pulse is taken as having finite width, and the setup and hold
times are calculated conservatively based on the pulse width.  That
is, inputs are required to be ready (setup) at a time defined off the
leading edge of the clock pulse, outputs are expected to change starting from
the leading edge of the same pulse (as the latch becomes transparent at that time), and 



\subsection{Clock network design decisions}

(Table/figure here: illustrate OCV fraction of cycle/delay as function of supply voltage for 18A)

(Figure: pulse width as function of depth of network...?)












\section{Conclusion}


\section{Acknowledgements}


\begin{thebibliography}{99}

\bibitem{whitenoise} {D.~C.~Chu.  Time Interval Averaging: Theory,
  Problems, and Solutions.  {\it Hewlett-Packard Journal}, June 1974.}

\bibitem{hppatent}{D.~C.~Chu.  Double vernier time interval
  measurement using triggered phase-locked oscillators.
  U.S.~Patent~4,164,648.  Published August 14, 1979.}
      
\bibitem{vernier} {D.~C.~Chu, M.~S.~Allen, and A.~S.~Foster.
  Universal Counter Resolves Picoseconds in Time Interval
  Measurements.  {\it Hewlett-Packard Journal}, August 1978.}

\bibitem{euclid}{Euclid. {\it The Thirteen Books of the Elements,}
  vol.~2, Book~VII.  Translated by Sir~Thomas Heath.  Second edition,
  unabridged.  Dover Publications, 1956.  Reprint of edition by
  Cambridge University Press, 1908.}
  
\bibitem{hardy}{G.~H.~Hardy and E.~M.~Wright.  {\it An Introduction to
    the Theory of Numbers.}  Oxford:\ Clarendon Press, 1938.}

\bibitem{horowitz+hill} {P.~Horowitz and W.~Hill.  {\it The Art of
    Electronics}, second edition, chapter~15.  Cambridge University
  Press, 1989.}

\bibitem{leviton}{D.~B.~Leviton and B.~J.~Frey.  Temperature-dependent
  absolute refractive index measurements of synthetic fused silica.
  NASA Goddard Space Flight Center, 2008.}

\bibitem{spwm3}{G.~Nelson, ed.  {\it Systems Programming with
    Modula-3.}  Prentice-Hall Series in Innovative Technology.
  Prentice-Hall, 1991.}
  
\bibitem{newton}{I.~Newton.  {\it Philosophi\ae{} Naturalis Principia
    Mathematica,} Book 1.  London:\ S.~Pepys, Royal Society Press,
  1686.}

\bibitem{vernier-orig}{P.~Vernier.  {\it La construction, l'usage, et les propri\'et\'es du quadrant nouveau de math\'ematique.}  Brussels 1631.}

\bibitem{sv}{ANSI/IEEE 1800-2017.  {\it IEEE Standard for
    SystemVerilog-Unified Hardware Design, Specification, and
    Verification Language.}  American National Standards Institute,
  2017.}

\bibitem{g+d}{L.~A.~Glasser and D.~W.~Dobberpuhl. {\it The Design and Analysis of VLSI Circuits.} Addison-Wesley, 1985.}

\bibitem{m+c}{C.~A.~Mead and L.~A.~Conway.  {\it Introduction to VLSI Systems.}  Addison-Wesley, 1980.}
  
  

\end{thebibliography}

\end{document}
