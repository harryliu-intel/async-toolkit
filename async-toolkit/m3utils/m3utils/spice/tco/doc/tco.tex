\documentclass{article}
\usepackage{graphicx}
\usepackage{epsf}

\usepackage{floatflt}
\usepackage{fancyhdr}
\usepackage{array}

\usepackage{sectsty}
%\allsectionsfont{\mdseries\sffamily}
\allsectionsfont{\mdseries\sc}

\usepackage{caption}
\captionsetup{margin=2pc,font=small,labelfont=bf}

%%%%%%%%%%%%%%Mika's figure macro%%%%%%%%%%%%%%%%%%%%%%%%%%%
\def\listcaption_ins_epsfig#1#2#3#4{
  \begin{figure}[!tbph!]
  \bigskip
  \begin{center}
  \includegraphics[width=#1]{#2}
  \end{center}
  \caption[#4]{#3}\label{fig:#2}
  \bigskip
  \end{figure}
}

\def\ins_epsfig#1#2#3{
  \begin{figure}[!tbph!]
  \bigskip
  \begin{center}
  \includegraphics[width=#1]{#2}
  \end{center}
  \caption{#3}\label{fig:#2}
  \bigskip
  \end{figure}
}

\def\rotins_epsfig_listcaption#1#2#3#4#5{
  \begin{figure}[!tbph!]
  \bigskip
  \begin{center}
  \includegraphics[angle=#4,width=#1]{#2}
  \end{center}
  \caption[#5]{#3}\label{fig:#2}
  \bigskip
  \end{figure}
}

%%%%%%%%%%%%%%%%%%%%%%%%%%%%%%%%%%%%%%%%%%%%%%%%%%%%%%%%%%%%%%%%%%%%%%
\oddsidemargin=0.15in
\evensidemargin=0.15in
\topmargin=0.2in
\textwidth=6.2truein


\pagestyle{fancy}
\lhead{\scriptsize\bfseries\sffamily DRAFT---INTEL CONFIDENTIAL---DRAFT}
\chead{}\rhead{\thepage}
\lfoot{}\cfoot{}\rfoot{}
\renewcommand{\headrulewidth}{0pt}
%%%%%%%%%%%%%%%%%%%%%%%%%%%%%%%%%%%%%%%%%%%%%%%%%%%%%%%%%%%%%%%%%%%%%%

\title{How to Minimize the Total Cost of a Computation}
\author{Mika Nystr\"om \\ {\tt mika.nystroem@intel.com}}
%\date{January 22, 2018}
\date{\today}

\begin{document}

\maketitle
\parindent=0pt
\parskip=1.5ex

\arraycolsep=1.4pt\def\arraystretch{1.5}

\begin{abstract}
Abstract.
\end{abstract}

\tableofcontents
\listoffigures
\listoftables

\section{Introduction}

Consider the following problem.  It is given that we are to start with
a semiconductor fab, and electricity, and use these materials to
perform some specific computation.  The computation is of sufficient
importance that we are willing to design a custom semiconductor device
for it, and as in almost every design problem, we are confronted with
numerous design decisions that we must make before the computing equipment
is ready to enter service.  This report examines the decision process
that we ought to follow when making these decisions if we are to
complete our task at minimum total cost.

\section{The Cost of Computing}

As engineers, we are all familiar with the tradeoff between
performance and power consumption, often expressed in the terms ``good, fast,
cheap: choose any two.''  In circuit design, we are generally trading off
speed (frequency $f$) against energy consumption (energy per operation $E$).

Let us for a moment assume that the implementation cost (i.e., silicon
area, testing cost, packaging cost, etc.) for the alternative
implementations is the same.  Then it is convenient to plot the
relationship between a number of designs on a $E-f$ diagram as in
figure~\ref{fig:ef}.  Generally, the $E-f$ curve for a set of
alternative implementations of the same computation will slope up to
the right as in the figure, although we can certainly imagine that a
number of alternative designs are available as in
figure~\ref{fig:nonpareto}, but in that case we observe that any
design $A$ up and to the left of another design $B$ strictly dominates
$B$, because $A$ is both producing the results faster and at lower
energy cost.  If we are given a discrete set of continuous sets of
designs (e.g., such as when we make a discrete design decision and after that can vary a physical parameter such as a voltage or
temperature), we can follow the contour of the set of graphs up and to the left, calling that the {\em Pareto frontier.}




\section{Conclusion}

\begin{thebibliography}{99}


\end{thebibliography}

\end{document}
