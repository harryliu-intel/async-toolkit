\documentclass{article}
\usepackage{graphicx}
\usepackage{epsf}

\usepackage{floatflt}
\usepackage{fancyhdr}
\usepackage{array}

\usepackage{sectsty}
%\allsectionsfont{\mdseries\sffamily}
\allsectionsfont{\mdseries\sc}

\usepackage{caption}
\captionsetup{margin=2pc,font=small,labelfont=bf}

%%%%%%%%%%%%%%Mika's figure macro%%%%%%%%%%%%%%%%%%%%%%%%%%%
\def\ins_epsfig#1#2#3{
  \begin{figure}[!tbph!]
  \bigskip
  \begin{center}
  \includegraphics[width=#1]{#2}
  \end{center}
  \caption{#3}\label{fig:#2}
  \bigskip
  \end{figure}
}

\def\rotins_epsfig#1#2#3#4{
  \begin{figure}[!tbph!]
  \bigskip
  \begin{center}
  \includegraphics[angle=#4,width=#1]{#2}
  \end{center}
  \caption{#3}\label{fig:#2}
  \bigskip
  \end{figure}
}

%%%%%%%%%%%%%%%%%%%%%%%%%%%%%%%%%%%%%%%%%%%%%%%%%%%%%%%%%%%%%%%%%%%%%%
\oddsidemargin=0.15in
\evensidemargin=0.15in
\topmargin=0.2in
\textwidth=6.2truein


\pagestyle{fancy}
\lhead{\scriptsize\bfseries\sffamily Vernier Timestamping Unit HAS 0.49}
\chead{}\rhead{\thepage}
\lfoot{}\cfoot{}\rfoot{}
\renewcommand{\headrulewidth}{0pt}
%%%%%%%%%%%%%%%%%%%%%%%%%%%%%%%%%%%%%%%%%%%%%%%%%%%%%%%%%%%%%%%%%%%%%%

\title{Vernier Timestamping Unit \\ High-Level Architecture Specification (HAS) 0.49}

%\date{January 22, 2018}
\date{\today}

\begin{document}

\maketitle


\parindent=0pt
\parskip=1.5ex

\arraycolsep=0.4pt\def\arraystretch{1.4}

\newpage

\tableofcontents
\listoffigures
\listoftables

\section{Revision History}

Initial distributed revision 0.49.

\section{Contacts}

Contact Mika Nystr\"om, {\tt mika.nystroem@intel.com}, or 
Jin Yan,  {\tt jin2.yan@intel.com}, for more information.

\section{Overview}

The Vernier Timestamping Unit (TSU) is a module that is used to
forward time measurements relative to a clock edge in one clock domain
$B$ to a time measurement relative to a clock edge in another clock
domain $A$.  The accuracy and precision of the TSU can be much better
than one clock cycle, given appropriate frequency (period) relationships
between the clock speeds.

The TSU is a soft macro and parameterizable.  It contains no memory or
register-file hard macros.  It relies on having synchronizer elements
available in the cell library, but otherwise is delivered as normal
SystemVerilog to be compiled with Design Compiler or similar.

\section{DFx Support}

Since the TSU is entirely a soft macro (apart from a number of
synchronizers), no support for DFx is provided.

\section{Timing Constraints}

Numerous clock-domain crossings are present in the TSU RTL.  Timing
constraint files are not provided by the RTL team as of the writing of
this document.  Numerous signals cross between clock domains in the
TSU, so this is an important shortcoming.

\section{Theory of Operation}

See the report~\cite{too}.  The TSU is called ``delay-calculating
synchronizing flip-flop'' in the report.

\section{Basic Interface}

\subsection{Symbology}
Suggested symbology for the TSU is shown in figure~\ref{fig:meas_sync_ff}.

\ins_epsfig{3in}{meas_sync_ff}{Proposed symbol for the TSU.}

The functional interfaces are intentionally kept as simple as possible.

\subsection{Clocks}
The TSU has two clocks, an input-side clock called $B$ and an
output-side clock called $A$.

\subsection{Functional interfaces}
The TSU receives {\em mark\/} signals in the $B$ domain with
associated delay-deltas representing an ancestor event in the past.

The TSU produces mark signals in the $A$ domain with associated
delay-deltas appropriately updated for the delay in proceeding through
the internal synchronization of the TSU.  The updating of the delay
takes into account pipeline delay within the TSU as well as the phase
relationship of $A$ and $B$.  The main function of the TSU is to
calculate this latter delay, thus removing the timing uncertainty that
would obtain when using a normal synchronizer for a clock-domain
crossing.

\subsection{Configuration interfaces}
Configuration inputs of the TSU indicate the periods of the $A$ and $B$ clocks;
resets and handshake control inputs sequence the internal operations
of the TSU.

All delay measurements (input mark delay, output mark delay, and clock periods)
are given in some arbitrary but consistent time unit.  (This is a constraint
on the usage of the TSU.)

\subsection{Performance limitation}
Finally, the TSU has limited bandwidth: it is able to produce a mark
on the $A$ output at best once in two $A$ cycles.  For some
combinations of clock speeds, the constraint can deteriorate to a
maximum of one mark per four $A$ cycles.  

\subsection{Control handshake}
While the TSU is enabled to operate ({\tt i\_vernier\_en} is
asserted), errors will be signalled on {\tt o\_vernier\_error} as a nonzero value.  

\subsection{Error conditions}
Error
conditions are enumerated in table~\ref{tab:errs}.
{
  \begin{table}
    \centering
    \begin{tabular}{|>{\tt}l|>{\tt}c|>{\raggedright\arraybackslash}p{3in}|}
      \hline
      \bf Name & \bf Value & \bf Description \\
      \hline
ERR\_VERNIER\_OK & 3'h0 & no error \\
      \hline
ERR\_UNEXPECTED\_RESET & 3'h1 & assertion of $B$ clock reset during operation\\
      \hline
ERR\_MARK\_OVERFLOW & 3'h2 & overflow of mark buffer (attempts to send marks at a higher rate than one per divided $B$ cycle) \\
      \hline
ERR\_BAD\_DIVISOR & 3'h3 & attempt to command a bad $B$ clock divisor \\
      \hline
\bf ---reserved--- & 3'h4{\rm --}3'h7 & reserved \\
      \hline
    \end{tabular}
    \caption{Possible TSU error conditions.}\label{tab:errs}
    \end{table}
}

\section{Parameters}


The parameters listed in table~\ref{tab:params} are specified in
instantiating the TSU soft macro.

{
  \begin{table}
    \centering
    \begin{tabular}{|>{\tt}l|>{\tt}l|c|>{\raggedright\arraybackslash}p{3in}|}
      \hline
      \bf Name \rule{0pt}{2.6ex}& \bf type & \bf sugg. range & \bf Description \\
      \hline
          FCLK\_DIV\_BITS & int & 3--8 & $\log_2$ of maximum $B$-clock divisor \\
      \hline
          RAT\_PREC\_BITS & int & 16--64 & number of bits used to represent delays and timestamps \\
      \hline
          SAMPLE\_TIMES & int & 2--4 & depth of library synchronizer used in the $B$ to $A$ clock sampling path \\
      \hline
    \end{tabular}
    \caption{Elaboration-time parameters for TSU RTL.}\label{tab:params}
  \end{table}
}

\section{I/O Pins}
The I/O pinout of the TSU macro is given in table~\ref{tab:pinout}.

\subsection{Tolerance input}
The parameter {\tt i\_denom\_tol} deserves special mention.  It is
used to calculate the minimum allowable clock divider for the $B$
clock internally to the TSU.  There is a constraint that the divided
$B$ clock (not part of the interface) must be slower than half the
speed of the $A$ clock on {\em every\/} cycle.  So that the TSU can
calculate the required divider, the tolerance or allowable error in the $B$ clock
from its specification must be given as an input.  It is important to
note that this error must be the sum of all possible
rising-edge--to--rising-edge errors of $B$ relative to $A$.  That is,
if $A$ also has rising-edge--to--rising-edge error, this must be
included.  For example, consider a case where $A$ and $B$ are both
Ethernet clocks (100~p.p.m.\ error spec) with 0.2\% jitter budget.  In
this case, the worst-case error is 0.42\% (0.2\% from jitter in $B$,
0.2\% from jitter in $A$, and 100~p.p.m.~from each of $A$ and $B$) and
must be given as the clock period of $B$ times 0.0042, rounded up to
the nearest TU.  System performance depends only weakly on this parameter,
so it is OK to set it to 1\% of $B$'s period, or even more.

{
  \begin{table}
    \centering
    \begin{tabular}{|>{\tt}l|>{\tt}l|c|>{\it}c|>{\raggedright\arraybackslash}p{2.4in}|}
      \hline
      \bf Name \rule{0pt}{2.6ex}& \bf Width & \bf Dir & \bf Dom & {\bf Description} \\
      \hline
      \multicolumn{5}{|l|}{\bf CLOCKS AND RESETS}\\
      \hline
      clk & logic & C & -- & $A$ clock \\
      rst\_n & logic & I & A & $A$ domain power-on reset \\
      i\_fclk & logic & C & -- & $B$ clock \\
      i\_fclk\_rst\_n & logic & I & B & $B$ domain power-on reset \\
      \hline
      \multicolumn{5}{|l|}{\bf CONTROL HANDSHAKE} \\
      \hline
      i\_vernier\_en & logic & I & A & master enable signal \\
      o\_vernier\_ready & logic & O & A & TSU is ready and outputs are accurate \\
      o\_vernier\_error & 3-1:0 & O & A & If nonzero, TSU has detected an error and outputs are idle until the next initalization cycle \\
      \hline
      \multicolumn{5}{|l|}{\bf CONFIGURATION I/Os} \\
      \hline
      i\_num & RAT\_PREC\_BITS-1:0 & I & A & period of $A$ clock in TUs.  MRS while {\tt i\_vernier\_en} is asserted \\
      i\_denom & RAT\_PREC\_BITS-1:0 & I & A & period of $B$ clock in TUs.  MRS while {\tt i\_vernier\_en} is asserted \\
      i\_denom\_tol & RAT\_PREC\_BITS-1:0 & I & A & tolerance in period of $B$ clock (relative to $A$ clock) in TUs.  MRS while {\tt i\_vernier\_en} is asserted \\
      i\_fclk\_div & FCLK\_DIV\_BITS-1:0 & I & A & $B$ clock divisor.  Set to 0 for automatic selection.  MRS while {\tt i\_vernier\_en} is asserted \\
      o\_fclk\_div & FCLK\_DIV\_BITS-1:0 & O & A & $B$ clock divisor as computed.  WRS while {\tt o\_vernier\_ready} is asserted.\\
      o\_prec & RAT\_PREC\_BITS-1:0 & O & A & Measurement precision in TUs. WRS while {\tt o\_vernier\_ready} is asserted.\\
      \hline
      \multicolumn{5}{|l|}{\bf FUNCTIONAL INTERFACES} \\
      \hline
      i\_fclk\_mark & logic & I & B & indicated input mark \\
      i\_fclk\_mark\_phase & RAT\_PREC\_BITS-1:0 & I & B & indicated input event age in TUs \\
      o\_mark & logic & O & A & indicated output mark \\
      o\_mark\_phase & RAT\_PREC\_BITS-1:0 & O & A & indicated output event age in TUs \\
      \hline
    \end{tabular}
    \caption[Pinout of TSU RTL.]{Pinout of TSU RTL. MRS/WRS means ``must/will remain stable''}\label{tab:pinout}
  \end{table}
}

\section{Operational State Machine}

\subsection{Power-on reset}
At power-on, it is expected that the $A$ and $B$ clock domain resets are asserted as normal.

\subsection{Start handshake}
In order to start a vernier unit, configuration parameters ({\tt
  i\_num}, {\tt i\_denom}, etc.) are presented on the interface, the
environment must ensure that clocks (both $A$ and $B$) are stable and
operating, and {\tt i\_vernier\_en} is then asserted by the
environment.

After a number of cycles (which can be large, in the thousands
possibly), a normally functioning vernier unit responds by raising
{\tt o\_vernier\_ready}.  At this time, the environment is invited to
commence sending a stream of marks and mark delays in the $B$ clock
domain, which the vernier unit will convert into a stream of marks and
delays in the $A$ clock domain.  It is an error for the environment
to send marks on $B$ until {\tt o\_vernier\_ready} has been asserted by TSU.

\subsection{Error termination}
In case of any error's being detected, the TSU will raise {\tt
  o\_vernier\_error} and stop producing outputs.  If this happens, the
environment must go through a handshake with {\tt i\_vernier\_en} by
deasserting and reasserting that signal before performing more delay
conversions.  {\it It is not guaranteed that the TSU will catch every attempt
to provide it with erroneous input.}

\subsection{Orderly reconfiguration}
If the TSU is to be reconfigured by the environment, for example,
because of a change in the speed of $B$, the procedure is to lower
{\tt i\_vernier\_en}, which will cause {\tt o\_vernier\_ready} and
{\tt o\_vernier\_error} to drop immediately.  The environment can then
change the configuration inputs {\tt i\_num}, {\tt i\_denom}, etc., then
re-raise {\tt i\_vernier\_en} and await {\tt o\_vernier\_ready} as usual.
This procedure should also be followed if the environment needs to assert
the $B$ domain reset---if not, the TSU will go into its error state and
need to be re-initialized later (although it is important to note that
even a $B$ domain reset out of sequence with controls should {\em not\/}
require that the $A$ domain be reset for recovery).

\section{Performance}

All aspects of the performance of the TSU is related to the clock
periods $p_A={\tt i\_num}$ and $p_B={\tt i\_denom}$ as well as the $B$-clock divisor $d={\tt o\_fclk\_div}.$

\subsection{$B$-clock divisor}

The $B$ clock is divided down
such that it is slower than twice the period of the $A$ clock.  We call
the divisor $d$.

\subsubsection{Explicit specification of divisor}
The divisor can either be specified by the user as {\tt i\_fclk\_div} or computed automatically.  Specifying a divisor that is too small or odd (other than 1) causes the TSU to stop in an error state.

\subsubsection{Automatic algorithm for divisor}
The automatic
algorithm for computing the divisor is as follows:

Choose $d$ as divisor such that out of the set

\begin{equation}
{\mathcal S} = \{ 1, 2, 4, \ldots , 2^{FCLK\_DIV\_BITS}-2\}
\end{equation}
it is the smallest member  $d \in {\mathcal S}$ that satisfies
\begin{equation}\label{eq:divconstraint}
  2 \, {\tt i\_num} < ( {\tt i\_denom} - {\tt i\_denom\_tol} ) d
\end{equation}
If there is no smallest member $d \in {\mathcal S}$ that satisfies
equation~\ref{eq:divconstraint}, the TSU does not initialize and
instead signals the condition on {\tt o\_vernier\_error}.

\subsection{Initialization time}
The initialization time is the time from {\tt i\_vernier\_en} asserted
to {\tt o\_vernier\_ready} asserted.  

The initialization time is specified to be
\begin{equation}
  t_{init} \le LCM(p_A,d\,p_B) + 2^{\tt FCLK\_DIV\_BITS}+10\, {\tt RAT\_PREC\_BITS} \, p_A + {\tt SAMPLE\_TIMES}\, (2d\,p_B + 2p_A)\quad .
\end{equation}
This expression for an upper bound on the initialization time arises
from the following sources:
\begin{itemize}
\item Settling time of the vernier algorithm
\item Search for $B$ clock divisor (note that this only takes as long
  as required so if a smaller divisor is sufficient, this term shrinks
  accordingly).  Also skipped if {\tt i\_fclk\_div}$\,\neq 0$.
\item Computation of various constants (using a bit-at-a-time divider)
\item Synchronization time for initialization handshake between $A$
  and $B$ clock domains (estimated at two complete handshakes between
  the clock domains)
\end{itemize}

\section{Precision}
The measurement precision of the TSU is
\begin{equation}
  \delta = GCD(p_A,d\,p_B)
\end{equation}
in TUs.  This quantity is output by the TSU as {\tt o\_prec}.

It is clear from the above that the precision $\delta$ may be
dependent on the choice of $d$.  If $d$ and $p_A$ have factors in
common, the precision will be degraded.  There are two general methods
of avoiding this issue: if $A$ is fast enough (more than twice as fast as $B$), we will have $d=1$, and
otherwise, in performance-critical applications, the user can specify
a suitably prime $d$ on the input {\tt i\_fclk\_div}.

\subsection{Accuracy}
The measurement accuracy of the TSU is
\begin{equation}
  \epsilon = \delta + e_{drift,A}\xi + e_{drift,B}\xi + \iota_A + \iota_B
\end{equation}
where $\delta$ is the precision and $e_{drift}$ is the relative
maximum drift error (e.g., 100~p.p.m.~for a standard Ethernet clock),
and $\xi$ is the vernier settling component of the initialization time
equal to $LCM(p_A,p_B)$ TUs, and $\iota$ is the clock jitter.  (When
performing the summation, ensure all terms have the same units!)

\subsection{Mark Throughput}\label{sec:divchoice}
The throughput of marks is strictly limited.  

Once the divided $B$ clock is transferred into the $A$ clock domain, at most
one mark per divided $B$ clock can be accepted.

For a high-enough $A$ clock speed, e.g., $A$ at 850~MHz and $B$ at
390~5/8~MHz, the $B$ divisor is~1, and the TSU is capable of servicing $B$
marks at one per $B$ cycle.

\begin{thebibliography}{9}

\bibitem{nbs}{B.~E.~Blair, ed.  {\it Time and Frequency: Theory and
      Fundamentals.}  National Bureau of Standards Monograph 140.
    Boulder, Colorado:\ United States Department of Commerce, National
    Bureau of Standards, May 1974.}

\bibitem{too}{M.~Nystr\"om and J.~Yan.  Precise time-interval
  measurement using vernier clocks.  DCG DSD Mount Steller team, 2018.
  Available from the authors.}
  
\end{thebibliography}

\end{document}
